\documentclass{book}
\usepackage[utf8]{inputenc}
\usepackage{verbatim} % for comments
\usepackage[margin=0.5in]{geometry} % for margins
\usepackage{amsmath} % for math
\usepackage{amssymb} % for math
\usepackage{latexsym} % for math
\usepackage{mathtools} % for math
\usepackage{mathrsfs} % for math
\usepackage{cancel} % for math

\usepackage{mleftright} % for left right
\mleftright
\newcommand{\is}{:\equiv}
\newcommand{\pnot}[1]{\widetilde{#1}}
\newcommand{\inot}{\not}
\newcommand{\txtand}{\left(\text{ and }\right)}
\newcommand{\txtiff}{\left(\text{ iff }\right)}
\newcommand{\txtforall}[2]{\left(\text{ for any }#1\right)\text{, }\left(#2\right)}
\newcommand{\txtexists}[2]{\left(\text{ there exists }#1\right)\text{, }\left(#2\right)}
\newcommand{\txtif}[2]{\text{ if }\left(#1\right)\text{, then }\left(#2\right)}
\newcommand{\contr}{\overleftarrow{\bot}}

\newcommand{\occurs}[2]{occurs(#1, #2)}
\newcommand{\free}[2]{free(#1, #2)}

\newcommand{\means}[2]{#1^{#2}}
\newcommand{\modvar}[2]{\left[#1\middle|#2\right]}
\newcommand{\extend}[1]{\overline{#1}}
\newcommand{\placeholder}{\square}

\newcommand{\grp}[1]{\left( #1 \right)}
\newcommand{\set}[1]{\left\{ #1 \right\}}
\newcommand{\seq}[1]{\left\langle #1 \right\rangle}
\newcommand{\vdc}[3]{\underset{#2}{\overset{#3}{\fbox{$#1$}}}}
\newcommand{\psx}[3]{\underset{#2}{\overset{#3}{#1}}}
\newcommand{\closedSet}[2]{closed(#1, #2)}
\newcommand{\closure}[2]{closure(#1, #2)}
\newcommand{\bound}[2]{bound(#1, #2)}
\newcommand{\occursAll}[2]{occurs_\forall(\set{#1}, #2)}
\newcommand{\freeAll}[2]{free_\forall(\set{#1}, #2)}
\newcommand{\boundAll}[2]{bound_\forall(\set{#1}, #2)}
\newcommand{\openWff}[2]{open(#1, #2)}
\newcommand{\closedWff}[2]{closed(#1, #2)}
\newcommand{\sub}[3]{\left|#1\right|_{#3}^{#2}}

%\usepackage{xcolor}
%\usepackage{pagecolor} % $ tlmgr install pagecolor
%\pagecolor{black}
%\color{white}

\title{Mathematical Logic Notes}
\author{John Paul Guzman}
\date{ }
\begin{document}
\maketitle
%\tableofcontents

================= META
Relation / Rel(R, S): $R \subseteq S$ and for any $r_1, r_2$, if $r_1 = r_2$, then $r_1 \in R$ iff $r_2 \in R$ 
Function / Func(f, X, Y): For any $x_1, x_2 \in X$, if $\seq{x_1, y_1}, \seq{x_2, y_2} \in f$ and $x_1 = x_2$, then $y_1 = y_2$
If $Function / Func(f, X, Y)$, then $Rel(f, X \times Y)$
Injection / Inj(f, X, Y): $Func(f, X, Y)$ and for any $x_1, x_2 \in X$, if $f(x_1) = f(x_2)$, then $x_1 = x_2$
Surjection / Sur(f, X, Y): $Func(f, X, Y)$ and for any $y \in Y$, there exists $x \in X$, $y = f(x)$
Bijection / Bij(f, X, Y): $Inj(f, X, Y)$ and $Sur(f, X, Y)$

COUNTABLE SETS

IMPLICATION EQUIVALENCES

Implication definition
	- If A, then B
		-- Not A or B

Implication over conjunction
	- If A and B, then C
		-- If A, then if B, then C

Contraposition
	- If A, then B
		-- If not B, then not A

Metaproof by contradictory
	- If not C, then CONTR
		-- If TAUT, then C
		-- C
=================

\chapter{Structures and Languages}
	===================================================================================================================
\section{Languages}
	===================================================================================================================
\subsection{(Definition) First-order Alphabet} %=1.2.1
	- The first-order alphabet ($\mathcal{L}$) is a tuple of collections of symbols that consists: \\
		-- Connectives: $\lor, \lnot$ \\
		-- Quantifier: $\forall$ \\
		-- Variables: $Var = \set{\vdc{v_i}{i \in \mathbb{N}}{ }}$ \\
		-- Equality: $\equiv$ \\
		-- Constants $Const$ \\
		-- Functions: $Func = \set{\vdc{f: Arity(f) = i}{i \in \mathbb{N}}{ }}$ \\
		-- Relations: $Rel = \set{\vdc{P: Arity(P) = i}{i \in \mathbb{N}}{ }}$ \\
		-- FOS: $\lor, \lnot, \forall, \equiv$ \\
	===================================================================================================================

	\section{Terms and Formulas}
	===================================================================================================================
\subsection{(Definition) Term} %=1.3.1
	- The term $t$ of the language $\mathcal{L}$ ($t \in Term(\mathcal{L})$) iff $t$ is a non-empty finite string and it satisfies exactly one of the following: \\
		-- $t \is v$ and $v \in Var$ \\
		-- $t \is c$ and $c \in Const$ \\
		-- $t \is f \vdc{t_i}{i=1}{Arity(f)}$ and $\set{\vdc{t_i}{i=1}{Arity(f)}} \subseteq Term(\mathcal{L})$* and $f \in Func$ \\
	- Terms encode the objects or nouns in the language \\
	===================================================================================================================
\subsection{(Definition) Formula} %=1.3.3
	- The formula $\phi$ of the language $\mathcal{L}$ ($\phi \in Form(\mathcal{L})$) iff $\phi$ is a non-empty finite string and it satisfies exactly one of the following: \\
		-- $\phi \is \equiv r s$ and $\set{r, s} \subseteq Term(\mathcal{L})$ \\
		-- $\phi \is R \vdc{t_i}{i=1}{Arity(R)}$ and $\set{\vdc{t_i}{i=1}{Arity(R)}} \subseteq Term(\mathcal{L})$ and $R \in Pred$ \\
		-- $\phi \is \lnot \alpha$ and $\alpha \in Form(\mathcal{L})$* \\
		-- $\phi \is \lor \alpha \beta$ and $\set{\alpha, \beta} \subseteq Form(\mathcal{L})$* \\
		-- $\phi \is \forall v \alpha$ and $\alpha \in Form(\mathcal{L})$* and $v \in Var$ \\
	- Formulas encode the statements or assertions in the language \\
	- Non-recursive definitions are called atomic formulas ($\phi \in AF(\mathcal{L})$) \\
	===================================================================================================================
\subsection{(Definition) Scope} %=1.3.3
	- The the scope of the quantifier $scope(\phi, \alpha) \is \alpha$ if $\phi \is \forall v \alpha$ \\
	- The symbols in $\alpha$ lies within the scope of $\forall$ \\
	===================================================================================================================

\section{Induction and Recursion}
	===================================================================================================================
\subsection{(Definition) Definition by recursion} %=1.4.1
	- The set $S$ is the (recursively defined) closure of the set $J$ under the set of rules $Q$ ($Cl(S, J, Q)$) iff $S$ is the smallest set that satisfies all of the following: \\
		-- $J \subseteq S$ \\
		-- For any $R \in Q$, for any $\seq{\vdc{s_i}{i=1}{ArityR(R)-1}, s} \in R$, if $\set{\vdc{s_i}{i=1}{ArityR(R)-1}} \subseteq S$, then $s \in S$ \\
	- In recusive definitions, the '$x \in X$ iff $P(x)$' qualifier is logically equivalent to the '$X \subseteq \set{y: P(y)}$' qualifier because \\
		-- For any $z$, $\pnot{P(z)}$ iff $z \inot \in X$ as well \\
		-- Therefore $X$ has to be the smallest set that satisfies $P$ \\
	===================================================================================================================
\subsection{(Metatheorem) Proof by induction on structure} %=1.4.1
	- If $J \subseteq \set{x: P(x)}$ and for any $R \in Q$, for any $\seq{\vdc{s_i}{i=1}{ArityR(R)-1}, s} \in R$, (if $\set{\vdc{s_i}{i=1}{ArityR(R)-1}} \subseteq \set{x: P(x)}$, then $s \in \set{x: P(x)}$), then $S_{J, Q} \subseteq \set{x: P(x)}$ \\
	- Proof: $S_{J, Q} \subseteq \set{x: P(x)}$ from (definition of $S_{J, Q}$: satisfies the qualifier of smallest set) \\
	===================================================================================================================
\subsection{(Metatheorem) Proof by induction on complexity} %=1.4.2
	- If $J \subseteq \set{x: P(x)}$ and (if $stage(J, Q, n) \subseteq \set{x: P(x)}$, then $stage(J, Q, n+1) \subseteq \set{x: P(x)}$), then $S_{J, Q} \subseteq \set{x: P(x)}$ \\
	- BACKLOG: PROPERLY DEFINE STAGE AND SAY STAGE = CLOSURE AND PROOF!!! \\
	===================================================================================================================
\subsection{(Definition) Initial segment} %=1.4.1.6
	- The string $s$ is an initial segment of the string $t$ ($IS(s, t)$) iff there exists the string $u \inot \is \epsilon$, $t \is s u$ \\
	===================================================================================================================
\subsection{(Metatheorem) Initial segments of terms} %=1.4.1.6
	- For any $s \in Term(\mathcal{L})$, for any $t \in Term(\mathcal{L})$, $\pnot{IS(s, t)}$ \\
	- Proof: \\
		-- $Term(\mathcal{L})_J \subseteq \set{s \in Term(\mathcal{L})_J: \txtforall{t \in Term(\mathcal{L})}{\pnot{IS(s, t)}}}$ \\
			--- If $s \is x \in Var \cup Const$, then \\
				---- If $t \is z \in Var \cup Const$, then $\pnot{IS(s, t)}$ from \\
					----- If $IS(s, t)$, then \\
						------ $t \is s u$ \\
						------ $x \is z u$ \\
						------ $x \is z$ \\
						------ $u \is \epsilon$ \\
						------ $u \inot \is \epsilon$ \\
						------ CONTRADICTION
				---- If $t \is f \vdc{t_i}{i=1}{Arity(f)}$, then $\pnot{IS(s, t)}$ from \\
					----- If $IS(s, t)$, then \\
						------ $t \is s u$ \\
						------ $f \vdc{t_i}{i=1}{Arity(f)} \is x u$ \\
						------ $f \is x$ \\
						------ $f \inot \is x$ \\
						------ CONTRADICTION
		-- $Term(\mathcal{L})_Q$ closed in $\set{s \in Term(\mathcal{L})_J: \txtforall{t \in Term(\mathcal{L})}{\pnot{IS(s, t)}}}$ \\
			--- If $s \is f \vdc{t_i}{i=1}{Arity(f)}$ and $f \in Func$ and $\set{\vdc{t_i}{i=1}{Arity(f)}} \subseteq Term(\mathcal{L})$ and \\
			--- For any $t_i \in \set{\vdc{t_i}{i=1}{Arity(f)}}$, for any $r \in Term(\mathcal{L})$, $\pnot{IS(t_i, r)}$, then \\
				---- If $t \is z \in Var \cup Const$, then $\pnot{IS(s, t)}$ from \\
					----- If $IS(s, t)$, then \\
						------ $t \is s u$ \\ <[1] (HYP: $IS(s, t)$)>
						------ $z \is f \vdc{t_i}{i=1}{Arity(f)} u$ \\ <[2] (HYP) on [1]>
						------ $z \is f$ \\ <[3] (DEF: Alphabet, String Concat) on [2]>
						------ $z \inot \is f$ \\ <[4] (DEF: Alphabet) on [3]>
						------ CONTRADICTION [3, 4] \\
				---- If $t \is f' \vdc{t'_i}{i=1}{Arity(f)} \in Term(\mathcal{L})$, then \\
					----- If $IS(s, t)$, then \\
						------ $t \is s u$ \\ <[1] (HYP: $IS(s, t)$)>
						------ $f' \vdc{t'_i}{i=1}{Arity(f)} \is f \vdc{t_i}{i=1}{Arity(f)} u$ \\ <[2] (HYP) on [1]>
						------ $f' \is f$ \\ <[3] (DEF: Alphabet, String Concat) on [2]>
						------ $\vdc{t'_i}{i=1}{Arity(f)} \is \vdc{t_i}{i=1}{Arity(f)} u$ \\ <[4] (DEF: String Concat) on [3]>
						------ For $i \in \mathbb{N}_1^{Arity(f)}$, $t'_i \is t_i$ from \\ <[5] (Induction) on [4]>
							------- If $t'_i \inot \is t_i$, then \\
								-------- $IS(t_i, t'_i)$ \\ <[5.1] (HYP: $t'_i \inot \is t_i$) on [4]> 
								-------- $\pnot{IS(t_i, t'_i)}$ \\ <[5.2] (HYP: $\pnot{IS(t_i, r)}$)>
								-------- CONTRADICTION [5.1, 5.2] \\
						------ $\vdc{t'_i}{i=1}{Arity(f)} \is \vdc{t_i}{i=1}{Arity(f)}$ \\ <[6] (DEF: String Concat) on [5]>
						------ $u \is \epsilon$ \\ <[7] (DEF: String Concat) on [6]>
						------ $u \inot \is \epsilon$ \\ <[8] (HYP: $IS(s, t)$)>
						------ CONTRADICTION [7, 8] \\
					----- $\pnot{IS(s, t)}$ \\
	===================================================================================================================
\subsection{(Metatheorem) Unique readability of terms} %=1.4.1.7
	- For any $t \in Term(\mathcal{L})$, it satisfies exactly one of the following: \\
		-- $t \is v \in Var$ and $v$ is unique \\
		-- $t \is c \in Const$ and $c$ is unique \\
		-- $t \is f \vdc{t_i}{i=1}{Arity(f)}$ and $f \in Func$ is unique and for any $i \in \set{\vdc{i}{i=1}{Arity(f)}}$, $t_i \in Term(\mathcal{L})$ is unique \\ 
	- Proof: \\
		-- If $t \in Var$, then variables are unique, then $t$ is unique \\
		-- If $t \in Const$, then constants are unique, then $t$ is unique \\
		-- If $t \is f \vdc{t_i}{i=1}{Arity(f)}$, then \\
			--- If $t \is f' \vdc{t'_i}{i=1}{Arity(f')}$, then \\
				---- $f \is f'$ \\
				---- $f \vdc{t_i}{i=1}{Arity(f)} \is f \vdc{t'_i}{i=1}{Arity(f)}$ \\
				---- $\vdc{t_i}{i=1}{Arity(f)} \is \vdc{t'_i}{i=1}{Arity(f)}$ \\
				---- If $Arity(f) = 1$ and $t_1 \inot \is t'_1$, then $IS(t_1, t'_1)$ or $IS(t'_1, t_1)$, then CONTRADICTION \\
				---- If $Arity(f) > 1$ and for any $i \in \set{\vdc{i}{i=1}{n-1}}$, $t_i \is t'_i$ and $t_n \inot \is t'_n$, then $IS(t_n, t'_n)$ or $IS(t'_n, t_n)$, then CONTRADICTION \\
				---- For any $i \in \set{\vdc{i}{i=1}{Arity(f)}}$, $t_i \is t'_i$ \\
	===================================================================================================================
\subsection{(Metatheorem) Initial segments of formulas} %=1.4.1.7
	- BACKLOG: \\
	===================================================================================================================
\subsection{(Metatheorem) Unique readability of formulas} %=1.4.1.8
	- BACKLOG:  \\
	===================================================================================================================
\subsection{(Definition) Language of Number theory} %=1.5.1
	- $\mathcal{L}_{NT} = \set{0, S, +, \centerdot, E, <}$ where: \\
		-- $0$ is a constant symbol to be interpreted as 0 \\
		-- $S$ is a 1-arity function symbol to be interpreted as increment by 1 \\
		-- $+$ is a 2-arity function symbol to be interpreted as addition \\
		-- $·$ is a 2-arity function symbol to be interpreted as multiplication \\
		-- $E$ is a 2-arity function symbol to be interpreted as exponentiation \\
		-- $<$ is a 2-arity relation symbol to be interpreted as less than \\
	===================================================================================================================

\section{Sentences}
	===================================================================================================================
\subsection{(Definition) Free variable in a formula} %=1.5.2
	- The variable $v$ is free in the formula $\phi$ ($\free{v}{\phi}$) iff it satisfies some of the following: \\
		-- $\phi \in AF(\mathcal{L})$ and $\occurs{v}{\phi}$ \\
		-- $\phi \is \lnot \alpha$ and $\free{v}{\alpha}$ \\
		-- $\phi \is \alpha \lor \beta$ and $\free{v}{\alpha}$ or $\free{v}{\beta}$ \\
		-- $\phi \is \forall w \alpha$ and $v \inot \is w$ and $\free{v}{\alpha}$ \\
	===================================================================================================================
\subsection{(Definition) Sentence} %=1.5.3
	- The $\phi \in Form(\mathcal{L})$ is a sentence ($\phi \in Sent(\mathcal{L})$) iff $\set{x \in Var: \free{x}{\phi}} = \emptyset$ \\
	===================================================================================================================
\subsection{(Definition) Bound variable in a formula} %=1.5.1.5
	- The variable $v$ is bound in the formula $\phi$ ($\bound{v}{\phi}$) iff $\occurs{v}{phi}$ and $\pnot{\free{v}{\phi}}$ \\
	===================================================================================================================

\section{Structures}
	===================================================================================================================
\subsection{(Definition) Structure} %=1.6.1
	- The $\mathcal{L}$-structure $\mathfrak{A}$ of the language $\mathcal{L}$ ($Struct(\mathfrak{A}, \mathcal{L})$) is the tuple of: \\
		-- Universe: non-empty set $A$ \\
		-- ConstI: for any $c \in Const$, $\means{c}{\mathfrak{A}} \in A$ \\
		-- FuncI: for any $f \in Func$, $\means{f}{\mathfrak{A}}: A^{Arity(f)} \rightarrow A$ \\
		-- RelI: for any $P \in Rel$, $\means{P}{\mathfrak{A}} \subseteq A^{Arity(P)}$ \\
	===================================================================================================================
\subsection{(Definition) Henkin structure} %=1.6.4
	- The $\mathcal{L}$-structure $\mathfrak{A}$ is a Henkin structure iff it satisfies all of the following: \\
		-- $A = \set{t \in Term(\mathcal{L}): \txtforall{x \in Var}{\pnot{\occurs{x}{t}}}}$ \\
		-- For any $c \in Const$, $\means{c}{\mathfrak{A}} = c$ \\
		-- For any $f \in Func$, for any $\set{\vdc{a_i}{i=1}{Arity(f)}} \subseteq A$, $\means{f}{\mathfrak{A}}(\vdc{a_i}{i=1}{Arity(f)}) = f \vdc{a_i}{i=1}{Arity(f)}$ \\
		-- For any $P \in Rel$, BACKLOG: not important \\
	- The Henkin structure uses the syntactic elements as objects of the universe - useful for the Completeness theorem \\
	===================================================================================================================
\subsection{(Definition) Isomorphic structures} %=1.6.1.5
	- The $\mathcal{L}$-structure $\mathfrak{A}$ is isomorphic to the $\mathcal{L}$-structure $\mathfrak{B}$ ($\mathfrak{A} \cong \mathfrak{B}$) iff there exists a function $i: A \rightarrow B$ and $Bij(i)$ and it satisfies all of the following: \\
		-- For any $c \in Const$, $i(\means{c}{\mathfrak{A}}) = \means{c}{\mathfrak{B}}$ \\
		-- For any $f \in Func$, for any $\set{\vdc{a_i}{i=1}{Arity(f)}} \subseteq A$, $i(\means{f}{\mathfrak{A}}(\vdc{a_i}{i=1}{Arity(f)})) = \means{f}{\mathfrak{B}}(\vdc{i(a_i)}{i=1}{Arity(f)})$ \\
		-- For any $P \in Rel$, for any $\set{\vdc{a_i}{i=1}{Arity(P)}} \subseteq A$, $\vdc{a_i}{i=1}{Arity(P)} \in \means{P}{\mathfrak{A}}$ iff $\vdc{i(a_i)}{i=1}{Arity(P)} \in \means{P}{\mathfrak{B}}$ \\
	- $i$ preserves structure by way of operations in $\mathfrak{A}$ have corresponding equivalent operations in $\mathfrak{B}$ \\
	===================================================================================================================
\subsection{(Definition) Equivalence relation} %=1.6.1.5.a
	- The relation $R$ on the set $S$ is an $EqRel(R, S)$ iff it satisfies all of the following: \\
		-- For any $a \in S$, $a R a$ \\
		-- For any $\set{a, b} \subseteq S$, if $a R b$, then $b R a$ \\
		-- For any $\set{a, b, c} \subseteq S$, if $a R b$ and $b R c$, then $a R c$ \\
	===================================================================================================================
\subsection{(Metatheorem) Isomorphic structure equivalence} %=1.6.1.5.a
	- $EqRel(\cong, \set{\mathfrak{X}: Struct(\mathfrak{X}, \mathcal{L}}))$ \\
	- Proof: \\
		-- For any $\mathcal{L}$-structure $\mathfrak{A}$, then \\
			--- $j: A \rightarrow A$ and for any $a \in A$, $j(a) = a$ \\
			--- BACKLOG: $j$ satisfies $\mathfrak{A} \cong \mathfrak{B}$ \\
		-- For any $\mathcal{L}$-structures $\set{\mathfrak{A}, \mathfrak{B}}$, then \\
			--- If $\mathfrak{A} \cong \mathfrak{B}$, then \\
				---- There exists $i_{A, B}$, $i_{A, B}$ satisfies $\mathfrak{A} \cong \mathfrak{B}$ \\
				---- BACKLOG: ${i_{A, B}}^{-1}$ satisfies $\mathfrak{B} \cong \mathfrak{A}$ \\
		-- For any $\mathcal{L}$-structure $\set{\mathfrak{A}, \mathfrak{B}, \mathfrak{C}}$, then \\
			--- If $\mathfrak{A} \cong \mathfrak{B}$ and $\mathfrak{B} \cong \mathfrak{C}$, then \\
				---- There exists $i_{A, B}$, $i_{A, B}$ satisfies $\mathfrak{A} \cong \mathfrak{B}$ \\
				---- There exists $i_{B, C}$, $i_{B, C}$ satisfies $\mathfrak{B} \cong \mathfrak{C}$ \\
				---- BACKLOG: $i{B, C} \circ i_{A, B}$ satisfies $\mathfrak{A} \cong \mathfrak{C}$ \\
	===================================================================================================================

\section{Truth in a Structure}
	===================================================================================================================
\subsection{(Definition) Variable-universe assignment function} %=1.7.1
	- The function $s$ is a variable-universe assignment function into the $\mathcal{L}$-structure $\mathfrak{A}$ iff $s: Var \rightarrow A$ \\
	===================================================================================================================
\subsection{(Definition) Term-universe assignment function} %=1.7.3
	- The function $\extend{s}$ is the function generated from the variable-universe assignment function $s$ iff $\extend{s}: Term(\mathcal{L}) \rightarrow A$ and it satisfies all of the following: \\
		-- If $t \is x \in Var$, then $\extend{s}(t) = \extend{s}(x) = s(x)$ \\
		-- If $t \is c \in Const$, then $\extend{s}(t) = \extend{s}(c) = \means{c}{\mathfrak{A}}$ \\
		-- If $t \is f \vdc{t_i}{i=1}{Arity(f)}$, then $\extend{s}(t) = \extend{s}(f \vdc{t_i}{i=1}{Arity(f)}) = \means{f}{\mathfrak{A}}(\vdc{\extend{s}(t_i)}{i=1}{Arity(f)})$ \\
	===================================================================================================================
\subsection{(Definition) Modification of variable-universe assignment function} %=1.7.2
	- The function $s[x|a]$ is an $x$-modification of the variable-universe assignment function $s$ iff $x \in Var$ and $a \in A$ and it satisfies all of the following: \\
		-- If $v \inot \is x$, then $s[x|a](v) = s(v)$ \\
		-- If $v \is x$, then $s[x|a](v) = s[x|a](x) = a$ \\
	- The mapping of $x$ is fixed to $a$ \\
	===================================================================================================================
\subsection{(Definition) Relative truth to assignment} %=1.7.4
	- The $\mathcal{L}$-structure $\mathfrak{A}$ satisfies the formula $\phi$ with the variable-universe assignment function $s$ ($\mathfrak{A} \vDash \phi[s]$) iff it satisfies all of the following: \\
		-- If $\phi \is \equiv r t$, then $\extend{s}(r) = \extend{s}(t)$ \\
		-- If $\phi \is P \vdc{t_i}{i=1}{Arity(P)}$, then $\seq{\vdc{\extend{s}(t_i)}{i=1}{Arity(P)}} \in \means{P}{\mathfrak{A}}$ \\
		-- If $\phi \is \lnot \alpha$, then $\mathfrak{A} \inot \vDash \alpha[s]$ \\
		-- If $\phi \is \lor \alpha \beta$, then $\mathfrak{A} \vDash \alpha[s]$ or $\mathfrak{A} \vDash \beta[s]$ \\
		-- If $\phi \is \forall x \alpha$, then for any $a \in A$, $\mathfrak{A} \vDash \alpha[s[x|a]]$ \\
	- The $\mathcal{L}$-structure $\mathfrak{A}$ satisfies the set of formulas $\Gamma$ with the variable-universe assignment function $s$ ($\mathfrak{A} \vDash \Gamma[s]$) if for any $\gamma \in \Gamma$, $\mathfrak{A} \vDash \gamma[s]$ \\
	===================================================================================================================
\subsection{(Metatheorem) Variable assignment determines term assignment} %=1.7.6
	- If $s_1$ and $s_2$ are variable-universe assignment functions into the $\mathcal{L}$-structure $\mathfrak{A}$ and for any $t \in Term(\mathcal{L})$, for any $v \in \set{x \in Var: \occurs{x}{t}}$, $s_1(v) = s_2(v)$, then $\extend{s_1}(t) = \extend{s_2}(t)$ \\
	- Proof: \\
		-- $Term(\mathcal{L})_J \subseteq \set{t \in Term(\mathcal{L}): \txtif{\txtforall{v \in \set{x \in Var: \occurs{x}{t}}}{s_1(v) = s_2(v)}}{\extend{s_1}(t) = \extend{s_2}(t)}}$ \\
			--- If $\txtforall{v \in \set{x \in Var: \occurs{x}{t}}}{s_1(v) = s_2(v)}$, then  
				---- If $t \is v \in Var$, then \\
					----- $\extend{s_1}(v) = \extend{s_2}(v)$ \\
					----- $\extend{s_1}(t) = \extend{s_2}(t)$ \\
				---- If $t \is c \in Const$, then \\
					----- $\means{c}{\mathfrak{A}} = \means{c}{\mathfrak{A}}$ \\
					----- $\extend{s_1}(c) = \extend{s_2}(c)$ \\
				----- $\extend{s_1}(t) = \extend{s_2}(t)$ \\
		-- $Term(\mathcal{L})_Q$ closed in $\set{t \in Term(\mathcal{L}): \txtif{\txtforall{v \in \set{x \in Var: \occurs{x}{t}}}{s_1(v) = s_2(v)}}{\extend{s_1}(t) = \extend{s_2}(t)}}$ \\
			--- If $t \is f \vdc{t_i}{i=1}{Arity(f)}$ and $f \in Func$ and $\set{\vdc{t_i}{i=1}{Arity(f)}} \subseteq Term(\mathcal{L})$ and \\
			--- for any $t_i \in \set{\vdc{t_i}{i=1}{Arity(f)}}$, $\txtif{\txtforall{v \in \set{x \in Var: \occurs{x}{t_i}}}{s_1(v) = s_2(v)}}{\extend{s_1}(t_i) = \extend{s_2}(t_i)}$, then \\
				---- For any $t_i \in \set{\vdc{t_i}{i=1}{Arity(f)}}$, $\set{x \in Var: \occurs{x}{t_i}} \subseteq \set{x \in Var: \occurs{x}{t}}$ \\ 
				---- If for any $v \in \set{x \in Var: \occurs{x}{t}}$, $s_1(v) = s_2(v)$, then \\
					----- For any $t_i \in \set{\vdc{t_i}{i=1}{Arity(f)}}$, $\extend{s_1}(t_i) = \extend{s_2}(t_i)$ \\
					----- $\seq{\vdc{\extend{s_1}(t_i)}{i=1}{Arity(f)}} = \seq{\vdc{\extend{s_2}(t_i)}{i=1}{Arity(f)}}$ \\
					----- $\means{f}{\mathfrak{A}}(\vdc{\extend{s_1}(t_i)}{i=1}{Arity(f)}) = \means{f}{\mathfrak{A}}(\vdc{\extend{s_2}(t_i)}{i=1}{Arity(f)})$ \\
					----- $\extend{s_1}(f \vdc{t_i}{i=1}{Arity(f)}) = \extend{s_2}(f \vdc{t_i}{i=1}{Arity(f)})$ \\
					----- $\extend{s_1}(t) = \extend{s_2}(t)$ \\
	===================================================================================================================
\subsection{(Metatheorem) Free variable assignment determines relative truth} %=1.7.7
	- If $s_1$ and $s_2$ are variable-universe assignment functions into the $\mathcal{L}$-structure $\mathfrak{A}$ and $\phi \in Form(\mathcal{L})$ and for any $v \in \set{x \in Var: \free{x}{\phi}}$, $s_1(v) = s_2(v)$, then $Form(\mathcal{L}) \subseteq \set{\phi \in Form(\mathcal{L}): \mathfrak{A} \vDash \phi[s_1] \txtiff \mathfrak{A} \vDash \phi[s_2]}$ \\
	- Proof: \\
		-- $Form(\mathcal{L})_J \subseteq \set{\phi \in Form(\mathcal{L}): \txtif{\txtforall{v \in \set{x \in Var: \free{x}{\phi}}}{s_1(v) = s_2(v)}}{\mathfrak{A} \vDash \phi[s_1] \txtiff \mathfrak{A} \vDash \phi[s_2]}}$ \\
			--- If $\phi \is \equiv r t$, then \\
				---- $\set{x \in Var: \free{x}{\phi}} = \set{x \in Var: \occurs{x}{\phi}}$ \\
				---- $\extend{s_1}(r) = \extend{s_2}(r)$ \\
				---- $\extend{s_1}(t) = \extend{s_2}(t)$ \\
				---- $\extend{s_1}(r) = \extend{s_1}(t)$ iff $\extend{s_2}(r) = \extend{s_2}(t)$ \\
				---- $\mathfrak{A} \vDash \phi[s_1]$ iff $\mathfrak{A} \vDash \phi[s_2]$ \\
			--- If $\phi \is P \vdc{t_i}{i=1}{Arity(P)}$, then \\
				---- $\set{x \in Var: \free{x}{\phi}} = \set{x \in Var: \occurs{x}{\phi}}$ \\
				---- For any $t_i \in \set{\vdc{t_i}{i=1}{Arity(P)}}$, $\extend{s_1}(t_i) = \extend{s_2}(t_i)$ \\
				---- $\seq{\vdc{\extend{s_1}(t_i)}{i=1}{Arity(P)}} = \seq{\vdc{\extend{s_2}(t_i)}{i=1}{Arity(P)}}$ \\
				---- $\seq{\vdc{\extend{s_1}(t_i)}{i=1}{Arity(P)}} \in \means{P}{\mathfrak{A}}$ iff $\seq{\vdc{\extend{s_2}(t_i)}{i=1}{Arity(P)}} \in \means{P}{\mathfrak{A}}$ \\
				---- $\mathfrak{A} \vDash \phi[s_1]$ iff $\mathfrak{A} \vDash \phi[s_2]$ \\
		-- $Form(\mathcal{L})_Q$ closed in $\set{\phi \in Form(\mathcal{L}): \txtif{\txtforall{v \in \set{x \in Var: \free{x}{\phi}}}{s_1(v) = s_2(v)}}{\mathfrak{A} \vDash \phi[s_1] \txtiff \mathfrak{A} \vDash \phi[s_2]}}$ \\
			--- If $\phi \is \lnot \alpha$ and $\alpha \in Form(\mathcal{L})$ and \\
			--- $\txtif{\txtforall{v \in \set{x \in Var: \free{x}{\alpha}}}{s'_1(v) = s'_2(v)}}{\mathfrak{A} \vDash \alpha[s'_1] \txtiff \mathfrak{A} \vDash \alpha[s'_2]}$, then \\
				---- $\set{x \in Var: \free{x}{\alpha}} \subseteq \set{x \in Var: \free{x}{\phi}}$ \\
				---- If $\txtforall{v \in \set{x \in Var: \free{x}{\phi}}}{s_1(v) = s_2(v)}$, then \\
					----- $\mathfrak{A} \vDash \alpha[s_1]$ iff $\mathfrak{A} \vDash \alpha[s_2]$ \\
					----- $\mathfrak{A} \inot \vDash \alpha[s_1]$ iff $\mathfrak{A} \inot \vDash \alpha[s_2]$ \\
					----- $\mathfrak{A} \vDash \lnot \alpha[s_1]$ iff $\mathfrak{A} \vDash \lnot \alpha[s_2]$ \\
					----- $\mathfrak{A} \vDash \phi[s_1]$ iff $\mathfrak{A} \vDash \phi[s_2]$ \\
			--- If $\phi \is \lor \alpha \beta$ and $\set{\alpha, \beta} \subseteq Form(\mathcal{L})$ and \\
			--- $\txtif{\txtforall{v \in \set{x \in Var: \free{x}{\alpha}}}{s'_1(v) = s'_2(v)}}{\mathfrak{A} \vDash \alpha[s'_1] \txtiff \mathfrak{A} \vDash \alpha[s'_2]}$ and \\
			--- $\txtif{\txtforall{v \in \set{x \in Var: \free{x}{\beta}}}{s''_1(v) = s''_2(v)}}{\mathfrak{A} \vDash \beta[s''_1] \txtiff \mathfrak{A} \vDash \beta[s''_2]}$, then \\
				---- $\set{x \in Var: \free{x}{\alpha}} \subseteq \set{x \in Var: \free{x}{\phi}}$ \\
				---- $\set{x \in Var: \free{x}{\beta}} \subseteq \set{x \in Var: \free{x}{\phi}}$ \\
				---- If $\txtforall{v \in \set{x \in Var: \free{x}{\phi}}}{s_1(v) = s_2(v)}$, then \\
					----- $\mathfrak{A} \vDash \alpha[s_1]$ iff $\mathfrak{A} \vDash \alpha[s_2]$ \\
					----- $\mathfrak{A} \vDash \beta[s_1]$ iff $\mathfrak{A} \vDash \beta[s_2]$ \\
					----- ($\mathfrak{A} \vDash \alpha[s_1]$ or $\mathfrak{A} \vDash \beta[s_1]$) iff ($\mathfrak{A} \vDash \alpha[s_2]$ or $\mathfrak{A} \vDash \beta[s_2]$) \\
					----- $\mathfrak{A} \vDash \lor \alpha \beta[s_1]$ iff $\mathfrak{A} \vDash \lor \alpha \beta[s_2]$ \\
					----- $\mathfrak{A} \vDash \phi[s_1]$ iff $\mathfrak{A} \vDash \phi[s_2]$ \\
			--- If $\phi \is \forall z \alpha$ and $z \in Var$ and $\alpha \in Form(\mathcal{L})$ and \\
			--- $\txtif{\txtforall{v \in \set{x \in Var: \free{x}{\alpha}}}{s'_1(v) = s'_2(v)}}{\mathfrak{A} \vDash \alpha[s'_1] \txtiff \mathfrak{A} \vDash \alpha[s'_2]}$, then \\
				---- $\set{x \in Var: \free{x}{\alpha}} \subseteq \set{x \in Var: \free{x}{\phi} \cup \set{z}}$ \\
				---- If $\txtforall{v \in \set{x \in Var: \free{x}{\phi}}}{s_1(v) = s_2(v)}$, then 
					----- For any $a \in A$, for any $v \in \set{x \in Var: \free{x}{\alpha}}$, $s_1[z|a](v) = s_2[z|a](v)$ \\
					----- For any $a \in A$, $\mathfrak{A} \vDash \alpha[s_1[z|a]]$ iff for any $a \in A$, $\mathfrak{A} \vDash \alpha[s_2[z|a]]$ \\ % s[z|a] == s'
					----- $\mathfrak{A} \vDash \phi[s_1]$ iff $\mathfrak{A} \vDash \phi[s_2]$ \\
	===================================================================================================================
\subsection{(Metatheorem) Sentences have fixed truth} %=1.7.8
	- If $\sigma \in Sent(\mathcal{L})$ and $\mathfrak{A}$ is an $\mathcal{L}$-structure, then for any variable-universe assignment functions $s$, $\mathfrak{A} \vDash \sigma[s]$ or for any variable-universe assignment functions $s'$, $\mathfrak{A} \inot \vDash \sigma[s']$ \\
	- Proof: \\
		-- $\set{x \in Var: \free{x}{\sigma}} = \emptyset$ \\
		-- For any variable-universe assignment functions $s_1$ and $s_2$, $\mathfrak{A} \vDash \sigma[s_1]$ iff $\mathfrak{A} \vDash \sigma[s_2]$ \\
	===================================================================================================================
\subsection{(Definition) Structure models formula} %=1.7.9
	- The $\mathcal{L}$-structure $\mathfrak{A}$ models $\phi \in Form(\mathcal{L})$ ($\mathfrak{A} \vDash \phi$) iff for any variable-universe assignment function $s$, $\mathfrak{A} \vDash \phi[s]$ \\
	- The $\mathcal{L}$-structure $\mathfrak{A}$ models $\Phi \subseteq Form(\mathcal{L})$ ($\mathfrak{A} \vDash \Phi$) iff for any $\phi \in \Phi$, $\mathfrak{A} \vDash \phi$ \\
	===================================================================================================================
\subsection{(Definition) Abbreviations} %=1.7.1.5
	- BACKLOG: $\land$, $\implies$, $\iff$, $\exists x Q(x)$, $(\forall P(x)) Q(x)$, $(\exists P(x)) Q(x)$ \\
	===================================================================================================================
\subsection{(Metatheorem) Semantics of abbreviations} %=1.7.1.7
	- BACKLOG: show they are semantically expected
	===================================================================================================================

\section{Logical Implication}
	===================================================================================================================
\subsection{(Definition) Logical implication} %=1.9.1
	- The set of formulas $\Delta$ logically implies the set of formulas $\Gamma$ ($\Delta \vDash \Gamma$) iff for any $\mathcal{L}$-structure $\mathfrak{A}$, if $\mathfrak{A} \vDash \Delta$, then $\mathfrak{A} \vDash \Gamma$ \\
	- $\Delta \vDash \gamma$ abbreviates $\Delta \vDash \set{\gamma}$ \\
	===================================================================================================================
\subsection{(Definition) Valid formula} %=1.9.2
	- The formula $\phi$ is valid ($\vDash \phi$) iff $\emptyset \vDash \phi$ \\
	===================================================================================================================
\subsection{(Metatheorem) Variables self-equiv are valid} %=1.9.1.2
	- For any $v \in Var$, then $\vDash \equiv v v$ \\
		-- For any structure $\mathfrak{A}$, for any variable-universe assignment funtion $s$, \\
			-- $s(v) = \extend{s}(v)$ \\
			-- $\extend{s}(v) = \extend{s}(v)$ \\
			-- $\mathfrak{A} \vDash (\equiv v v)[s]$ \\
	====================================================================================================================
\subsection{(Definition) Universal closure} %=1.9.1.3
	- The universal closure of $\phi \in Form(\mathcal{L})$ with the free variables $\vdc{v_i}{i=1}{n}$ satisfies $UC(\phi) \is \vdc{\forall v_i}{i=1}{n} \phi$ \\
	- The universal closures of $\Phi \subseteq Form(\mathcal{L})$ satisfies $UC(\Phi) = \set{UC(\phi): \phi \in \Phi}$ \\ 
	====================================================================================================================
\subsection{(Metatheorem) Universal closure preserves validity} %=1.9.1.3
	- For any $\phi \in Form(\mathcal{L})$, for any $x \in Var$, for any structure $\mathfrak{A}$, $\mathfrak{A} \vDash \phi$ iff $\mathfrak{A} \vDash \forall x \phi$ \\
		-- If $\mathfrak{A} \vDash \phi$, then \\
			--- For any variable-universe assignment function $s$, $\mathfrak{A} \vDash \phi[s]$ \\
			--- For any $a \in A$, $\mathfrak{A} \vDash \phi[s[x|a]]$ \\
			--- $\mathfrak{A} \vDash \forall x \phi$ \\
		-- If $\vDash \forall x \phi$, then \\
			--- For any variable-universe assignment function $s$, $\mathfrak{A} \vDash (\forall x \phi)[s]$ \\
			--- For any $a \in A$, $\mathfrak{A} \vDash \phi[s[x|a]]$ \\
			--- $\mathfrak{A} \vDash \phi[s[x|s(x)]]$ \\
			--- $\mathfrak{A} \vDash \phi[s]$ \\
	====================================================================================================================
\subsection{(Metatheorem) Logical equivalence} %=1.9.1.4
	- $\phi$ has a logical equivalence to $\psi$ iff $\vDash (\phi \implies \psi)$ and $\vDash (\phi \implies \psi)$ \\
	- $\phi$ has a weak logical equivalence to $\psi$ iff $\phi \vDash \psi$ and $\psi \vDash \phi$ \\
	====================================================================================================================
\subsection{(Metatheorem) Strong logical equivalence property} %=1.9.1.4.a
	- If $\vDash (\phi \implies \psi)$, then $\phi \vDash \psi$ \\
	- Proof: \\
		-- If $\vDash (\phi \implies \psi)$, then \\
			--- For any structure $\mathfrak{A}$, \\
				---- For any variable-universe assignment function $s$, \\
					----- $\mathfrak{A} \vDash (\phi \implies \psi)[s]$ \\
					----- If $\mathfrak{A} \vDash \phi[s]$, then $\mathfrak{A} \vDash \psi[s]$ \\
				---- If (for any variable-universe assignment function $s_1$, $\mathfrak{A} \vDash \phi[s_1]$), then \\
					----- For any variable-universe assignment function $s_2$, \\
						------ $\mathfrak{A} \vDash \phi[s_2]$ \\ <HYP>
						------ If $\mathfrak{A} \vDash \phi[s_2]$, then $\mathfrak{A} \vDash \psi[s_2]$ \\
						------ $\mathfrak{A} \vDash \psi[s_2]$ \\
					----- For any variable-universe assignment function $s_2$, $\mathfrak{A} \vDash \psi[s_2]$ \\
				---- If $\mathfrak{A} \vDash \phi$, then $\mathfrak{A} \vDash \psi$ \\
			--- $\phi \vDash \psi$ \\
	====================================================================================================================
\subsection{(Metatheorem) Weak logical equivalence property} %=1.9.1.4.b
	- Not (If $\phi \vDash \psi$, then $\vDash (\phi \implies \psi)$) \\
	- Equivallently, $\phi \vDash \psi$ and $\inot \vDash (\phi \implies \psi)$ \\
	- Proof by counter-example: \\
		-- Let $\phi \is (x < y)$ and $\psi \is (z < w)$ \\
		-- For any structure $\mathfrak{A}$, \\
			--- If $\mathfrak{A} \vDash (x < y)$, then \\
				----	For any variable-universe assignment function $s_1$, $\mathfrak{A} \vDash (x < y)[s_1]$ \\
				---- $\means{<}{\mathfrak{A}} = A \times A$ \\
				---- For any variable-universe assignment function $s_2$, $\mathfrak{A} \vDash (z < w)[s_2]$ \\
				---- $\mathfrak{A} \vDash (z < w)$ \\
		-- $(x < y) \vDash (z < w)$ \\
		-- Let $\mathfrak{N} = \seq{\mathbb{N}, <_{std}}$ \\
		-- $\mathfrak{N} \inot \vDash (x < y) \implies (z < w)[s[x|0][y|1][w|0][z|1]]$ \\
		-- $\mathfrak{N} \inot \vDash (x < y) \implies (z < w)$ \\
		-- $\inot \vDash (x < y) \implies (z < w)$ \\
		-- $\inot \vDash \phi \implies \psi$ \\
	====================================================================================================================

\section{Substitutions and Substitutability}
	===================================================================================================================
\subsection{(Definition) Substitution in a term} %=1.8.1
	- The term $\sub{u}{x}{t}$ is the term $u$ with the variable $x$ replaced by the term $t$ iff it satisfies some of the following: \\
		-- If $u \is y \in Var$ and $y \neq x$, then $\sub{u}{x}{t} \is \sub{y}{x}{t} \is y$ \\
		-- If $u \is x$, then $\sub{u}{x}{t} \is \sub{x}{x}{t} \is t$ \\
		-- If $u \is c \in Const$, then $\sub{u}{x}{t} \is \sub{c}{x}{t} \is c$ \\
		-- If $u \is f \vdc{u_i}{i=1}{Arity(f)}$, then $\sub{u}{x}{t} \is \sub{f \vdc{u_i}{i=1}{Arity(f)}}{x}{t} \is f \vdc{\sub{u_i}{x}{t}}{i=1}{Arity(f)}$ \\
	===================================================================================================================
\subsection{(Definition) Substitution in a formula} %=1.8.2
	- The formula $\sub{\phi}{x}{t}$ is the formula $\phi$ with the variable $x$ replaced by the term $t$ iff it satisfies some of the following: \\
		-- If $\phi$ is atomic \\
			--- If $\phi \is \equiv u_1 u_2$, then $\sub{\phi}{x}{t} \is \sub{\equiv u_1 u_2}{x}{t} \is \equiv \sub{u_1}{x}{t} \sub{u_2}{x}{t}$ \\
		--- If $\phi \is P \vdc{u_i}{i=1}{Arity(P)}$, then $\sub{\phi}{x}{t} \is \sub{P \vdc{u_i}{i=1}{Arity(P)}}{x}{t} \is P \vdc{\sub{u_i}{x}{t}}{i=1}{Arity(P)}$ \\
		-- If $\phi$ is not atomic \\
			--- If $\phi \is \lnot \alpha$, then $\sub{\phi}{x}{t} \is \sub{\lnot \alpha}{x}{t} \is \lnot \sub{\alpha}{x}{t}$ \\
			--- If $\phi \is \lor \alpha \beta$, then $\sub{\phi}{x}{t} \is \sub{\lor \alpha \beta}{x}{t} \is \lor \sub{\alpha}{x}{t} \sub{\beta}{x}{t}$ \\
			--- If $\phi \is \forall y \alpha$, then \\
				---- If $y \is x$, then $\sub{\phi}{x}{t} \is \sub{\forall y \alpha}{x}{t} \is \forall y \alpha$ \\
				---- If $y \inot \is x$, then $\sub{\phi}{x}{t} \is \sub{\forall y \alpha}{x}{t} \is \forall y \sub{\alpha}{x}{t}$ \\
	===================================================================================================================
\subsection{(Definition) Substitutable term} %=1.8.3
	- The term $t$ is substitutable for the variable $x$ in the formula $\phi$ ($Subbable(t, x, \phi)$) iff it satisfies some of the following: \\
		-- $\phi$ is atomic \\
		-- $\phi \is \lnot \alpha$ and $Subbable(t, x \alpha)$ \\
		-- $\phi \is \lor \alpha \beta$ and $Subbable(t, x \alpha)$ and $Subbable(t, x \beta)$ \\
		-- $\phi \is \forall y \alpha$ and it satisfies some of the following: \\
			--- $\pnot{\free{x}{\phi}}$ \\
			--- $\pnot{\occurs{y}{t}}$ and $Subbable(t, x \alpha)$ \\
	- Identifies if the substitution preserves the context of the variables; i.e., bound variables stay bound, free variables stay free
	- Some operations will not be permitted even though substitution is always defined \\
	===================================================================================================================
\subsection{(Metatheorem) Closed terms are subbable} %=1.8.1.3
	- If the term $t$ is closed, then $Subbable(t, x, \phi)$ \\
	- If $\phi$ atomic, done \\
	- If $\phi \is \lnot \alpha$ and if $t$ is closed, then $Subbable(t, x, \alpha)$ \\
		-- If $t$ is closed, then \\
			--- $Subbable(t, x, \alpha)$ \\
			--- $Subbable(t, x, \phi)$ \\
	- If $\phi \is \lor \alpha \beta$ and if $t$ is closed, then $Subbable(t, x, \alpha)$ and if $t$ is closed, then $Subbable(t, x, \beta)$ \\
		-- If $t$ is closed, then \\
			--- $Subbable(t, x, \alpha)$ \\
			--- $Subbable(t, x, \beta)$\\
			--- $Subbable(t, x, \phi)$ \\
	- If $\phi \is \forall y \alpha$, and if $t$ is closed, then $Subbable(t, x, \alpha)$ \\
		-- If $t$ is closed, then \\
			--- $Subbable(t, x, \alpha)$ \\
			--- $\pnot{\occurs{y}{t}}$ \\
			--- $Subbable(t, x, \phi)$ \\
	===================================================================================================================
\subsection{(Metatheorem) Variables are self-subabble} %=1.8.1.4
	- $Subbable(x, x, \phi)$ \\
		-- If $\phi$ atomic, done \\
		-- If $\phi \is \lnot \alpha$ and $Subbable(x, x, \alpha)$, then $Subbable(x, x, \phi)$ \\
		-- If $\phi \is \lor \alpha \beta$ and $Subbable(x, x, \alpha)$ and $Subbable(x, x, \beta)$, then $Subbable(x, x, \phi)$ \\
		-- If $\phi \is \forall y \alpha$ and $Subbable(x, x, \alpha)$, then $Subbable(x, x, \phi)$ \\
			--- If $y \is x$, then $\pnot{\free{x}{\phi}}$, $Subbable(x, x, \phi)$ \\
			--- If $y \inot \is x$, then $\pnot{\occurs{y}{t}}$, then $Subbable(x, x, \phi)$ \\
	===================================================================================================================
\subsection{(Metatheorem) Substitutions of non-free variables is the identity} %=1.8.1.5
	- If $\pnot{\free{x}{\phi}}$, then $\sub{\phi}{x}{t} \is \phi$ \\
		-- If $\phi$ is atomic, then \\
			--- If $\pnot{\free{x}{\phi}}$, $\pnot{\occurs{x}{\phi}}$, then sub is identity (BACKLOG: not proven) \\
		-- If $\phi$ is not atomic, then \\
			--- If $\phi \is \lnot \alpha$ and if $\pnot{\free{x}{\alpha}}$, then $\sub{\alpha}{x}{t} \is \alpha$, then \\
				---- If $\pnot{\free{x}{\phi}}$, then \\
					----- $\pnot{\free{x}{\alpha}}$ \\
					----- $\sub{\alpha}{x}{t} \is \alpha$ \\
					----- $\sub{\phi}{x}{t} \is \sub{\lnot \alpha}{x}{t} \is \lnot \sub{\alpha}{x}{t} \is \lnot \alpha \is \phi$ \\
			--- If $\phi \is \lor \alpha \beta$, BACKLOG: do \\
			--- If $\phi \is \forall y \alpha$, BACKLOG: do \\
	===================================================================================================================
\subsection{(Metatheorem) Subbable is decidable} %=1.8.1.7
	- BACKLOG: do
	===================================================================================================================


\chapter{Deductions}
	===================================================================================================================
\section{Deductions}
	===================================================================================================================
\subsection{(Definition) Meta-restrictions for deduction} %=2.1
	- $\Lambda$ is the set of formulas that are logical axioms \\
	- $\Sigma$ is the set of formulas that are non-logical axioms \\
	- $R_I$ is the set of relations that are rules of inferences \\
 - In order to do this, we will impose the following restrictions on our logical axioms and rules of inference: \\
  -- 1. (Logical) axioms are decidable \\
  -- 2. Rules of inference are decidable \\
  -- 3. Rules of inference have finite inputs \\
  -- 4. (Logical) axiom are valid \\
  -- 5. Our rules of inference will preserve truth. For any $\seq{\Gamma, \phi} \in R_I$, $\Gamma \vDash \phi$ \\
 - (1-3) States that each step must be checkable and computable in finite time \\
 - (4-5) States that each step is valid \\
	===================================================================================================================
\subsection{(Definition) Deduction} %=2.2.1
	- The finite sequence $\seq{\vdc{\phi_i}{i=1}{n}}$ is a deduction from the non-logical axioms $\Sigma$ ($\Sigma \vdash \seq{\vdc{\phi_i}{i=1}{n}}$) iff $n \in \mathbb{N}$ and for any $1 \leq i \leq n$, it satisfies some of the following: \\
		-- $\phi_i \in \Lambda$ \\
		-- $\phi_i \in \Sigma$ \\
		-- There exists $R \in R_I$, $\seq{\Gamma, \phi_i} \in R$ and $\Gamma \subseteq \set{\vdc{\phi_j}{j=1}{i-1}}$ \\
	- $\Sigma \vdash \phi_n$ abbreviates $\Sigma \vdash \seq{\vdc{\phi_i}{i=1}{n}}$ \\
	===================================================================================================================
\subsection{(Metatheorem) Top-down definition equivalence of deduction} %=2.2.4
	- $Thm_\Sigma = \set{\phi \in Form(\mathcal{L}): \Sigma \vdash \phi} = Cl(\Lambda \cup \Sigma, R_I)$ \\
	- Proof: \\
		-- $Cl(\Lambda \cup \Sigma, R_I) \subseteq Thm_\Sigma$ \\
			--- If $\phi \in \Lambda \cup \Sigma$ then \\
				---- $\Sigma \vdash \seq{\phi}$ \\
				---- $\Sigma \vdash \phi$ \\
				---- $\phi \in Thm_\Sigma$ \\
			--- If there exists $R \in R_I$, $\seq{\Gamma, \phi} \in R$ and $\Gamma \subseteq Thm_\Sigma$, then \\
				---- $\Sigma \vdash \seq{\Gamma}$ \\
				---- $\Sigma \vdash \seq{\Gamma, \phi}$ \\
				---- $\Sigma \vdash \phi$ \\
				---- $\phi \in Thm_\Sigma$ \\
		-- $Thm_\Sigma \subseteq Cl(\Lambda \cup \Sigma, R_I)$ \\
			--- If $\phi_i \in Thm_\Sigma$, then \\
				---- If $i = 1$, then \\
					----- $\Sigma \vdash \seq{\phi_i}$ \\
					----- $\phi_i \in \Lambda \cup \Sigma$ \\
					----- $\phi_i \in Cl(\Lambda \cup \Sigma, R_I)$ \\
				---- If $i > 1$ and $\set{\vdc{\phi_j}{j=1}{i-1}} \subseteq Cl(\Lambda \cup \Sigma, R_I)$, then \\
					----- If $\phi_i \in \Lambda \cup \Sigma$, then $\phi_i \in Cl(\Lambda \cup \Sigma, R_I)$ \\
					----- If there exists $R \in R_I$, $\seq{\Gamma, \phi_i} \in R$ and $\Gamma \subseteq \set{\vdc{\phi_j}{j=1}{i-1}}$, then \\
						------ $\Gamma \subseteq Cl(\Lambda \cup \Sigma, R_I)$ \\
						------ $\phi_i \in Cl(\Lambda \cup \Sigma, R_I)$ \\
	===================================================================================================================

\section{Logical Axioms}
	- $\Lambda$ is the collection of all logical axioms \\
	===================================================================================================================
\subsection{(Definition) Equality axioms} %=2.3.1
	- E1: For any $v \in Var$, $\equiv v v \in \Lambda$ \\
	- E2: For any $f \in Func$, $((\land \vdc{\equiv x_i y_i}{i=1}{Arity(f)}) \implies (f(\vdc{x_i}{i=1}{Arity(f)}) \equiv f(\vdc{y_i}{i=1}{Arity(f)}))) \in \Lambda$ \\
	- E3: For any $P \in Rel \cup \set{\equiv}$, $((\land \vdc{\equiv x_i y_i}{i=1}{Arity(P)}) \implies (P(\vdc{x_i}{i=1}{Arity(P)}) \implies P(\vdc{y_i}{i=1}{Arity(P)}))) \in \Lambda$ \\
	- E2 and E3 allows equal parameters to be swapped \\
	===================================================================================================================
\subsection{(Definition) Quantifier axioms} %=2.3.2
	- Q1: For any $Subbable(t, x, \phi)$, $((\forall x \phi) \implies \sub{\phi}{x}{t}) \in \Lambda$ \\
	- Q2: For any $Subbable(t, x, \phi)$, $(\sub{\phi}{x}{t} \implies (\exists x \phi)) \in \Lambda$ \\
	- Q1 and Q2 use the $Subbable$ qualifier to preserve the nature of the variables \\
		===================================================================================================================
\subsection{(Metatheorem) Logical axioms are decidable} %=2.3.3
	- BACKLOG: (Equality axioms are decidable + Quantifier axioms are decidable) = $\Lambda$ are decidable \\
	===================================================================================================================

\section{Rules of Inference}
	===================================================================================================================
\subsection{(Definition) Propositional formula} %=2.4.1
	- The propositional formula $\phi$ of the language $\mathcal{L}$ ($\phi \in Prop(\mathcal{L})$) iff $\phi \in Form(\mathcal{L})$ and it satisfies some of the following: \\
		-- $\phi \in AF(\mathcal{L})$ \\
		-- $\phi \is \forall x \alpha$ \\
		-- $\phi \is \lnot \alpha$ and $\alpha \in Prop(\mathcal{L})$* \\
		-- $\phi \is \lor \alpha \beta$ and $\set{\alpha, \beta} \subseteq Prop(\mathcal{L})$* \\
	- Non-recursive definitions are called propositional variables ($\phi \in PV(\mathcal{L})$) \\
	===================================================================================================================
\subsection{(Definition) Truth assignment} %=2.4.1
	- The variable-truth assignment $v$ is the function $v: Prop(\mathcal{L})_J \rightarrow \set{\bot, \top}$ \\
	- The formula-truth assignment $\extend{v}$ of the variable-truth assignment $v$ is the function $\extend{v}: Prop(\mathcal{L}) \rightarrow \set{\bot, \top}$ and it satisfies some of the following: \\
		-- $\phi \in Prop(\mathcal{L})_J$ and $\extend{v}(\phi) = v(\phi)$ \\
		-- $\phi \in Prop$ and $\phi \is \lnot \alpha$ and \\
			--- If $\extend{v}(\alpha) = \bot$, then $\extend{v}(\phi) = \top$ \\
			--- If $\extend{v}(\alpha) = \top$, then $\extend{v}(\phi) = \bot$ \\
		-- $\phi \in Prop$ and $\phi \is \lor \alpha \beta$ and \\
			--- If $\extend{v}(\alpha) = \bot$ and $\extend{v}(\beta) = \bot$, then $\extend{v}(\phi) = \bot$ \\
			--- If $\extend{v}(\alpha) = \bot$ and $\extend{v}(\beta) = \top$, then $\extend{v}(\phi) = \top$ \\
			--- If $\extend{v}(\alpha) = \top$ and $\extend{v}(\beta) = \bot$, then $\extend{v}(\phi) = \top$ \\
			--- If $\extend{v}(\alpha) = \top$ and $\extend{v}(\beta) = \top$, then $\extend{v}(\phi) = \top$ \\
	- The set of formulas $\Phi$ is true for the variable-truth assignment $v$ ($\extend{v}^*(\Phi) = \top$) iff for any $\phi \in \Phi$, $\extend{v}(\phi) = \top$ \\
	===================================================================================================================
\subsection{(Metatheorem) Formulas are propositional} %=2.4.1
	- $Form(\mathcal{L}) = Prop(\mathcal{L})$ \\
	- Proof: \\
		-- $Prop(\mathcal{L}) \subseteq Form(\mathcal{L})$ from definition \\
		-- $Form(\mathcal{L}) \subseteq Prop(\mathcal{L})$ \\
			--- If $\phi \in AF(\mathcal{L})$, then $\phi \in Prop(\mathcal{L})$ \\
			--- If $\phi \inot \in AF(\mathcal{L})$, then \\
				---- If $\phi \is \forall x \alpha$ and $\alpha \in Prop(\mathcal{L})$, then $\phi \in Prop(\mathcal{L})$ \\
				---- If $\phi \is \lnot \alpha$ and $\alpha \in Prop(\mathcal{L})$, then $\phi \in Prop(\mathcal{L})$ \\
				---- If $\phi \is \lor \alpha \beta$ and $\set{\alpha, \beta} \subseteq Prop(\mathcal{L})$ \\
	===================================================================================================================
\subsection{(Definition) Propositional consequence} %=2.4.1
	- The formula $\phi$ is a propositional consequence of the set of formulas $\Gamma$ ($\Gamma \vDash_{PC} \phi$) iff for any variable-truth assignment $v$, if $\extend{v}^*(\Gamma) = \top$, then $\extend{v}(\phi) = \top$ \\
	- The formula $\phi$ is a tautology iff $\emptyset \vDash_{PC} \phi$ \\
	- $\vDash_{PC} \phi$ abbreviates $\emptyset \vDash_{PC} \phi$ \\
	===================================================================================================================
\subsection{(Metatheorem) Deduction theorem for PL} %=2.4.2
	- $\set{\vdc{\gamma_i}{i=1}{n}} \vDash_{PC} \phi$ iff $\vDash_{PC} (\land \vdc{\gamma_i}{i=1}{n}) \implies \phi$ \\
		-- If $n = 1$, then \\
			--- If $\gamma_1 \vDash_{PC} \phi$, then \\
				---- For any variable-truth assignment $v$, \\
					----- If $\extend{v}(\gamma_1) = \top$, then $\extend{v}(\phi) = \top$ \\
					----- If $\extend{v}(\gamma_1) = \top$, $\extend{v}(\gamma_1 \implies \phi) = \top$ \\
					----- If $\extend{v}(\gamma_1) = \bot$, $\extend{v}(\gamma_1 \implies \phi) = \top$ \\
				---- $\vDash_{PC} \gamma_1 \implies \phi$ \\
			--- If $\vDash_{PC} \gamma_1 \implies \phi$, then \\
				---- For any variable-truth assignment $v$, \\
					----- If $\extend{v}(\gamma_1) = \top$, then $\extend{v}(\phi) = \top$ \\
				---- $\gamma_1 \vDash_{PC} \phi$ \\
			--- $\gamma_1 \vDash_{PC} \phi$ iff $\vDash \gamma_1 \implies \phi$ \\
		-- If $n > 1$ and $\set{\vdc{\gamma_i}{i=1}{n-1}} \vDash_{PC} \phi$ iff $\vDash_{PC} (\land \vdc{\gamma_i}{i=1}{n-1}) \implies \phi$, then \\
			--- $\set{\vdc{\gamma_i}{i=1}{n-1}} \cup \gamma_n$ ... ditto basis step arguments \\
			--- Proof: TODO: from truth tables and definitions \\
	===================================================================================================================
\subsection{(Definition) PC rules of inference} %=2.4.5
	- PC: If $\Gamma \vDash_{PC} \phi$, then $\seq{\Gamma, \phi} \in R_I$ \\
	- This allows tautologies to be immediately useable in deductions \\
	===================================================================================================================
\subsection{(Definition) QR rules of inference} %=2.4.6
	- QR1: If $\pnot{\free{x}{\psi}}$, then $\seq{\set{\psi \implies \phi}, \psi \implies (\forall x \phi)} \in R_I$ \\
	- QR2: If $\pnot{\free{x}{\psi}}$, then $\seq{\set{\phi \implies \psi}, (\exists x \phi) \implies \psi} \in R_I$ \\
	- The qualifier $\pnot{\free{x}{\psi}}$ is used to denote that there are no assumptions about $x$ in $\psi$ \\
		===================================================================================================================
\subsection{(Metatheorem) Rules of inferences are decidable} %=2.4.3
	- BACKLOG: (PC rules are decidable + QR axioms are decidable) = $\Lambda$ are decidable \\
	===================================================================================================================
\subsection{(Metatheorem) Tautologies and models have similar shapes} %=2.4.3.6
	- For any $\mathcal{L}$-structure $\mathfrak{A}$, for any variable-universe assignment function $s$, for any $\phi \in Form(\mathcal{L})$, if for any $p \in \set{p \in PV(\mathcal{L}): \occurs{p}{\phi}}$, $v(p_i) = \top$ iff $\mathfrak{A} \vDash p_i[s]$, then $\extend{v}(\phi) = \top$ iff $\mathfrak{A} \vDash \phi[s]$ \\
		-- If $\phi \in PV(\mathcal{L})$, then $\extend{v}(\phi) = v(\phi) = \top$ iff $\mathfrak{A} \vDash \phi[s]$ \\
		-- If $\phi \is \lnot \alpha$ and (if for any $p \in \set{p \in PV(\mathcal{L}): \occurs{p}{\alpha}}$, $v(p_i) = \top$ iff $\mathfrak{A} \vDash p_i[s]$, then $\extend{v}(\alpha) = \top$ iff $\mathfrak{A} \vDash \alpha[s]$), then \\
			--- If for any $p \in \set{p \in PV(\mathcal{L}): \occurs{p}{\phi}}$, $v(p_i) = \top$ iff $\mathfrak{A} \vDash p_i[s]$, then \\
				---- $\set{p \in PV(\mathcal{L}): \occurs{p}{\alpha}} \subseteq \set{p \in PV(\mathcal{L}): \occurs{p}{\phi}}$ \\
				---- $\extend{v}(\alpha) = \top$ iff $\mathfrak{A} \vDash \alpha[s]$ \\
				---- $\extend{v}(\alpha) = \bot$ iff $\mathfrak{A} \inot \vDash \alpha[s]$ \\
				---- $\extend{v}(\lnot \alpha) = \top$ iff $\mathfrak{A} \vDash (\lnot \alpha)[s]$ \\
				---- $\extend{v}(\phi) = \top$ iff $\mathfrak{A} \vDash \phi[s]$ \\
		-- If $\phi \is \lor \alpha \beta$ and (if for any $p \in \set{p \in PV(\mathcal{L}): \occurs{p}{\alpha}}$, $v(p_i) = \top$ iff $\mathfrak{A} \vDash p_i[s]$, then $\extend{v}(\alpha) = \top$ iff $\mathfrak{A} \vDash \alpha[s]$) and (if for any $p \in \set{p \in PV(\mathcal{L}): \occurs{p}{\beta}}$, $v(p_i) = \top$ iff $\mathfrak{A} \vDash p_i[s]$, then $\extend{v}(\beta) = \top$ iff $\mathfrak{A} \vDash \beta[s]$), then \\
			--- If for any $p \in \set{p \in PV(\mathcal{L}): \occurs{p}{\phi}}$, $v(p_i) = \top$ iff $\mathfrak{A} \vDash p_i[s]$, then \\
				---- $\set{p \in PV(\mathcal{L}): \occurs{p}{\alpha}} \subseteq \set{p \in PV(\mathcal{L}): \occurs{p}{\phi}}$ \\
				---- $\set{p \in PV(\mathcal{L}): \occurs{p}{\beta}} \subseteq \set{p \in PV(\mathcal{L}): \occurs{p}{\phi}}$ \\
				---- $\extend{v}(\alpha) = \top$ iff $\mathfrak{A} \vDash \alpha[s]$ \\
				---- $\extend{v}(\beta) = \top$ iff $\mathfrak{A} \vDash \beta[s]$ \\
				---- ($\extend{v}(\alpha) = \top$ or $\extend{v}(\beta) = \top$) iff ($\mathfrak{A} \vDash \alpha[s]$ or $\mathfrak{A} \vDash \beta[s]$) \\
				---- $\extend{v}(\lor \alpha \beta) = \top$ iff $\mathfrak{A} \vDash (\lor \alpha \beta)[s]$ \\
				---- $\extend{v}(\phi) = \top$ iff $\mathfrak{A} \vDash \phi[s]$ \\
	===================================================================================================================
\subsection{(Metatheorem) Tautologies are valid} %=2.4.3.6
	- If $\vDash_{PC} \phi$, then $\vDash \phi$ \\
		-- If $\phi \in PV(\mathcal{L})$, then $\inot \vDash_{PC} \phi$ \\
		-- If $\phi \inot \in PV(\mathcal{L})$, then \\
			--- If $\vDash_{PC} \phi$, then \\
				---- For any $\mathcal{L}$-structure $\mathfrak{A}$, for any variable-universe assignment function $s$, \\
					----- For any $p \in \set{p \in PV(\mathcal{L}): \occurs{p}{\phi}}$, $v*(p) = \top$ iff $\mathfrak{A} \vDash p[s]$ \\
					----- $\extend{v*}(\phi) = \top$ iff $\mathfrak{A} \vDash \phi[s]$ \\
					----- $\extend{v*}(\phi) = \top$ \\
					----- $\mathfrak{A} \vDash \phi[s]$ \\
				---- $\vDash \phi$ \\
	===================================================================================================================

\section{Soundness}
	- Preserve truth: if $\vdash$, then $\vDash$ \\
	===================================================================================================================
\subsection{(Metatheorem) Logical axioms are valid} %=2.5.1
	- If $\phi \in \Lambda$, then $\vDash \phi$ \\
		-- If $\phi \in E1$, then \\
			--- $\phi \is \equiv v v$ \\
			--- $\vDash \equiv v v$ <Variables self-equiv are valid> \\
			--- $\vDash \phi$ \\
		-- If $\phi \in E2$, then \\
			--- $\phi \is (\land \vdc{\equiv x_i y_i}{i=1}{Arity(f)}) \implies (\equiv f(\vdc{x_i}{i=1}{Arity(f)}) f(\vdc{y_i}{i=1}{Arity(f)}))$ \\
			--- For any structure $\mathfrak{A}$, for any variable-universe assignment $s$, \\
				---- If $\mathfrak{A} \vDash (\land \vdc{\equiv x_i y_i}{i=1}{Arity(f)})[s]$, then \\
					----- For any $i \in \set{\vdc{i}{i=1}{Arity(f)}}$, $\mathfrak{A} \vDash (\equiv x_i y_i)[s]$ \\
					----- For any $i \in \set{\vdc{i}{i=1}{Arity(f)}}$, $s(x_i) = \extend{s}(x_i) = \extend{s}(y_i) = s(y_i)$ \\
					----- $\means{f}{\mathfrak{A}}(\vdc{\extend{s}(x_i)}{i=1}{Arity(f)}) = \means{f}{\mathfrak{A}}(\vdc{\extend{s}(y_i)}{i=1}{Arity(f)})$ \\
					----- $\extend{s}(f \vdc{x_i}{i=1}{Arity(f)}) = \extend{s}(f \vdc{y_i}{i=1}{Arity(f)})$ \\
					----- $\mathfrak{A} \vDash (\equiv f(\vdc{x_i}{i=1}{Arity(f)}) f(\vdc{y_i}{i=1}{Arity(f)}))[s]$ \\
				---- $\mathfrak{A} \vDash ((\land \vdc{\equiv x_i y_i}{i=1}{Arity(f)}) \implies (\equiv f(\vdc{x_i}{i=1}{Arity(f)}) f(\vdc{y_i}{i=1}{Arity(f)})))[s]$ \\
			--- $\vDash (\land \vdc{\equiv x_i y_i}{i=1}{Arity(f)}) \implies (\equiv f(\vdc{x_i}{i=1}{Arity(f)}) f(\vdc{y_i}{i=1}{Arity(f)}))$ \\
			--- $\vDash \phi$ \\
		-- If $\phi \in E3$, then \\
			--- $\phi \is (\land \vdc{\equiv x_i y_i}{i=1}{Arity(R)}) \implies (\equiv R(\vdc{x_i}{i=1}{Arity(R)}) R(\vdc{y_i}{i=1}{Arity(R)}))$ \\
			--- For any structure $\mathfrak{A}$, for any variable-universe assignment $s$, \\
				---- If $\mathfrak{A} \vDash (\land \vdc{\equiv x_i y_i}{i=1}{Arity(f)})[s]$, then \\
					----- For any $i \in \set{\vdc{i}{i=1}{Arity(f)}}$, $\mathfrak{A} \vDash (\equiv x_i y_i)[s]$ \\
					----- For any $i \in \set{\vdc{i}{i=1}{Arity(f)}}$, $s(x_i) = \extend{s}(x_i) = \extend{s}(y_i) = s(y_i)$ \\
					----- $\seq{\vdc{\extend{s}(x_i)}{i=1}{Arity(f)}} \in \means{R}{\mathfrak{A}}$ iff $\seq{\vdc{\extend{s}(y_i)}{i=1}{Arity(f)}} \in \means{R}{\mathfrak{A}}$ \\
					----- If $\seq{\vdc{\extend{s}(x_i)}{i=1}{Arity(f)}} \in \means{R}{\mathfrak{A}}$, then $\seq{\vdc{\extend{s}(y_i)}{i=1}{Arity(f)}} \in \means{R}{\mathfrak{A}}$ \\
					----- If $\mathfrak{A} \vDash (R(\vdc{x_i}{i=1}{Arity(R)}))[s]$, then $\mathfrak{A} \vDash (R(\vdc{y_i}{i=1}{Arity(R)}))[s]$ \\
					----- $\mathfrak{A} \vDash (R(\vdc{x_i}{i=1}{Arity(R)}) \implies R(\vdc{y_i}{i=1}{Arity(R)}))[s]$ \\
				---- $\mathfrak{A} \vDash ((\land \vdc{\equiv x_i y_i}{i=1}{Arity(f)}) \implies (R(\vdc{x_i}{i=1}{Arity(R)}) \implies R(\vdc{y_i}{i=1}{Arity(R)})))[s]$ \\
			--- $\vDash (\land \vdc{\equiv x_i y_i}{i=1}{Arity(f)}) \implies (R(\vdc{x_i}{i=1}{Arity(R)}) \implies R(\vdc{y_i}{i=1}{Arity(R)}))$ \\
			--- $\vDash \phi$ \\
		-- If $\phi \in Q1$, then \\
			--- $\phi \is (\forall x \phi) \implies \sub{\phi}{x}{t}$ and $Subbable(t, x, \phi)$ \\
			--- For any structure $\mathfrak{A}$, for any variable-universe assignment $s$, \\
				---- If $\mathfrak{A} \vDash (\forall x \phi)[s]$, then \\
					----- For any $a \in A$, $\mathfrak{A} \vDash \phi[s[x|a]]$ \\
					----- $\mathfrak{A} \vDash \phi[s[x|\extend{s}(t)]]$ \\ <NEW THEOREM>
					----- $\mathfrak{A} \vDash (\sub{\phi}{x}{t})[s]$ \\
				---- $\mathfrak{A} \vDash ((\forall x \phi) \implies \sub{\phi}{x}{t})[s]$ \\
			--- $\vDash (\forall x \phi) \implies \sub{\phi}{x}{t}$ \\
			--- $\vDash \phi$ \\
		-- If $\phi \in Q2$, then \\
			--- $\phi \is \sub{\phi}{x}{t} \implies (\exists x \phi)$ and $Subbable(t, x, \phi)$ \\
			--- For any structure $\mathfrak{A}$, for any variable-universe assignment $s$, \\
				---- If $\mathfrak{A} \vDash (\sub{\phi}{x}{t})[s]$, then \\
					----- $\mathfrak{A} \vDash \phi[s[x|\extend{s}(t)]]$ \\ <NEW THEOREM>
					----- $\extend{s}(t) \in A$ \\
					----- There exists $a \in A$, $\mathfrak{A} \vDash \phi[s[x|a]]$ \\
					----- $\mathfrak{A} \vDash (\exists x \phi)[s]$ \\
				---- $\mathfrak{A} \vDash (\sub{\phi}{x}{t} \implies (\exists x \phi))[s]$ \\
			--- $\vDash \sub{\phi}{x}{t} \implies (\exists x \phi)$ \\
			--- $\vDash \phi$ \\
	===================================================================================================================
\subsection{(Metatheorem) Rules of inference are closed under validity} %=2.5.2
	- If $\seq{\Gamma, \phi} \in R_I$, then $\Gamma \vDash \phi$ \\
		-- If $\seq{\Gamma, \phi} \in PC$, then \\
			--- $\Gamma \vDash_{PC} \phi$ \\
			--- $\vDash_{PC} (\land \vdc{\gamma}{\gamma \in \Gamma}{ }) \implies \phi$ \\ <NEW THEOREM>
			--- $\vDash (\land \vdc{\gamma}{\gamma \in \Gamma}{ }) \implies \phi$ \\
			--- If $\vDash \Gamma$, then \\
				---- For any $\gamma \in \Gamma$, $\vDash \gamma$ \\
				---- $\vDash (\land \vdc{\gamma}{\gamma \in \Gamma}{ })$ \\
				---- $\vDash \phi$ \\
			--- $\Gamma \vDash \phi$ \\
		-- If $\seq{\Gamma, \phi} \in QR1$, then \\
			--- $\phi \is \alpha \implies (\forall x \beta)$ and $\Gamma = \set{\alpha \implies \beta}$ and $\pnot{\free{x}{\alpha}}$ \\
			--- For any structure $\mathfrak{A}$, if $\mathfrak{A} \vDash \Gamma$, then \\
				---- $\mathfrak{A} \vDash \alpha \implies \beta$ \\
				---- For any variable-universe assignment $s$, $\mathfrak{A} \vDash (\alpha \implies \beta)[s]$ \\
				---- For any variable-universe assignment $s'$, \\
					----- If $\mathfrak{A} \vDash \alpha[s']$, then \\
						------ For any $a \in A$, \\
							------- $\mathfrak{A} \vDash (\alpha \implies \beta)[s'[x|a]]$ \\
							------- If $\mathfrak{A} \vDash \alpha[s'[x|a]]$, then $\mathfrak{A} \vDash \beta[s'[x|a]]$ \\
							------- $\mathfrak{A} \vDash \alpha[s']$ iff $\mathfrak{A} \vDash \alpha[s'[x|a]]$ \\ <NOT FREE IN ALPHA>
							------- $\mathfrak{A} \vDash \beta[s'[x|a]]$ \\
						------ $\mathfrak{A} \vDash (\forall x \beta)[s']$ \\
					----- $\mathfrak{A} \vDash (\alpha \implies \forall x \beta)[s']$ \\
				---- $\mathfrak{A} \vDash \alpha \implies \forall x \beta$ \\
				---- $\mathfrak{A} \vDash \phi$ \\
			--- $\Gamma \vDash \phi$ \\
		-- If $\seq{\Gamma, \phi} \in QR2$, then \\
			--- $\phi \is (\exists x \beta) \implies \alpha$ and $\Gamma = \set{\beta \implies \alpha}$ and $\pnot{\free{x}{\alpha}}$ \\
			--- For any structure $\mathfrak{A}$, if $\mathfrak{A} \vDash \Gamma$, then \\
				---- $\mathfrak{A} \vDash \beta \implies \alpha$ \\
				---- For any variable-universe assignment $s$, $\mathfrak{A} \vDash (\beta \implies \alpha)[s]$ \\
				---- For any variable-universe assignment $s'$, \\
					----- If $\mathfrak{A} \vDash (\exists x \beta)[s']$, then \\
						------ There exists $a \in A$, \\
							------- $\mathfrak{A} \vDash (\beta \implies \alpha)[s'[x|a]]$ \\
							------- If $\mathfrak{A} \vDash \beta[s'[x|a]]$, then $\mathfrak{A} \vDash \alpha[s'[x|a]]$ \\
							------- $\mathfrak{A} \vDash \beta[s'[x|a]]$ \\
							------- $\mathfrak{A} \vDash \alpha[s'[x|a]]$ \\
							------- $\mathfrak{A} \vDash \alpha[s'[x|a]]$ iff $\mathfrak{A} \vDash \alpha[s']$ \\ <NOT FREE IN ALPHA>
							------- $\mathfrak{A} \vDash \alpha[s']$ \\
						------ $\mathfrak{A} \vDash \alpha[s']$ \\
					----- $\mathfrak{A} \vDash ((\exists x \beta) \implies \alpha)[s']$ \\
				---- $\mathfrak{A} \vDash (\exists x \beta) \implies \alpha$ \\
				---- $\mathfrak{A} \vDash \phi$ \\
			--- $\Gamma \vDash \phi$ \\
	===================================================================================================================
\subsection{(Definition) Soundness} %=2.5.3
	- If $\Sigma \vdash \phi$, then $\Sigma \vDash \phi$ \\
	===================================================================================================================
\subsection{(Metatheorem) Soundness of First-order Logic} %=2.5.3
	- If $\Sigma \vdash \phi$, then $\Sigma \vDash \phi$ \\
	- $\set{\phi: \Sigma \vdash \phi} \subseteq \set{\phi: \Sigma \vDash \phi}$ \\
		-- If $\phi \in \Lambda$, then $\vDash \phi$, then $\Sigma \vDash \phi$ \\
		-- If $\phi \in \Sigma$, then $\Sigma \vDash \phi$ \\ 
		-- If $\seq{\Gamma, \phi} \in R_I$ and $\Gamma \subseteq \set{\phi: \Sigma \vDash \phi}$, then \\
			--- $\Sigma \vDash \Gamma$ \\
			--- $\Gamma \vDash \phi$ \\
			--- $\Sigma \vDash \phi$ \\
	- Brain dead syntactic manipulation corresponding to truth \\
	===================================================================================================================

\section{Two Technical Lemmas}
	===================================================================================================================
\subsection{(Metatheorem) Substitution and modification identity on assignments} %=2.6.1
	- $\extend{s}(\sub{u}{x}{t}) = \extend{s[x|\extend{s}(t)]}(u)$ \\
		-- If $u \in Var$ and $u \is x$, then \\
			--- $\extend{s}(\sub{x}{x}{t}) = \extend{s}(t) = \extend{s[x|\extend{s}(t)]}(x)$ \\
			--- $\extend{s}(\sub{u}{x}{t}) = \extend{s[x|\extend{s}(t)]}(u)$ \\
		-- If $u \in Var$ and $u \is y \inot \is x$, then \\
			--- $\extend{s}(\sub{y}{x}{t}) = \extend{s}(y) = \extend{s[x|\extend{s}(t)]}(y)$ \\
			--- $\extend{s}(\sub{u}{x}{t}) = \extend{s[x|\extend{s}(t)]}(u)$ \\
		-- If $u \in Const$ and $u \is c$, then \\
			--- $\extend{s}(\sub{c}{x}{t}) = \extend{s}(c) = \means{c}{\mathfrak{A}}$ \\
			--- $\extend{s}(\sub{u}{x}{t}) = \extend{s[x|\extend{s}(t)]}(u)$ \\
		-- If $u \is f \vdc{t_i}{i=1}{Arity(f)}$ and $\set{\vdc{t_i}{i=1}{Arity(f)}} \subseteq \set{r: \extend{s}(\sub{r}{x}{t}) = \extend{s[x|\extend{s}(t)]}(r)}$, then \\
			--- $\means{f}{\mathfrak{A}}(\vdc{\extend{s}(\sub{t_i}{x}{t})}{i=1}{Arity(f)}) = \means{f}{\mathfrak{A}}(\vdc{\extend{s[x|\extend{s}(t)]}(t_i)}{i=1}{Arity(f)})$ \\
			--- $\extend{s}(\sub{f \vdc{t_i}{i=1}{Arity(f)}}{x}{t}) = \extend{s}(f \vdc{\sub{t_i}{x}{t}}{i=1}{Arity(f)}) = \extend{s[x|\extend{s}(t)]}(f \vdc{(t_i)}{i=1}{Arity(f)})$ \\
			--- $\extend{s}(\sub{u}{x}{t}) = \extend{s[x|\extend{s}(t)]}(u)$ \\
	===================================================================================================================
\subsection{(Metatheorem) Substitution and modification identity on models} %=2.6.2
	- If $Subbable(t, x, \phi)$, then $\mathfrak{A} \vDash \sub{\phi}{x}{t}[s]$ iff $\mathfrak{A} \vDash \phi[s[x|\extend{s}(t)]]$ \\
		-- If $\phi \is \equiv u_1 u_2$, then \\
			--- $\extend{s}(\sub{u_1}{x}{t}) = \extend{s}(\sub{u_2}{x}{t})$ iff $\extend{s[x|\extend{s}(t)]}(u_1) = \extend{s[x|\extend{s}(t)]}(u_2)$ \\
			--- $\mathfrak{A} \vDash (\sub{\equiv u_1 u_2}{x}{t})[s]$ iff $\mathfrak{A} \vDash (\equiv \sub{u_1}{x}{t} \sub{u_2}{x}{t})[s]$ iff $\mathfrak{A} \vDash (\equiv u_1 u_2)[s[x|\extend{s}(t)]]$ \\
			--- $\mathfrak{A} \vDash \sub{\phi}{x}{t}[s]$ iff $\mathfrak{A} \vDash \phi[s[x|\extend{s}(t)]]$ \\
		-- If $\phi \is R \vdc{u_i}{i=1}{Arity(R)}$, then \\
			--- $\seq{\vdc{\extend{s}(\sub{u_i}{x}{t})}{i=1}{Arity(R)}} \in \means{R}{\mathfrak{A}}$ iff $\seq{\vdc{\extend{s[x|\extend{s}(t)]}(u_i)}{i=1}{Arity(R)}} \in \means{R}{\mathfrak{A}}$ \\
			--- $\mathfrak{A} \vDash (\sub{R \vdc{u_i}{i=1}{Arity(R)}}{x}{t})[s]$ iff $\mathfrak{A} \vDash (R \vdc{\sub{u_i}{x}{t}}{i=1}{Arity(R)})[s]$ iff $\mathfrak{A} \vDash (R \vdc{u_i}{i=1}{Arity(R)})[s[x|\extend{s}(t)]]$ \\
			--- $\mathfrak{A} \vDash \sub{\phi}{x}{t}[s]$ iff $\mathfrak{A} \vDash \phi[s[x|\extend{s}(t)]]$ \\
		-- If $\phi \is \lnot \alpha$ and $\set{\alpha} \subseteq \set{\gamma: \txtif{Subbable(t, x, \gamma)}{\mathfrak{A} \vDash \sub{\gamma}{x}{t}[s] \txtiff \mathfrak{A} \vDash \gamma[s[x|\extend{s}(t)]]}}$, then \\
			--- $Subbable(t, x, \alpha)$ \\
			--- $\mathfrak{A} \vDash \sub{\alpha}{x}{t}[s]$ iff $\mathfrak{A} \vDash \alpha[s[x|\extend{s}(t)]]$ \\
			--- $\mathfrak{A} \inot \vDash \sub{\alpha}{x}{t}[s]$ iff $\mathfrak{A} \inot \vDash \alpha[s[x|\extend{s}(t)]]$ \\
			--- $\mathfrak{A} \vDash (\sub{\lnot \alpha}{x}{t})[s]$ iff $\mathfrak{A} \vDash (\lnot \sub{\alpha}{x}{t})[s]$ iff $\mathfrak{A} \vDash (\lnot \alpha)[s[x|\extend{s}(t)]]$ \\
			--- $\mathfrak{A} \vDash \sub{\phi}{x}{t}[s]$ iff $\mathfrak{A} \vDash \phi[s[x|\extend{s}(t)]]$ \\
		-- If $\phi \is \lor \alpha \beta$ and $\set{\alpha, \beta} \subseteq \set{\gamma: \txtif{Subbable(t, x, \gamma)}{\mathfrak{A} \vDash \sub{\gamma}{x}{t}[s] \txtiff \mathfrak{A} \vDash \gamma[s[x|\extend{s}(t)]]}}$ \\
			--- $Subbable(t, x, \alpha)$ \\
			--- $\mathfrak{A} \vDash \sub{\alpha}{x}{t}[s]$ iff $\mathfrak{A} \vDash \alpha[s[x|\extend{s}(t)]]$ \\
			--- $Subbable(t, x, \beta)$ \\
			--- $\mathfrak{A} \vDash \sub{\beta}{x}{t}[s]$ iff $\mathfrak{A} \vDash \beta[s[x|\extend{s}(t)]]$ \\
			--- ($\mathfrak{A} \vDash \sub{\alpha}{x}{t}[s]$ or $\mathfrak{A} \vDash \sub{\beta}{x}{t}[s]$) iff ($\mathfrak{A} \vDash \alpha[s[x|\extend{s}(t)]]$ or $\mathfrak{A} \vDash \beta[s[x|\extend{s}(t)]]$) \\
			--- $\mathfrak{A} \vDash (\sub{\lor \alpha \beta}{x}{t})[s]$ iff $\mathfrak{A} \vDash (\lor \sub{\alpha}{x}{t} \sub{\beta}{x}{t})[s]$ iff $\mathfrak{A} \vDash (\lor \alpha \beta)[s[x|\extend{s}(t)]]$ \\
			--- $\mathfrak{A} \vDash \sub{\phi}{x}{t}[s]$ iff $\mathfrak{A} \vDash \phi[s[x|\extend{s}(t)]]$ \\
		-- If $\phi \is \forall y \alpha$ and $\set{\alpha} \subseteq \set{\gamma: \txtif{Subbable(t, x, \gamma)}{\mathfrak{A} \vDash \sub{\gamma}{x}{t}[s] \txtiff \mathfrak{A} \vDash \gamma[s[x|\extend{s}(t)]]}}$, then \\
			--- If $y \is x$, then \\
				---- $\mathfrak{A} \vDash \sub{\forall y \alpha}{x}{t}[s]$ iff $\mathfrak{A} \vDash (\forall y \alpha)[s]$ \\ <DEF SUB>
				---- $\mathfrak{A} \vDash (\forall y \alpha)[s]$ iff $\mathfrak{A} \vDash (\forall y \alpha)[s[x|\extend{s}(t)]]$ \\ <THM AGREE ALL FREE>
				---- $\mathfrak{A} \vDash \sub{\forall y \alpha}{x}{t}[s]$ iff $\mathfrak{A} \vDash (\forall y \alpha)[s[x|\extend{s}(t)]]$ \\
				---- $\mathfrak{A} \vDash \sub{\phi}{x}{t}[s]$ iff $\mathfrak{A} \vDash \phi[s[x|\extend{s}(t)]]$ \\
			--- If $y \inot \is x$, then \\
				---- If $\pnot{\free{x}{\phi}}$, then \\
					----- $\mathfrak{A} \vDash \sub{\forall y \alpha}{x}{t}[s]$ iff $\mathfrak{A} \vDash (\forall y \alpha)[s]$ \\ <Substitutions of non-free variables is the identity>
					----- $\mathfrak{A} \vDash (\forall y \alpha)[s]$ iff $\mathfrak{A} \vDash (\forall y \alpha)[s[x|\extend{s}(t)]]$ \\ <THM AGREE ALL FREE>
					----- $\mathfrak{A} \vDash \sub{\forall y \alpha}{x}{t}[s]$ iff $\mathfrak{A} \vDash (\forall y \alpha)[s[x|\extend{s}(t)]]$ \\
					----- $\mathfrak{A} \vDash \sub{\phi}{x}{t}[s]$ iff $\mathfrak{A} \vDash \phi[s[x|\extend{s}(t)]]$ \\
				---- If $\pnot{\occurs{y}{t}}$ and $Subbable(t, x \alpha)$, then \\
					----- For any $a \in A$, $\mathfrak{A} \vDash \sub{\alpha}{x}{t}[(s[y|a])]$ iff $\mathfrak{A} \vDash \alpha[(s[y|a])[x|\extend{s}(t)]]$ \\ <IH WHERE s=s[y|a]>
					----- $\mathfrak{A} \vDash \sub{\forall y \alpha}{x}{t}[s]$ iff $\mathfrak{A} \vDash (\forall y \sub{\alpha}{x}{t})[s]$ iff $\mathfrak{A} \vDash (\forall y \alpha)[s[x|\extend{s}(t)]]$ \\
					----- $\mathfrak{A} \vDash \sub{\phi}{x}{t}[s]$ iff $\mathfrak{A} \vDash \phi[s[x|\extend{s}(t)]]$ \\
	===================================================================================================================

\section{Properties of Our Deductive System}
	===================================================================================================================
\subsection{(Metatheorem) equiv is an equivalence relation} %=2.7.1
	- For any $\set{x, y, z} \in Var$, \\
		-- $\vdash x \equiv x$ \\
		-- $\vdash x \equiv y \implies y \equiv x$ \\
		-- $\vdash (x \equiv y \land y \equiv z) \implies x \equiv z$ \\
	- Proof: \\
		-- $\vdash x \equiv x$ \\ <E1>
		-- $\vdash x \equiv y \implies y \equiv x$ \\
			--- $((x \equiv y) \land (x \equiv x)) \implies ((x \equiv x) \implies (y \equiv x))$ \\ <E3>
			--- $x \equiv x$ \\ <E1>
			--- $(x \equiv y) \implies ((x \equiv x) \implies (y \equiv x))$ \\ <PC>
			--- $(x \equiv y) \implies y \equiv x$ \\ <PC>
		-- $\vdash (x \equiv y \land y \equiv z) \implies (x \equiv z)$ \\
			--- $(x \equiv x \land y \equiv z) \implies ((x \equiv y) \implies (x \equiv z))$ \\ <E3>
			--- $x \equiv x$ \\ <E1>
			--- $(y \equiv z) \implies ((x \equiv y) \implies (x \equiv z))$ \\ <PC>
			--- $(y \equiv z \land x \equiv y) \implies (x \equiv z)$ \\ <PC>
			--- $(x \equiv y \land y \equiv z) \implies (x \equiv z)$ \\ <PC>
	===================================================================================================================
\subsection{(Metatheorem) Universal closure preserves deductiblity} %=2.7.2
	- $\Sigma \vdash \phi$ iff $\Sigma \vdash \forall x \phi$ \\
		-- If $\Sigma \vdash \phi$, then \\
			--- $\Sigma \vdash \phi$ \\
			--- $\Sigma \vdash ((\forall z (z \equiv z)) \lor \lnot (\forall z (z \equiv z))) \implies \phi$ \\ <PC>
			--- $\Sigma \vdash ((\forall z (z \equiv z)) \lor \lnot (\forall z (z \equiv z))) \implies \forall x \phi$ \\ <QR1>
			--- $\Sigma \vdash ((\forall z (z \equiv z)) \lor \lnot (\forall z (z \equiv z)))$ \\ <PC>
			--- $\Sigma \vdash \forall x \phi$ \\ <PC>
		-- If $\Sigma \vdash \forall x \phi$, then \\
			--- $\Sigma \vdash  \forall x \phi$ \\
			--- $\Sigma \vdash  \forall x \phi \implies \sub{\phi}{x}{x}$ \\ <Q1>
			--- $\Sigma \vdash  \sub{\phi}{x}{x}$ \\ <PC>
			--- $\Sigma \vdash \phi$ \\
		- $((\forall z (z \equiv z)) \lor \lnot (\forall z (z \equiv z)))$ is a closed formula that is tautological \\
		- Keep structures + variable-universe assignment functions in mind when interpreting universal closure deductions \\
		- We can replace axioms with all sentences without changing the strength of the deductive system \\
	===================================================================================================================
\subsection{(Metatheorem) Universal closure preserves strength of axioms} %=2.7.3
	- $\Sigma \vdash \phi$ iff $UC(\Sigma) \vdash \phi$ \\
	- Proof: \\
		-- If $\Sigma \vdash \phi$, then
			--- $UC(\Sigma) \vdash \Sigma$ \\
			--- $UC(\Sigma) \vdash \phi$ \\
		-- If $UC(\Sigma) \vdash \phi$, then \\
			--- $\Sigma \vdash UC(\Sigma)$ \\
			--- $\Sigma \vdash \phi$ \\
	- We can universally close the set of formulas $\Sigma$ and it will deduce the same as $UC(\Sigma)$ sentences \\
	===================================================================================================================
\subsection{(Metatheorem) Deduction theorem} %=2.7.4
	- If $\theta$ is a sentence, then $\Sigma \cup \set{\theta} \vdash \phi$ iff $\Sigma \vdash \theta \implies \phi$ \\
		-- If $\Sigma \vdash \theta \implies \phi$, then \\
			--- $\Sigma \cup \set{\theta} \vdash \theta \implies \phi$ \\
			--- $\Sigma \cup \set{\theta} \vdash \theta$ \\
			--- $\Sigma \cup \set{\theta} \vdash \phi$ \\ <PC>
		-- $\set{\alpha: \Sigma \cup \set{\theta} \vdash \alpha} \subseteq \set{\alpha: \Sigma \vdash \theta \implies \alpha}$ \\
			--- If $\alpha \in \Lambda$, then \\
				---- $\vdash \alpha$ \\
				---- $\Sigma \vdash \theta \implies \alpha$ \\ <PC>
			--- If $\alpha \in \Sigma$, then \\
				---- $\Sigma \vdash \alpha$ \\
				---- $\Sigma \vdash \theta \implies \alpha$ \\ <PC>
			--- If $\alpha \is \theta$, then \\
				---- $\vdash \theta \implies \theta$ \\ <PC>
				---- $\Sigma \vdash \theta \implies \alpha$
			--- If $\seq{\Gamma, \alpha} \in PC$ and for any $\gamma \in \Gamma$, $\Sigma \vdash \theta \implies \gamma$, then \\
				---- $\Sigma \implies \Gamma$ \\
				---- $\Sigma \implies \alpha$ \\ <PC>
			--- If $\seq{\Gamma, \alpha} \in QR1$ and for any $\gamma \in \Gamma$, $\Sigma \vdash \theta \implies \gamma$, then \\
				---- $\Gamma = \set{\rho \implies \tau}$ \\
				---- $\alpha \is \rho \implies \forall x \tau$ \\
				---- $\pnot{\free{x}{\rho}}$ \\
				---- $\Sigma \vdash \theta \implies (\rho \implies \tau)$ \\
				---- $\Sigma \vdash (\theta \land \rho) \implies \tau$ \\ <PC>
				---- $\pnot{\free{x}{\theta}}$ \\
				---- $\pnot{\free{x}{\theta \land \rho}}$ \\
				---- $\Sigma \vdash (\theta \land \rho) \implies \forall x \tau$ \\ <QR1>
				---- $\Sigma \vdash \theta \implies (\rho \implies \forall x \tau)$ \\ <PC>
				---- $\Sigma \vdash \phi$ \\
			--- If $\seq{\Gamma, \alpha} \in QR2$ and for any $\gamma \in \Gamma$, $\Sigma \vdash \theta \implies \gamma$, then \\
				---- $\Gamma = \set{\tau \implies \rho}$ \\
				---- $\alpha \is \exists x \tau \implies \rho$ \\
				---- $\pnot{\free{x}{\rho}}$ \\
				---- $\Sigma \vdash \theta \implies (\tau \implies \rho)$ \\
				---- $\Sigma \vdash (\theta \land \tau) \implies \rho$ \\ <PC>
				---- $\Sigma \vdash (\tau \land \theta) \implies \rho$ \\ <PC>
				---- $\Sigma \vdash \tau \implies (\theta \implies \rho)$ \\ <PC>
				---- $\pnot{\free{x}{\theta}}$ \\
				---- $\pnot{\free{x}{\theta \land \rho}}$ \\
				---- $\Sigma \vdash \exists x \tau \implies (\theta \implies \rho)$ \\ <QR2>
				---- $\Sigma \vdash (\exists x \tau \land \theta) \implies \rho$ \\ <PC>
				---- $\Sigma \vdash (\theta \land \exists x \tau) \implies \rho$ \\ <PC>
				---- $\Sigma \vdash \theta \implies (\exists x \tau \implies \rho)$ \\ <PC>
				---- $\Sigma \vdash \theta \implies \alpha$ \\
		-- If $\Sigma \cup \set{\theta} \vdash \phi$, then $\Sigma \vdash \theta \implies \phi$ \\
	===================================================================================================================
\subsection{(Metatheorem) Proof by contradiction} %=2.7.1.4
		-- If $\Sigma \vdash \phi$, then \\
			--- $\Sigma \cup \set{\lnot \phi} \vdash \phi$ \\
			--- $\Sigma \cup \set{\lnot \phi} \vdash \lnot \phi$ \\
			--- $\Sigma \cup \set{\lnot \phi} \vdash ((\forall z (z \equiv z)) \land \lnot (\forall z (z \equiv z)))$ \\ <PC>
		-- If $\Sigma \cup \set{\lnot \phi} \vdash ((\forall z (z \equiv z)) \land \lnot (\forall z (z \equiv z)))$, then \\
			--- $\Sigma \vdash \lnot \phi \implies ((\forall z (z \equiv z)) \land \lnot (\forall z (z \equiv z)))$ \\ <Deduction theorem>
			--- $\Sigma \vdash (\lnot (\forall z (z \equiv z)) \lor (\forall z (z \equiv z))) \implies \phi$ \\ <PC>
			--- $\Sigma \vdash \phi$ \\ <PC>
	===================================================================================================================
\subsection{(Metatheorem) Strong to weak quantification} %=2.7.1.5
	- If $P$ is a 1-ary relation symbol, then $\vdash (\forall x (P(x))) \implies (\exists x (P(x)))$ \\
	- Proof: \\
		-- $\lnot ((\forall x (P(x))) \implies (\exists x (P(x)))) \vdash \lnot (\lnot (\forall x (P(x))) \lor (\exists x (P(x))))$ \\ <PC>
		-- $\lnot (\lnot (\forall x (P(x))) \lor (\exists x (P(x)))) \vdash (\forall x (P(x)) \land \lnot (\exists x (P(x)))$ \\ <PC>
		-- $(\forall x (P(x)) \land \lnot (\exists x (P(x))) \is (\forall x (P(x)) \land \lnot (\lnot \forall x (\lnot P(x)))$ \\
		-- $(\forall x (P(x)) \land \lnot (\lnot \forall x (\lnot P(x))) \vdash (\forall x (P(x)) \land (\forall x (\lnot P(x)))$ \\ <PC>
		-- $(\forall x (P(x)) \land (\forall x (\lnot P(x)) \vdash ((\forall z (z \equiv z)) \land \lnot (\forall z (z \equiv z)))$ \\ <PC>
		-- $\lnot ((\forall x (P(x))) \implies (\exists x (P(x)))) \vdash ((\forall z (z \equiv z)) \land \lnot (\forall z (z \equiv z)))$
	===================================================================================================================
\subsection{(Metatheorem) Quantifier switcheroni} %=2.7.1.6
	- $\forall x \forall y P(x, y) \vdash \forall y \forall x P(x, y)$ \\
	- Proof: \\
		 -- $\vdash \forall x \forall y P(x, y) \implies \forall y \forall x P(x, y)$ \\ <PC>
		 -- $\forall x \forall y P(x, y) \vdash \forall y \forall x P(x, y)$ \\ <Deduction theorem>
	===================================================================================================================
\subsection{(Metatheorem) Quantifier combineroni} %=2.7.1.7
	- $\vdash (\forall x (P(x)) \land \forall x (Q(x))) \implies \forall x (P(x) \land Q(x))$ \\ <PC>
	===================================================================================================================

\section{Non-logical Axioms}
	- The non-logical axioms characterizes the behavior of a specific theory \\
	- Non-logical axioms have to be decidable as well \\
	===================================================================================================================
\subsection{(Definition) Weak number theory} %=2.8.3
	- The non-logical axioms of Number theory $N \subseteq Form(\mathcal{L}_{NT})$ consists of: \\
		-- $\forall x \lnot (S x \equiv 0) \forall \forall $ \\
		-- $\forall x \forall y (S x \equiv S y \implies x \equiv y)$ \\
		-- $\forall x (x + 0 \equiv 0)$ \\
		-- $\forall x \forall y (x + S y \equiv S (x + y))$ \\
		-- $\forall x (x \centerdot 0 \equiv 0)$ \\
		-- $\forall x \forall y (x \centerdot S y \equiv (x \centerdot y) + x)$ \\
		-- $\forall x (x E 0 \equiv S 0)$ \\
		-- $\forall x \forall y (x E (S y) \equiv (x E y) \centerdot x)$ \\
		-- $\forall x (\lnot x < 0)$ \\
		-- $\forall x \forall y (x < S y \iff (x < y \lor x \equiv y))$ \\
		-- $\forall x \forall y (x < y \lor x \equiv y \lor y < x)$ \\
	- $\hat{a} \is \overleftarrow{a}$ \\
	===================================================================================================================
\subsection{(Metatheorem) Weak number theory theorems} %=2.8.4
	- For any natural numbers $a, b$, \\
		-- If $a = b$, then $N \vdash \hat{a} \equiv \hat {b}$ \\
		-- If $a \inot = b$, then $N \vdash \lnot (\hat{a} \equiv \hat {b})$ \\
		-- If $a < b$, then $N \vdash \hat{a} < \hat{b}$ \\
		-- BACKLOG: ...
	- BACKLOG: Proof: \\
	===================================================================================================================
\subsection{(Metatheorem) Weakness of weak number theory 1} %=2.8.1.7
	- $N \inot \vdash \lnot (x < x)$ \\
	- BACKLOG: p.298 Construct a structure $\mathfrak{A}$ that satisfies $\mathfrak{A} \vDash N$ and $\mathfrak{A} \inot \vDash \forall x \lnot (x < x)$ \\
	===================================================================================================================
\subsection{(Metatheorem) Weakness of weak number theory 2} %=2.8.1.8
	- $N \inot \vdash (x + y) \equiv (y + x)$ \\
	- BACKLOG: p.298 Construct a structure $\mathfrak{A}$ that satisfies $\mathfrak{A} \vDash N$ and $\mathfrak{A} \inot \vDash (x + y) \equiv (y + x)$ \\
	===================================================================================================================


\chapter{Completeness and Compactness}
	===================================================================================================================
\section{Naively}
	===================================================================================================================
\subsection{(Definition) Completeness} %=3.1.1
	- If $\Sigma \vDash \phi$, then $\Sigma \vdash \phi$ \\
	===================================================================================================================

\section{Completeness}
	===================================================================================================================
\subsection{(Definition) Contradictory sentence} %=3.2.1
	- The sentence $\contr \is ((\forall z (z \equiv z)) \land \lnot (\forall z (z \equiv z)))$ \\
	- For any language $\mathcal{L}$, $\contr \in Sent(\mathcal{L})$ \\
	===================================================================================================================
\subsection{(Definition) Inconsistent and unsatisfiable} %=3.2.1
	- The set of formulas $\Sigma$ is inconsistent iff $\Sigma \vdash \contr$ \\
	- The set of formulas $\Sigma$ is consistent iff $\Sigma \inot \vdash \contr$ \\
	- The set of formulas $\Sigma$ is unsatisfiable iff $\Sigma \vDash \contr$ \\
	- The set of formulas $\Sigma$ is satisfiable iff $\Sigma \inot \vDash \contr$ \\
	===================================================================================================================
\subsection{(Metatheorem) Contradiction has no model} %=3.2.1
	- $\mathfrak{A} \inot \vDash \contr$ \\
	- Proof: \\
		-- If $\mathfrak{A} \vDash \contr$, then \\
			--- $\mathfrak{A} \vDash ((\forall z (z \equiv z)) \land \lnot (\forall z (z \equiv z)))$ \\
			--- $\mathfrak{A} \vDash (\forall z (z \equiv z))$ and $\mathfrak{A} \vDash \lnot (\forall z (z \equiv z))$ \\ <Definition>
			--- $\mathfrak{A} \inot \vDash (\forall z (z \equiv z))$ \\ <Definition>
			--- Not $\mathfrak{A} \vDash (\forall z (z \equiv z))$ \\ <Definition>
			--- $\mathfrak{A} \vDash (\forall z (z \equiv z))$ and not $\mathfrak{A} \vDash (\forall z (z \equiv z))$ \\
			--- CONTR \\
		-- $\mathfrak{A} \inot \vDash ((\forall z (z \equiv z)) \land \lnot (\forall z (z \equiv z)))$ \\
		-- $\mathfrak{A} \inot \vDash \contr$ \\
	===================================================================================================================
\subsection{(Metatheorem) Unsatisfiable equivalence} %=3.2.1
	- $\Sigma \vDash \contr$ iff for any $\mathfrak{A}$, $\mathfrak{A} \inot \vDash \Sigma$ \\
	- Proof: \\
		-- $\Sigma \vDash \contr$ iff \\
			--- For any $\mathfrak{A}$, $\mathfrak{A} \inot \vDash \Sigma$ iff \\
			--- For any $\mathfrak{A}$, if $\mathfrak{A} \vDash \Sigma$, then $\mathfrak{A} \vDash \contr$ iff \\
			--- For any $\mathfrak{A}$, no $\mathfrak{A} \vDash \Sigma$ or $\mathfrak{A} \vDash \contr$ iff \\
			--- For any $\mathfrak{A}$, $\mathfrak{A} \inot \vDash \Sigma$ \\
		- $\Sigma \inot \vDash \contr$ iff there exists $\mathfrak{A}$, $\mathfrak{A} \vDash \Sigma$ \\
	===================================================================================================================
\subsection{(Metatheorem) Completeness of First-order Logic: Proof lemma schema}
	- Prove: (I) If $UC(\Sigma) \inot \vdash \contr$, then there exists $\mathfrak{A}$, $\mathfrak{A} \vDash UC(\Sigma)$ \\
	- Corollaries: $\Sigma \vDash \phi$, then $\Sigma \vdash \phi$ \\
		-- If $UC(\Sigma) \inot \vdash \contr$, \\
			--- There exists $\mathfrak{A}$, $\mathfrak{A} \vDash UC(\Sigma)$ and $\mathfrak{A} \inot \vDash \contr$ \\ <Contradiction has no model>
			--- Not for any $\mathfrak{A}$, if $\mathfrak{A} \vDash UC(\Sigma)$, then $\mathfrak{A} \vDash \contr$ \\ <Definition>
			--- $UC(\Sigma) \inot \vDash \contr$ \\ <Definition>
		-- If $UC(\Sigma) \inot \vdash \contr$, then $UC(\Sigma) \inot \vDash \contr$ \\ <Abbreviate>
		-- If $UC(\Sigma) \vDash \contr$, then $UC(\Sigma) \vdash \contr$ \\ <Contraposition>
		-- $UC(\Sigma) \vDash \contr$ iff \\
			--- For any $\mathfrak{A}$, if $\mathfrak{A} \vDash UC(\Sigma)$, then $\mathfrak{A} \vDash \contr$ iff \\ <Definition>
			--- For any $\mathfrak{A}$, if $\mathfrak{A} \vDash \Sigma$ or $\mathfrak{A} \vDash \contr$ iff \\ <Universal closure preserves validity>
			--- $\Sigma \vDash \contr$ \\ <Definition>
		-- $UC(\Sigma) \vDash \contr$ iff $\Sigma \vDash \contr$ \\ <Abbreviate>
		-- If $\Sigma \vDash \contr$, then $UC(\Sigma) \vdash \contr$ \\ <Equivalence>
		-- $UC(\Sigma) \vdash \contr$ iff $\Sigma \vdash \contr$ \\ <Universal closure preserves strength of axioms>
		-- If $\Sigma \vDash \contr$, then $\Sigma \vdash \contr$ \\ <Equivalence>
		-- If $\Sigma \vDash \phi$, then \\
			--- For any $\mathfrak{A}$, \\ 
				---- If $\mathfrak{A} \vDash \Sigma$, then $\mathfrak{A} \vDash \phi$ \\
				---- If $\mathfrak{A} \vDash \Sigma \cup \set{\lnot \phi}$, then \\
					----- $\mathfrak{A} \vDash \Sigma$ \\
					----- $\mathfrak{A} \vDash \phi$ \\
					----- $\mathfrak{A} \vDash \lnot \phi$ \\
					----- $\mathfrak{A} \inot \vDash \phi$ \\
					----- $\mathfrak{A} \vDash \phi$ and $\mathfrak{A} \inot \vDash \phi$ \\
					----- $\mathfrak{A} \vDash \phi$ and not $\mathfrak{A} \vDash \phi$ \\
					----- CONTR \\
				---- $\mathfrak{A} \inot \vDash \Sigma \cup \set{\lnot \phi}$ \\
			--- For any $\mathfrak{A}$, $\mathfrak{A} \inot \vDash \Sigma \cup \set{\lnot \phi}$ \\ <Abbreviate>
			--- $\Sigma \cup \set{\lnot \phi} \vDash \contr$ \\ <Unsatisfiable equivalence>
			--- $\Sigma \cup \set{\lnot \phi} \vdash \contr$ \\
			--- $\Sigma \vdash \phi$ \\ <Proof by contradiction>
		-- If $\Sigma \vDash \phi$, then $\Sigma \vdash \phi$ \\ <Abbreviate>
	===================================================================================================================
\subsection{(Definition) Henkin theory for countable language}
	- A theory with added constants and axioms to make it easier to model with a universe of variable free terms \\
	- $\Sigma' \subseteq Sent(\mathcal{L}')$ is the Henkin theory of $\Sigma \subseteq Sent(\mathcal{L})$ iff 
		-- $\mathcal{L}'$ construction: language with Henkin constants \\
			--- $\mathcal{L}_0 = \mathcal{L}$ \\
			--- $\mathcal{L}_{i+1} = \mathcal{L}_{i} \cup^{Const} \set{c_{(i, j)}: j \in \mathbb{N}}$ \\
			--- $\mathcal{L}' = \cup^{Const}_{i \in \mathbb{N}} \mathcal{L}_i$ \\
		-- $\hat{\Sigma}$ construction: theory with Henkin axioms \\
			--- $\Sigma_0 = \Sigma$ \\
			--- $H_{i+1} = \set{(\exists x \theta_j \implies \sub{\theta_j}{x}{c_{(i, j)}}): \exists x \theta_j \in Sent(\mathcal{L}_i)}$ \\
			--- $\Sigma_{i+1} = \Sigma_i \cup H_{i+1}$ \\
			--- $\hat{\Sigma} = \cup_{i \in \mathbb{N}} \Sigma_i$ \\
		-- $\Sigma'$ construction: theory with chosen enumerated axioms \\
			--- $\Sigma^0 = \hat{\Sigma}$ \\
			--- $\alpha_i \in Sent(\mathcal{L}')$ \\
			--- $\Sigma^{i+1} = \Sigma^i \cup \set{\alpha_i}$ iff $\Sigma^i \cup \set{\alpha_i} \inot \vdash \contr$ \\
			--- $\Sigma^{i+1} = \Sigma^i \cup \set{\lnot \alpha_i}$ iff $\Sigma^i \cup \set{\alpha_i} \vdash \contr$ \\
			--- $\Sigma' = \cup_{i \in \mathbb{N}} \Sigma^i$ \\
	===================================================================================================================
\subsection{(Definition) Deduction language notation}
	- $\Sigma \vdash_{\mathcal{L}} \phi$ abbreviates $\phi \in Cl(\Sigma \cup \Lambda(\mathcal{L}), RI(\mathcal{L})))$ \\
	===================================================================================================================
\subsection{(Metatheorem) Expansion by Henkin constants preserves consistency} %=3.2.3
	- If $\Sigma \subseteq Sent(\mathcal{L})$, then if $\Sigma \inot \vdash_{\mathcal{L}} \contr$, then $\Sigma \inot \vdash_{\mathcal{L}'} \contr$ \\ 
	- Proof: \\
		-- If $\Sigma \subseteq Sent(\mathcal{L})$ and $\Sigma \inot \vdash_{\mathcal{L}} \contr$, then \\
			--- If $\Sigma \vdash_{\mathcal{L}'} \contr$, then \\
				---- There exists $D'$, $D'$ has the smallest number $n$ of added Henkin constants that satisfies $\contr \in D'$ \\
				---- If $n = 0$, then \\
					----- $\Sigma \vdash_{\mathcal{L}'} \contr$ iff $\Sigma \vdash_{\mathcal{L}} \contr$ \\
					----- $\Sigma \vdash_{\mathcal{L}} \contr$ \\
					----- $\Sigma \inot \vdash_{\mathcal{L}} \contr$ \\
					----- Not $\Sigma \vdash_{\mathcal{L}} \contr$ and $\Sigma \vdash_{\mathcal{L}} \contr$ \\
					----- CONTR \\
				---- If $n > 0$, then \\
					----- There exists $c$, $c$ is an added constant that occurs in $D'$ \\
					----- There exists $v$, $v$ is a variable that does not occur in $D'$ \\ <INFINITE VARS>
					----- There exists $D$, $D = \seq{\vdc{d_i: \sub{d_i}{v}{c} \is d_i'}{i=1}{|D'|}}$ \\
					----- For any $d_i \in \set{\vdc{d_i}{i=1}{|D|}}$, \\
						------ If $d_i' \in \Lambda$, then \\
							------- If $d_i' \in E1 \cup E2 \cup E3$, then \\
								-------- $d_i' \is d_i$ \\
								-------- $d_i \in \Sigma$ \\
							------- If $d_i' \in Q1$, then \\
								-------- $d_i' \is ((\forall x \phi') \implies \sub{\phi'}{x}{t})$ \\ 
								-------- $Subbable(t, x, \phi')$ \\
								-------- $Subbable(t, x, \phi)$ \\
								-------- $d_i \is ((\forall x \phi) \implies \sub{\phi}{x}{t})$ \\
								-------- $d_i \in Q1$ \\
							------- If $d_i' \in Q2$, then proof isomorphic to $d_i' \in Q1$ \\
						------ If $d_i' \in \Sigma$, then \\
						------ $d_i' \is d_i$ \\
						------ $d_i \in \Sigma$ \\
						------ If $\seq{\Gamma', d_i'} \in R_I$, then \\
							------- If $\seq{\Gamma', d_i'} \in PR$, then \\
							------- $\Gamma' \vDash_{PC} d_i'$ \\
							------- $\Gamma \vDash_{PC} d_i$ \\
							------- $\seq{\Gamma, d_i} \in PR$ \\
							------- If $\seq{\Gamma', d_i'} \in QR1$, then \\
								-------- $\Gamma' = \set{\psi' \implies \phi'}$ \\
								-------- $\Gamma' \subseteq \set{\vdc{d_j'}{j=1}{i-1}}$ \\
								-------- $d_i' = \psi' \implies (\forall x \phi')$ \\
								-------- $\pnot{\free{x}{\psi'}}$ \\
								-------- $\pnot{\free{x}{\psi}}$ \\
								-------- $\Gamma = \set{\psi \implies \phi}$ \\
								-------- $\Gamma \subseteq \set{\vdc{d_j}{j=1}{i-1}}$ \\
								-------- $d_i = \psi \implies (\forall x \phi)$ \\
								-------- $\seq{\Gamma, d_i} \in QR1$ \\
							------- If $\seq{\Gamma', d_i'} \in QR2$, then proof isomorphic to $\seq{\Gamma', d_i'} \in QR1$ \\
					----- $D$ has $n-1$ added constants for a deduction of $\contr$ \\
					----- $n \leq n-1$ \\
					----- CONTR \\
			--- $\Sigma \inot \vdash_{\mathcal{L}'} \contr$ \\ <Metaproof by contradiction>
		-- If $\Sigma \subseteq Sent(\mathcal{L})$, then if $\Sigma \inot \vdash_{\mathcal{L}} \contr$, then $\Sigma \inot \vdash_{\mathcal{L}'} \contr$ \\ <Implication over conjunction>
	===================================================================================================================
\subsection{(Metatheorem) Expansion by Henkin axioms preserves consistency} %=3.2.4
	- If $\Sigma \subseteq Sent(\mathcal{L}')$, then if $\Sigma \inot \vdash \contr$, then $\hat{\Sigma} \inot \vdash \contr$ \\
	- Proof: \\
		-- If $\Sigma \subseteq Sent(\mathcal{L}')$ and $\Sigma \inot \vdash \contr$, then \\
			--- If $\hat{\Sigma} \vdash \contr$, then \\
				---- There exists $n$, $n$ is the smallest number of added Henkin axioms for any deduction of $\contr$ \\
				---- There exists $H$ and $\alpha$, $|H \cup \set{\alpha}| = n$ and $\Sigma \cup H \cup \set{\alpha} \vdash \contr$ \\
				---- There exists $v$, $v$ is a variable that does not occur in $\Sigma$ \\ <INFINITE VARS>			
				---- There exists $c$, $\alpha \is \exists x \phi \implies \sub{\phi}{x}{c}$ \\
				---- $\Sigma \cup H \vdash \lnot \alpha$ \\ <Proof by contradiction>
				---- $\Sigma \cup H \vdash \lnot (\exists x \phi \implies \sub{\phi}{x}{c})$ \\
				---- $\Sigma \cup H \vdash (\exists x \phi \land \lnot \sub{\phi}{x}{c})$ \\ <PC>
				---- $\Sigma \cup H \vdash \exists x \phi$ \\ <PC>
					----- $\Sigma \cup H \vdash \lnot \forall x \lnot \phi$ \\
				---- $\Sigma \cup H \vdash \lnot \sub{\phi}{x}{c}$ \\ <PC>
					----- $\Sigma \cup H \vdash \lnot \sub{\phi}{x}{c}$ \\
					----- $\Sigma \cup H \vdash \lnot \sub{\phi}{x}{z}$ \\
					----- $\Sigma \cup H \vdash \lnot \forall z \sub{\phi}{x}{z}$ \\
					----- $Subbable(z, x, \lnot \sub{\phi}{x}{z})$ \\
					----- $\vdash (\forall z \lnot \sub{\phi}{x}{z}) \implies \sub{\lnot \sub{\phi}{x}{z}}{z}{x}$ \\ <Q1>
					----- $\Sigma \cup H \vdash \sub{\lnot \sub{\phi}{x}{z}}{z}{x}$ \\ <PC>
					----- $\Sigma \cup H \vdash \lnot \phi$ \\ <PC>
					----- $\Sigma \cup H \vdash \forall x \lnot \phi$ \\ <PC>
				---- $\Sigma \cup H \vdash (\lnot \forall x \lnot \phi) \land (\forall x \lnot \phi)$ \\ <PC>
				---- $\Sigma \cup H \vdash \contr$ \\ <PC>
				---- $|H| = n-1$ \\
				---- $n \leq n-1$ \\
				---- CONTR \\
			--- $\hat{\Sigma} \inot \vdash \contr$ \\ <Metaproof by contradiction>
		-- If $\Sigma \subseteq Sent(\mathcal{L}')$, then if $\Sigma \inot \vdash \contr$, then $\hat{\Sigma} \inot \vdash \contr$ \\ <Implication over conjunction>
	===================================================================================================================
\subsection{(Metatheorem) Consistency from below} %=3.2.1.2
	- If for any $i \in \mathbb{N}$, $\Sigma_i \inot \vdash \contr$ and $\Sigma_i \subseteq \Sigma_{i+1}$, then $\cup_{i \in \mathbb{N}} \Sigma_i \inot \vdash \contr$ \\
	- Proof: \\
		-- If for any $i \in \mathbb{N}$, $\Sigma_i \inot \vdash \contr$ and $\Sigma_i \subseteq \Sigma_{i+1}$, then \\
			--- If $\cup_{i \in \mathbb{N}} \Sigma_i \vdash \contr$, then \\
				---- There exists $k \in \mathbb{N}$, $\Sigma_k \vdash \contr$ \\ <DEDUCTIONS ARE FINITE CHOOSE SUFFICIENTLY LARGE K>
				---- $\Sigma_k \inot \vdash \contr$ \\
				---- $\Sigma_k \vdash \contr$ and not $\Sigma_k \vdash \contr$ \\
				---- CONTR \\
			--- $\cup_{i \in \mathbb{N}} \Sigma_i \inot \vdash \contr$ \\ <Metaproof by contradiction>
	===================================================================================================================
\subsection{(Metatheorem) Consistency step} %=3.2.1.3
	- If $\Sigma \inot \vdash \contr$, then if $\Sigma \cup \set{\alpha} \vdash \contr$, then $\Sigma \cup \set{\lnot \alpha} \inot \vdash \contr$ \\
	- Proof: \\
		-- If $\Sigma \inot \vdash \contr$, then \\
			--- If $\Sigma \cup \set{\alpha} \vdash \contr$ and $\Sigma \cup \set{\lnot \alpha} \vdash \contr$, then \\
				---- $\Sigma \vdash \alpha \implies \contr$ \\ <Deduction theorem>
				---- $\Sigma \vdash \lnot \alpha \implies \contr$ \\  <Deduction theorem>
				---- $\seq{\alpha \implies \contr, \lnot \alpha \implies \contr, \contr} \in PC$ \\
				---- $\Sigma \vdash \contr$ \\ <PC>
				---- $\Sigma \vdash \contr$ and not $\Sigma \vdash \contr$ \\
				---- CONTR \\
			--- Not ($\Sigma \cup \set{\alpha} \vdash \contr$ and $\Sigma \cup \set{\lnot \alpha} \vdash \contr$) \\ <Metaproof by contradiction>
			--- If $\Sigma \cup \set{\alpha} \vdash \contr$, then $\Sigma \cup \set{\lnot \alpha} \inot \vdash \contr$ \\ <Implication definition>
	- If $\Sigma \inot \vdash \contr$, then if $\Sigma \cup \set{\lnot \alpha} \vdash \contr$, then $\Sigma \cup \set{\alpha} \inot \vdash \contr$ \\
	===================================================================================================================
\subsection{(Metatheorem) Expansion by chosen enumerated axioms preserves consistency} %=3.2.1.4
	- If $\Sigma \subseteq Sent(\mathcal{L}')$, then if $\hat{\Sigma} \inot \vdash \contr$, then $\Sigma' \inot \vdash \contr$ \\
	- Proof: \\
		-- If $\Sigma \subseteq Sent(\mathcal{L}')$, then \\
			--- If $k = 0$, $\Sigma^k = \Sigma^0 = \hat{\Sigma} \inot \vdash \contr$ \\
			--- If $k > 0$ and $\Sigma^k \inot \vdash \contr$, then \\
				---- If $\Sigma^{k+1} = \Sigma^k \cup \set{\alpha_k}$, then \\
					----- $\Sigma^{k+1} = \Sigma^k \cup \set{\alpha_k} \inot \vdash \contr$ \\
				---- If $\Sigma^{k+1} = \Sigma^k \cup \set{\lnot \alpha_k}$, then \\
					----- $\Sigma^k \cup \set{\alpha_k} \vdash \contr$ \\
					----- $\Sigma^k \cup \set{\lnot \alpha_k} \inot \vdash \contr$ \\ <Consistency step>
				---- $\Sigma^{k+1} \inot \vdash \contr$ \\
			--- For any $k \in \mathbb{N}$, $\Sigma_k \inot \vdash \contr$ \\ <Induction>
			--- For any $k \in \mathbb{N}$, $\Sigma_k \subseteq \Sigma_{k+1}$ \\
			--- For any $k \in \mathbb{N}$, $\Sigma_k \subseteq \Sigma_{k+1}$ and $\Sigma_k \inot \vdash \contr$ \\
			--- $\Sigma' = \cup_{i \in \mathbb{N}} \Sigma_i \inot \vdash \contr$ \\ <Consistency from below>
	===================================================================================================================
\subsection{(Metatheorem) Expansion by chosen enumerated axioms is deductively closed} %=3.2.5
	- If $\phi \in Sent(\mathcal{L}')$, then $\phi \in \Sigma'$ iff $\Sigma' \vdash \phi$ \\
	- Proof: \\
		-- If $\phi \in \Sigma'$, then $\Sigma' \vdash \phi$ \\ <Definition>
		-- If $\Sigma' \vdash \phi$, then \\
			--- There exists $i$, $\Sigma^i \vdash \phi$ \\ <DEDUCTIONS ARE FINITE>
			--- $\Sigma^i \inot \vdash \contr$ \\ <Expansion by chosen enumerated axioms preserves consistency>
			--- $\Sigma^i \cup {\lnot \phi} \vdash \contr$ \\ <Proof by contradiction>
			--- $\Sigma^i \cup {\phi} \inot \vdash \contr$ \\ <Consistency step>
			--- $\Sigma^{i+1} = \Sigma^i \cup \set{\phi}$ \\ <Definition>
			--- $\phi \in \Sigma^{i+1}$ \\
			--- $\phi \in \Sigma'$ \\ <Definition>
	===================================================================================================================
\subsection{(Metatheorem) Expansion by chosen enumerated axioms is maximal}
	- If $\phi \in Sent(\mathcal{L}')$, then $\phi \in \Sigma'$ iff $\lnot \phi \inot \in \Sigma'$ \\
	- Proof: \\
		-- If $\phi \in Sent(\mathcal{L}')$, then \\
			--- $\phi \in \Sigma'$ iff \\
				---- $\Sigma' \vdash \phi$ iff \\ <Expansion by chosen enumerated axioms is deductively closed>
				---- $\Sigma' \inot \vdash \lnot \phi$ iff \\ <Expansion by chosen enumerated axioms preserves consistency>
				---- $\lnot \phi \inot \in \Sigma'$ \\ <Expansion by chosen enumerated axioms is deductively closed>
			--- $\phi \in \Sigma'$ iff $\lnot \phi \inot \in \Sigma'$ \\ <Abbreviate>
	===================================================================================================================
\subsection{(Definition) VFT}
	- $VFT(\mathcal{L}') = \set{t \in Term(\mathcal{L}'): \txtforall{v \in Var}{\pnot{\occurs{v}{t}}}}$ \\
	===================================================================================================================
\subsection{(Definition) VFTS relation}
	- $\seq{t_1, t_2} \in \sim \subseteq VFT(\mathcal{L}')^2$ iff $t_1 \equiv t_2 \in \Sigma'$ \\
	===================================================================================================================
\subsection{(Metatheorem) VFTS is an equivalence relation} %=3.2.1.7
	- $\sim$ is an equivalence relation on $VFT(\mathcal{L}')^2$ \\
	- Proof: \\
		-- $t_1 \sim t_1$ \\
			--- $\Sigma' \vdash x \equiv x$ \\ <E1>
			--- $\Sigma' \vdash \forall x (x \equiv x)$ \\ <Universal closure preserves deductiblity>
			--- $Subbable(t_1, x, x \equiv x)$ \\ <Definition>
			--- $\Sigma' \vdash \forall x (x \equiv x) \implies \sub{x \equiv x}{x}{t_1}$ \\ <Q1>
			--- $\Sigma' \vdash \sub{x \equiv x}{x}{t_1}$ \\ <PC>
			--- $\Sigma' \vdash t_1 \equiv t_1$ \\ <Definition>
			--- $t_1 \equiv t_1 \in \Sigma'$ \\ <Expansion by chosen enumerated axioms is deductively closed>
			--- $t_1 \sim t_1$ \\ <Definition>
		-- If $t_1 \sim t_2$, then $t_2 \sim t_1$ \\
			--- If $t_1 \sim t_2$, then \\
				---- $t_1 \equiv t_2 \in \Sigma'$ \\ <Definition>
				---- $\Sigma' \vdash t_1 \equiv t_2$ \\ <Expansion by chosen enumerated axioms is deductively closed>
				---- $\vdash t_1 \equiv t_2 \implies t_2 \equiv t_1$ \\ <equiv is an equivalence relation>
				---- $\seq{t_1 \equiv t_2, t_1 \equiv t_2 \implies t_2 \equiv t_1, t_2 \equiv t_1} \in PC$ \\
				---- $\Sigma' \vdash t_2 \equiv t_1$ \\ <PC>
				---- $t_2 \equiv t_1 \in \Sigma'$ \\ <Expansion by chosen enumerated axioms is deductively closed>
				---- $t_2 \sim t_1$ \\ <Definition>
		-- If $t_1 \sim t_2$ and $t_2 \sim t_3$, then $t_1 \sim t_3$ \\
			--- If $t_1 \sim t_2$ and $t_2 \sim t_3$, then \\
				---- $t_1 \equiv t_2 \in \Sigma'$ \\ <Definition>
				---- $t_2 \equiv t_3 \in \Sigma'$ \\ <Definition>
				---- $\Sigma' \vdash t_1 \equiv t_2$ \\ <Expansion by chosen enumerated axioms is deductively closed>
				---- $\Sigma' \vdash t_2 \equiv t_3$ \\ <Expansion by chosen enumerated axioms is deductively closed>
				---- $\vdash (\vdash t_1 \equiv t_2 \land t_2 \equiv t_3) \implies t_1 \equiv t_3$ \\ <equiv is an equivalence relation>
				---- $\seq{t_1 \equiv t_2, t_2 \equiv t_3, (t_1 \equiv t_2 \land t_2 \equiv t_3) \implies t_1 \equiv t_3, t_1 \equiv t_3} \in PC$ \\
				---- $\Sigma' \vdash t_1 \equiv t_3$ \\ <PC>
				---- $t_1 \equiv t_3 \in \Sigma'$ \\ <Expansion by chosen enumerated axioms is deductively closed>
				---- $t_1 \sim t_3$ \\ <Definition>
	===================================================================================================================
\subsection{(Definition) VFT in Sigma' equivalence class}
	- $[t]^\sim = \set{s \in VFT(\mathcal{L}'): t \sim s}$ \\
	===================================================================================================================
\subsection{(Definition) Henkin universe}
	- $A' = \set{[t]: t \in VFT(\mathcal{L}')} $ \\
	===================================================================================================================
\subsection{(Definition) Henkin ConstI}
	- $ConstI'$ is for any $c \in Const(\mathcal{L}')$, $\means{c}{\mathfrak{A}'} = [c]$ \\
	===================================================================================================================
\subsection{(Definition) Henkin FuncI}
	- $FuncI'$ is for any $f \in Func(\mathcal{L}')$, $\means{f}{\mathfrak{A}'}(\vdc{[t_i]}{i=1}{Arity(f)}) = [f \vdc{t_i}{i=1}{Arity(f)}]$ \\
	===================================================================================================================
\subsection{(Metatheorem) Henkin FuncI is a function} %=3.2.1.8
	- $Func(\means{f}{\mathfrak{A}'}, A'^{Arity(f)}, A')$ \\
	- Proof: \\
		-- For any $\set{\vdc{[t_i]}{i=1}{Arity(f)}}, \set{\vdc{[t'_i]}{i=1}{Arity(f)}} \subseteq A'$, \\
			--- If $\seq{\vdc{[t_i]}{i=1}{Arity(f)}} = \vdc{[t'_i]}{i=1}{Arity(f)}$, then \\
				---- $\vdash (\land \vdc{x_i \equiv y_i}{i=1}{Arity(f)}) \implies (f(\vdc{x_i}{i=1}{Arity(f)}) \equiv f(\vdc{y_i}{i=1}{Arity(f)}))$ \\ <E2>
				---- $E \is (\land \vdc{x_i \equiv y_i}{i=1}{Arity(f)}) \implies (f(\vdc{x_i}{i=1}{Arity(f)}) \equiv f(\vdc{y_i}{i=1}{Arity(f)}))$ \\
				---- $\vdash \vdc{\forall x_i}{i=1}{Arity(f)} \vdc{\forall y_i}{i=1}{Arity(f)} E$ \\ <Universal closure preserves deductiblity>
				---- For any $t_i, t'_i$, $Subbable(t_i, x_i, E)$ and $Subbable(t'_i, y_i, E)$ \\ <Definition>
				---- $\vdash \vdc{\forall x_i}{i=1}{Arity(f)} \vdc{\forall y_i}{i=1}{Arity(f)} E \implies \sub{\sub{E}{\vdc{x_i}{i=1}{Arity(f)}}{\vdc{t_i}{i=1}{Arity(f)}}}{\vdc{y_i}{i=1}{Arity(f)}}{\vdc{t'_i}{i=1}{Arity(f)}}$ \\ <Q1>
				---- $\vdash \sub{\sub{E}{\vdc{x_i}{i=1}{Arity(f)}}{\vdc{t_i}{i=1}{Arity(f)}}}{\vdc{y_i}{i=1}{Arity(f)}}{\vdc{t'_i}{i=1}{Arity(f)}}$ \\ <PC>
				---- $\vdash (\land \vdc{t_i \equiv t'_i}{i=1}{Arity(f)}) \implies (f(\vdc{t_i}{i=1}{Arity(f)}) \equiv f(\vdc{t'_i}{i=1}{Arity(f)}))$ \\ <Definition>
				---- $\vdash (\land \vdc{t_i \equiv t'_i}{i=1}{Arity(f)})$ \\
				---- $\vdash (f(\vdc{t_i}{i=1}{Arity(f)}) \equiv f(\vdc{t'_i}{i=1}{Arity(f)}))$ \\ <PC>
				---- $(f(\vdc{t_i}{i=1}{Arity(f)}) \equiv f(\vdc{t'_i}{i=1}{Arity(f)})) \in \Sigma'$ \\ <Expansion by chosen enumerated axioms is deductively closed>
				---- $(f(\vdc{t_i}{i=1}{Arity(f)}) \sim f(\vdc{t'_i}{i=1}{Arity(f)}))$ \\ <Definition>
				---- $[f(\vdc{t_i}{i=1}{Arity(f)})] = [f(\vdc{t'_i}{i=1}{Arity(f)})]$ \\ <Definition>
			--- If $\seq{\vdc{[t_i]}{i=1}{Arity(f)}} = \vdc{[t'_i]}{i=1}{Arity(f)}$, then $[f(\vdc{t_i}{i=1}{Arity(f)})] = [f(\vdc{t'_i}{i=1}{Arity(f)})]$ \\ <Abbreviate>
	===================================================================================================================
\subsection{(Definition) Henkin RelI}
	- $RelI'$ is for any $P \in Rel(\mathcal{L}')$, $\seq{\vdc{[t_i]}{i=1}{Arity(P)}} \in \means{P}{\mathfrak{A}'}$ iff $P \vdc{t_i}{i=1}{Arity(P)} \in \Sigma'$ \\
	===================================================================================================================
\subsection{(Metatheorem) Henkin RelI is a relation} %=3.2.1.8
	- $Rel(\means{P}{\mathfrak{A}}, A'^{Arity(P)})$ \\
	- Proof: \\
		-- If $\seq{\vdc{[t_i]}{i=1}{Arity(P)}} = \seq{\vdc{[t'_i]}{i=1}{Arity(P)}}$, then \\
			--- $P \vdc{t_i}{i=1}{Arity(P)} \in \Sigma'$ iff $P \vdc{t'_i}{i=1}{Arity(P)} \in \Sigma'$ \\
			--- $\seq{\vdc{[t_i]}{i=1}{Arity(P)}} \in \means{P}{\mathfrak{A}'}$ iff $\seq{\vdc{[t'_i]}{i=1}{Arity(P)}} \in \means{P}{\mathfrak{A}'}$ \\ <Definition>
	===================================================================================================================
\subsection{(Definition) Henkin structure}
	- $\mathfrak{A}'$ is the $\mathcal{L}'$-structure $\seq{A', ConstI', FuncI', RelI'}$ \\
	===================================================================================================================
\subsection{(Metatheorem) Henkin structure models Henkin theory: Proof lemma schema} %=3.2.6
	- Prove: (I) If $\sigma' \in Sent(\mathcal{L}')$, then $\sigma' \in \Sigma'$ iff $\mathfrak{A}' \vDash \sigma'$ \\
	- Corollaries: $\mathfrak{A}' \vDash \Sigma'$ \\
		-- For any $\sigma' \in \Sigma'$, \\
			--- $\Sigma' \vdash \sigma'$ \\ <Definition>
			--- $\Sigma' \vdash UC(\sigma')$ \\ <Universal closure preserves deductiblity>
			--- $UC(\sigma') \in \Sigma'$ \\ <Expansion by chosen enumerated axioms is deductively closed>
			--- $UC(\sigma') \in Sent(\mathcal{L}')$ \\ <Definition>
			--- $\mathfrak{A}' \vDash UC(\sigma')$ \\ <(I)>
			--- $\mathfrak{A}' \vDash \sigma'$ \\ <Universal closure preserves validity>
		-- For any $\sigma' \in \Sigma'$, $\mathfrak{A}' \vDash \sigma'$ \\ <Abbreviate>
		-- $\mathfrak{A}' \vDash \Sigma'$ \\ <Definition>
	===================================================================================================================
\subsection{(Metatheorem) VFT-universe assignment in Henkin structure} %=3.2.6
	- For any $t \in VFT(\mathcal{L}')$, for any variable-universe assignment $s$ of $\mathfrak{A}'$, $\extend{s}(t) = [t]$ \\
	- Proof: \\
		-- For any $t \in VFT(\mathcal{L}')$, \\
			--- If $t \is c$, then \\
				---- $c \in Const(\mathcal{L}')$ \\ <Definition>
				---- $\extend{s}(t) = $ \\
					----- $\extend{s}(c) = $ \\
					----- $\means{c}{\mathfrak{A}'} = $ \\ <Definition>
					----- $[c] = $ \\ <Definition>
					----- $[t]$ \\
				---- $\extend{s}(t) = [t]$ \\ <Abbreviate>
			--- If $t \is f \vdc{c_i}{i=1}{Arity(f)}$ and $\set{\vdc{c_i}{i=1}{Arity(f)}} \subseteq \set{z: \extend{s}(z) = [z]}$, then \\
				---- $\set{\vdc{c_i}{i=1}{Arity(f)}} \subseteq Const(\mathcal{L}')$ \\ <Definition>
				---- $\extend{s}(t) = $ \\
					----- $\extend{s}(f \vdc{c_i}{i=1}{Arity(f)}) = $ \\
					----- $\means{f}{\mathfrak{A}'}(\vdc{\extend{s}(c_i)}{i=1}{Arity(f)}) = $ \\ <Definition>
					----- $\means{f}{\mathfrak{A}'}(\vdc{[c_i]}{i=1}{Arity(f)}) = $ \\ <Inductive hypothesis>
					----- $[f \vdc{c_i}{i=1}{Arity(f)}] = $ \\ <Definition>
					----- $[t]$ \\
				---- $\extend{s}(t) = [t]$ \\ <Definition>
		-- For any $t \in VFT(\mathcal{L}')$, for any $s$, $\extend{s}(t) = [t]$ \\ <Induction>
	===================================================================================================================
\subsection{(Metatheorem) Henkin structure models Henkin theory: Lemma (I)} %=3.2.6
	- If $\sigma' \in Sent(\mathcal{L}')$, then $\sigma' \in \Sigma'$ iff $\mathfrak{A}' \vDash \sigma'$ \\
	- Proof: \\
		-- If $\sigma' \in Sent(\mathcal{L}')$, then \\
			--- If $\sigma' \is t_1 \equiv t_2$, then \\
				---- $\set{t_1, t_2} \subseteq VFT(\mathcal{L}')$ \\
				---- $\sigma' \in \Sigma'$ iff \\
					----- $t_1 \equiv t_2 \in \Sigma'$ iff \\
					----- $t_1 \sim t_2$ iff \\ <Definition>
					----- $[t_1] = [t_2]$ iff \\ <Definition>
					----- For any $s$, $\extend{s}(t_1) = \extend{s}(t_2)$ iff \\ <Definition>
					----- For any $s$, $\mathfrak{A}' \vDash (t_1 \equiv t_2)[s]$ \\ <Definition>
					----- $\mathfrak{A}' \vDash t_1 \equiv t_2$ iff \\
					----- $\mathfrak{A}' \vDash \sigma$ \\
				---- $\sigma' \in \Sigma'$ iff $\mathfrak{A}' \vDash \sigma'$ \\ <Abbreviate>
			--- If $\sigma' \is P \vdc{t_i}{i=1}{Arity(P)}$, then \\
				---- $\set{\vdc{t_i}{i=1}{Arity(P)}} \subseteq VFT(\mathcal{L}')$ \\
				---- $\sigma \in \Sigma'$ iff \\
					----- $P \vdc{t_i}{i=1}{Arity(P)} \in \Sigma'$ iff \\
					----- $\seq{\vdc{[t_i]}{i=1}{Arity(P)}} \in \means{P}{\mathfrak{A}'}$ iff \\ <Definition>
					----- $\mathfrak{A}' \vDash P \vdc{t_i}{i=1}{Arity(P)}$ \\ <Definition>
					----- $\mathfrak{A}' \vDash \sigma'$ \\
				---- $\sigma' \in \Sigma'$ iff $\mathfrak{A}' \vDash \sigma'$ \\ <Abbreviate>
			--- If $\sigma' \is \lnot \alpha$ and $\set{\alpha} \subseteq \set{\zeta: \zeta \in \Sigma' \txtiff \mathfrak{A}' \vDash \zeta}$, then \\
				---- $\sigma' \in \Sigma'$ iff \\
					----- $\lnot \alpha \in \Sigma'$ iff \\
					----- $\alpha \inot \in \Sigma'$ iff \\ <Expansion by chosen enumerated axioms is maximal>
					----- $\mathfrak{A}' \inot \vDash \alpha$ iff \\ <Inductive hypothesis>
					----- $\mathfrak{A}' \vDash \lnot \alpha$ iff \\ <Definition>
					----- $\mathfrak{A}' \vDash \sigma'$	\\
				---- $\sigma' \in \Sigma'$ iff $\mathfrak{A}' \vDash \sigma'$ \\ <Abbreviate>
			--- If $\sigma' \is \alpha \lor \beta$ and $\set{\alpha, \beta} \subseteq \set{\zeta: \zeta \in \Sigma' \txtiff \mathfrak{A}' \vDash \zeta}$, then \\
				---- $\mathfrak{A}' \vDash \sigma'$ iff \\
					----- $\mathfrak{A}' \vDash \alpha \lor \beta$ iff \\
					----- $\mathfrak{A}' \vDash \alpha$ or $\mathfrak{A}' \vDash \beta$ iff \\ <Definition>
					----- $\alpha \in \Sigma'$ or $\beta \in \Sigma'$ iff \\ <Inductive hypothesis>
					----- $\Sigma' \vdash \alpha$ or $\Sigma' \vdash \beta$ iff \\ <Expansion by chosen enumerated axioms is deductively closed>
					----- $\Sigma' \vdash \alpha \lor \beta$ iff \\ <PC>
					----- $\alpha \lor \beta \in \Sigma'$ iff \\ <Expansion by chosen enumerated axioms is deductively closed>
					----- $\sigma' \in \Sigma'$ \\
				---- $\sigma' \in \Sigma'$ iff $\mathfrak{A}' \vDash \sigma'$ \\ <Abbreviate>
			--- If $\sigma' \is \forall x \alpha$ and $Stage(Comp(\sigma') - 1) \subseteq \set{\zeta: \zeta \in \Sigma' \txtiff \mathfrak{A}' \vDash \zeta}$, then \\
				---- If $\sigma' \in \Sigma'$, then \\
					----- $\forall x \alpha \in \Sigma'$ \\
					----- $\Sigma' \vdash \forall x \alpha$ \\ <Expansion by chosen enumerated axioms is deductively closed>
					----- For any $t \in VFT(\mathcal{L}')$, \\
						------ $Subbable(t, x \alpha)$ \\ <Definition>
						------ $\vdash \forall x \alpha \implies \sub{\alpha}{x}{t}$ \\ <Q1>
						------ $\seq{\forall x \alpha, \forall x \alpha \implies \sub{\alpha}{x}{t}, \sub{\alpha}{x}{t}} \in PC$ \\
						------ $\Sigma' \cup \vdash \sub{\alpha}{x}{t}$ \\ <PC>
						------ $\sub{\alpha}{x}{t} \in \Sigma'$ \\ <Expansion by chosen enumerated axioms is deductively closed>
						------ $\mathfrak{A}' \vDash \sub{\alpha}{x}{t}$ \\ <Inductive hypothesis>
					----- For any $t \in VFT(\mathcal{L}')$, $\mathfrak{A}' \vDash \sub{\alpha}{x}{t}$ \\ <Abbreviate>
					-----	For any variable-universe assignment $s$, for any $[t] \in A'$, \\
						------ $t \in VFT(\mathcal{L}')$ \\ <Definition>
						------ $\mathfrak{A}' \vDash \sub{\alpha}{x}{t}$ \\
						------ $Subbable(t, x, \alpha)$ \\ <Definition>
						------ $\mathfrak{A}' \vDash \alpha[s[x|\extend{s}(t)]$ \\ <Substitution and modification identity on models>
						------ $\extend{s}(t) = [t]$ \\ <VFT-universe assignment in Henkin structure>
						------ $\mathfrak{A}' \vDash \alpha[s[x|[t]]]$ \\
					----- For any variable-universe assignment $s$, for any $[t] \in A'$, $\mathfrak{A}' \vDash \alpha[s[x|[t]]]$ \\ <Abbreviate>
					----- For any variable-universe assignment $s$, $\mathfrak{A}' \vDash (\forall x \alpha)[s]$ \\ <Definition>
					----- $\mathfrak{A}' \vDash \sigma'$ \\
				---- If $\sigma' \in \Sigma'$, then $\mathfrak{A}' \vDash \sigma'$ \\ <Abbreviate>
				---- If $\sigma' \inot \in \Sigma'$, then \\ 
					----- $\forall x \alpha \inot \in \Sigma'$ \\
					----- $\lnot \forall x \alpha \in \Sigma'$ \\ <Expansion by chosen enumerated axioms is maximal>
					----- $\exists x \lnot \alpha \in \Sigma'$ \\ <Definition>
					----- There exists $c_{(i, j)}$, $(\exists x \lnot \alpha \implies \sub{\lnot \alpha}{x}{c_{(i, j)}}) \in \Sigma'$ \\ <Definition>
					----- $\seq{\exists x \lnot \alpha, \exists x \lnot \alpha \implies \sub{\lnot \alpha}{x}{c_{(i, j)}}, \sub{\lnot \alpha}{x}{c_{(i, j)}}} \in PC$
					----- $\Sigma' \vdash \sub{\lnot \alpha}{x}{c_{(i, j)}}$ \\ <PC>
					----- $\sub{\lnot \alpha}{x}{c_{(i, j)}} \in \Sigma'$ \\ <Expansion by chosen enumerated axioms is deductively closed>
					----- $\mathfrak{A}' \vDash \sub{\lnot \alpha}{x}{c_{(i, j)}}$ \\ <Inductive hypothesis>
					----- There exists $s$, there exists $[t] \in A'$, \\
						------ $\mathfrak{A}' \vDash \sub{\lnot \alpha}{x}{c_{(i, j)}}[s]$ \\ <Definition>
						------ $Subbable(c_{(i, j)}, x, \lnot \alpha)$ \\ <Definition>
						------ $\mathfrak{A}' \vDash (\lnot \alpha)[s[x|\extend{s}(c_{(i, j)})]]$ \\ <Substitution and modification identity on models>
						------ $\extend{s}(c_{(i, j)} = [c_{(i, j)}]$ \\ <VFT-universe assignment in Henkin structure>
						------ $\mathfrak{A}' \vDash \lnot \alpha[s[x|[c_{(i, j)}]]]$ \\
						------ $\mathfrak{A}' \inot \vDash \alpha[s[x|[c_{(i, j)}]]]$ \\ <Definition>
						------ $[t] = [c_{(i, j)}]$ \\
						------ $\mathfrak{A}' \inot \vDash \alpha[s[x|[t]]]$ \\
					----- There exists $s$, there exists $[t] \in A'$, $\mathfrak{A}' \inot \vDash \alpha[s[x|[t]]]$ \\ <Abbreviate>
					----- There exists $s$, not for any $[t] \in A'$, $\mathfrak{A}' \vDash \alpha[s[x|[t]]]$ \\
					----- There exists $s$, not $\mathfrak{A}' \vDash (\forall x \in \alpha)[s]$ \\
					----- Not for any $s$, $\mathfrak{A}' \vDash (\forall x \in \alpha)[s]$ \\
					----- $\mathfrak{A}' \inot \vDash (\forall x \in \alpha)$ \\
					----- $\mathfrak{A}' \inot \vDash \sigma'$ \\
				---- If $\sigma' \inot \in \Sigma'$, then $\mathfrak{A}' \inot \vDash \sigma'$ \\ <Abbreviate>
				---- If $\mathfrak{A}' \vDash \sigma'$, then $\sigma' \in \Sigma'$ \\ <Contraposition>
				---- $\sigma' \in \Sigma'$ iff $\mathfrak{A}' \vDash \sigma'$ \\ <Conjunction>
			--- $\sigma' \in \Sigma'$ iff $\mathfrak{A}' \vDash \sigma'$ \\ <Induction>
		-- If $\sigma' \in Sent(\mathcal{L}')$, $\sigma' \in \Sigma'$ iff $\mathfrak{A}' \vDash \sigma'$ \\ <Abbreviate>
	===================================================================================================================
\subsection{(Definition) Structure reduct to a language}
	- The $\mathcal{L}$-structure $\mathfrak{A}^+ \upharpoonright_\mathcal{L}$ is the reduct of the $\mathcal{L}^+$-structure $\mathfrak{A}^+$ iff \\
		-- $\mathcal{L}$ is the restriction on constants of $\mathcal{L}^+$ \\
		-- $Universe(\mathfrak{A}^+ \upharpoonright_\mathcal{L}) = Universe(\mathfrak{A}^+)$ \\
		-- $ConstI(\mathfrak{A}^+ \upharpoonright_\mathcal{L})$ is for any $c \in Const(\mathcal{L})$, $\means{c}{\mathfrak{A}^+ \upharpoonright_\mathcal{L}} = \means{c}{\mathfrak{A}^+}$ \\
		-- $FuncI(\mathfrak{A}^+ \upharpoonright_\mathcal{L})$ is for any $f \in Func(\mathcal{L})$, $\means{f}{\mathfrak{A}^+ \upharpoonright_\mathcal{L}} = \means{f}{\mathfrak{A}^+}$ \\
		-- $RelI(\mathfrak{A}^+ \upharpoonright_\mathcal{L})$ is for any $P \in Rel(\mathcal{L})$, $\means{P}{\mathfrak{A}^+ \upharpoonright_\mathcal{L}} = \means{P}{\mathfrak{A}^+}$ \\
	===================================================================================================================
\subsection{(Metatheorem) Henkin structure reduct models consistent theory: Proof lemma schema} %=3.2.7
	- Prove: (I) If $\sigma \in Sent(\mathcal{L})$, then $\sigma \in \Sigma'$ iff $\mathfrak{A}' \upharpoonright_\mathcal{L} \vDash \sigma$ \\
	- Corollaries: $\mathfrak{A}' \upharpoonright_\mathcal{L} \vDash \Sigma$ \\
		-- For any $\sigma \in \Sigma$, \\
			--- $\sigma \in \Sigma'$ \\ <Definition>
			--- $\Sigma' \vdash \sigma$ \\ <Definition>
			--- $\Sigma' \vdash UC(\sigma)$ \\ <Universal closure preserves deductiblity>
			--- $UC(\sigma) \in \Sigma'$ \\ <Expansion by chosen enumerated axioms is deductively closed>
			--- $UC(\sigma) \in Sent(\mathcal{L})$ \\ <Definition>
			--- $\mathfrak{A}' \upharpoonright_\mathcal{L} \vDash UC(\sigma)$ \\ <(I)>
			--- $\mathfrak{A}' \upharpoonright_\mathcal{L} \vDash \sigma$ \\ <Universal closure preserves validity>
		-- For any $\sigma \in \Sigma$, $\mathfrak{A}' \upharpoonright_\mathcal{L} \vDash \sigma$ \\ <Abbreviate>
		-- $\mathfrak{A}' \upharpoonright_\mathcal{L} \vDash \Sigma$ \\ <Definition>
	===================================================================================================================
\subsection{(Metatheorem) VFT-universe assignment in Henkin structure reduct} %=3.2.7
	- For any $t \in VFT(\mathcal{L})$, for any variable-universe assignment $s$ of $\mathfrak{A}' \upharpoonright_\mathcal{L}$, $\extend{s}(t) = [t]$ \\
	- Proof: \\
		-- For any $t \in VFT(\mathcal{L})$, \\
			--- If $t \is c$, then \\
				---- $c \in Const(\mathcal{L})$ \\ <Definition>
				---- $\extend{s}(t) = $ \\
					----- $\extend{s}(c) = $ \\
					----- $\means{c}{\mathfrak{A}' \upharpoonright_\mathcal{L}} = $ \\ <Definiton>
					----- $\means{c}{\mathfrak{A}'} = $ \\ <Definition>
					----- $[c] = $ \\ <Definition>
					----- $[t]$ \\
				---- $\extend{s}(t) = [t]$ \\ <Abbreviate>
			--- If $t \is f \vdc{c_i}{i=1}{Arity(f)}$ and $\set{\vdc{c_i}{i=1}{Arity(f)}} \subseteq \set{z: \extend{s}(z) = [z]}$, then \\
				---- $\set{\vdc{c_i}{i=1}{Arity(f)}} \subseteq Const(\mathcal{L})$ \\ <Definition>
				---- $\extend{s}(t) = $ \\
					----- $\extend{s}(f \vdc{c_i}{i=1}{Arity(f)}) = $ \\
					----- $\means{f}{\mathfrak{A}' \upharpoonright_\mathcal{L}}(\vdc{\extend{s}(c_i)}{i=1}{Arity(f)}) = $ \\ <Definition>
					----- $\means{f}{\mathfrak{A}' \upharpoonright_\mathcal{L}}(\vdc{[c_i]}{i=1}{Arity(f)}) = $ \\ <Inductive hypothesis>
					----- $\means{f}{\mathfrak{A}'}(\vdc{[c_i]}{i=1}{Arity(f)}) = $ \\ <Definition>
					----- $[f \vdc{c_i}{i=1}{Arity(f)}] = $ \\ <Definition>
					----- $[t]$ \\
				---- $\extend{s}(t) = [t]$ \\ <Abbreviate>
		-- For any $t \in VFT(\mathcal{L})$, for any variable-universe assignment $s$ of $\mathfrak{A}' \upharpoonright_\mathcal{L}$, $\extend{s}(t) = [t]$ \\ <Induction>
	===================================================================================================================
\subsection{(Metatheorem) Henkin structure reduct models consistent theory: Lemma (I)}
	- If $\sigma \in Sent(\mathcal{L})$, then $\sigma \in \Sigma'$ iff $\mathfrak{A}' \upharpoonright_\mathcal{L} \vDash \sigma$ \\
	- Proof: \\
		-- If $\sigma \in Sent(\mathcal{L})$, then \\
			--- If $\sigma \is t_1 \equiv t_2$, then \\
				---- $\set{t_1, t_2} \subseteq VFT(\mathcal{L})$ \\
				---- $\sigma \in \Sigma'$ iff \\
					----- $t_1 \equiv t_2 \in \Sigma'$ iff \\
					----- $t_1 \sim t_2$ iff \\ <Definition>
					----- $[t_1] = [t_2]$ iff \\ <Definition>
					----- For any $s$, $\extend{s}(t_1) = \extend{s}(t_2)$ iff \\ <Definition>
					----- For any $s$, $\mathfrak{A}' \upharpoonright_\mathcal{L} \vDash (t_1 \equiv t_2)[s]$ \\ <Definition>
					----- $\mathfrak{A}' \upharpoonright_\mathcal{L} \vDash t_1 \equiv t_2$ iff \\
					----- $\mathfrak{A}' \upharpoonright_\mathcal{L} \vDash \sigma$ \\
				---- $\sigma \in \Sigma'$ iff $\mathfrak{A}' \upharpoonright_\mathcal{L} \vDash \sigma$ \\ <Abbreviate>
			--- If $\sigma \is P \vdc{t_i}{i=1}{Arity(P)}$, then \\
				---- $\set{\vdc{t_i}{i=1}{Arity(P)}} \subseteq VFT(\mathcal{L})$ \\
				---- $\sigma \in \Sigma'$ iff \\
					----- $P \vdc{t_i}{i=1}{Arity(P)} \in \Sigma'$ iff \\
					----- $\seq{\vdc{[t_i]}{i=1}{Arity(P)}} \in \means{P}{\mathfrak{A}'}$ iff \\ <Definition>
					----- $\mathfrak{A}' \vDash P \vdc{t_i}{i=1}{Arity(P)}$ iff \\ <Definition>
					----- $\mathfrak{A}' \upharpoonright_\mathcal{L} \vDash P \vdc{t_i}{i=1}{Arity(P)}$ iff \\ <Definition>
					----- $\mathfrak{A}' \upharpoonright_\mathcal{L} \vDash \sigma$ \\
				---- $\sigma \in \Sigma'$ iff $\mathfrak{A}' \upharpoonright_\mathcal{L} \vDash \sigma$ \\ <Abbreviate>
			--- If $\sigma \is \lnot \alpha$ and $\set{\alpha} \subseteq \set{\zeta: \zeta \in \Sigma' \txtiff \mathfrak{A}' \upharpoonright_\mathcal{L} \vDash \zeta}$, then \\
				---- $\sigma \in \Sigma'$ iff \\
					----- $\lnot \alpha \in \Sigma'$ iff \\
					----- $\alpha \inot \in \Sigma'$ iff \\ <Expansion by chosen enumerated axioms is maximal>
					----- $\mathfrak{A}' \upharpoonright_\mathcal{L} \inot \vDash \alpha$ iff \\ <Inductive hypothesis>
					----- $\mathfrak{A}' \upharpoonright_\mathcal{L} \vDash \lnot \alpha$ iff \\ <Definition>
					----- $\mathfrak{A}' \upharpoonright_\mathcal{L} \vDash \sigma$	\\
				---- $\sigma \in \Sigma'$ iff $\mathfrak{A}' \upharpoonright_\mathcal{L} \vDash \sigma$ \\ <Abbreviate>
			--- If $\sigma \is \alpha \lor \beta$ and $\set{\alpha, \beta} \subseteq \set{\zeta: \zeta \in \Sigma' \txtiff \mathfrak{A}' \upharpoonright_\mathcal{L} \vDash \zeta}$, then \\
				---- $\mathfrak{A}' \upharpoonright_\mathcal{L} \vDash \sigma$ iff \\
					----- $\mathfrak{A}' \upharpoonright_\mathcal{L} \vDash \alpha \lor \beta$ iff \\
					----- $\mathfrak{A}' \upharpoonright_\mathcal{L} \vDash \alpha$ or $\mathfrak{A}' \upharpoonright_\mathcal{L} \vDash \beta$ iff \\ <Definition>
					----- $\alpha \in \Sigma'$ or $\beta \in \Sigma'$ iff \\ <Inductive hypothesis>
					----- $\Sigma' \vdash \alpha$ or $\Sigma' \vdash \beta$ iff \\ <Expansion by chosen enumerated axioms is deductively closed>
					----- $\Sigma' \vdash \alpha \lor \beta$ iff \\ <PC>
					----- $\alpha \lor \beta \in \Sigma'$ iff \\ <Expansion by chosen enumerated axioms is deductively closed>
					----- $\sigma \in \Sigma'$ \\
				---- $\sigma \in \Sigma'$ iff $\mathfrak{A}' \upharpoonright_\mathcal{L} \vDash \sigma$ \\ <Abbreviate>
			--- If $\sigma \is \forall x \alpha$ and $Stage(Comp(\sigma) - 1) \subseteq \set{\zeta: \zeta \in \Sigma' \txtiff \mathfrak{A}' \upharpoonright_\mathcal{L} \vDash \zeta}$, then \\
				---- If $\sigma \in \Sigma'$, then \\
					----- $\forall x \alpha \in \Sigma'$ \\
					----- $\Sigma' \vdash \forall x \alpha$ \\ <Expansion by chosen enumerated axioms is deductively closed>
					----- For any $t \in VFT(\mathcal{L})$, \\
						------ $Subbable(t, x \alpha)$ \\ <Definition>
						------ $\vdash \forall x \alpha \implies \sub{\alpha}{x}{t}$ \\ <Q1>
						------ $\seq{\forall x \alpha, \forall x \alpha \implies \sub{\alpha}{x}{t}, \sub{\alpha}{x}{t}} \in PC$ \\
						------ $\Sigma' \cup \vdash \sub{\alpha}{x}{t}$ \\ <PC>
						------ $\sub{\alpha}{x}{t} \in \Sigma'$ \\ <Expansion by chosen enumerated axioms is deductively closed>
						------ $\mathfrak{A}' \upharpoonright_\mathcal{L} \vDash \sub{\alpha}{x}{t}$ \\ <Inductive hypothesis>
					----- For any $t \in VFT(\mathcal{L})$, $\mathfrak{A}' \upharpoonright_\mathcal{L} \vDash \sub{\alpha}{x}{t}$ \\ <Abbreviate>
					----- For any variable-universe assignment $s$, for any $[t] \in A'$, \\
						------ $t \in VFT(\mathcal{L})$ \\ <Definition>
						------ $\mathfrak{A}' \upharpoonright_\mathcal{L} \vDash \sub{\alpha}{x}{t}$ \\
						------ $Subbable(t, x, \alpha)$ \\ <Definition>
						------ $\mathfrak{A}' \upharpoonright_\mathcal{L} \vDash \alpha[s[x|\extend{s}(t)]$ \\ <Substitution and modification identity on models>
						------ $\extend{s}(t) = [t]$ \\ <VFT-universe assignment in Henkin structure reduct>
						------ $\mathfrak{A}' \upharpoonright_\mathcal{L} \vDash \alpha[s[x|[t]]]$ \\
					----- For any variable-universe assignment $s$, for any $[t] \in A'$, $\mathfrak{A}' \upharpoonright_\mathcal{L} \vDash \alpha[s[x|[t]]]$ \\ <Abbreviate>
					----- For any variable-universe assignment $s$, $\mathfrak{A}' \upharpoonright_\mathcal{L} \vDash (\forall x \alpha)[s]$ \\ <Definition>
					----- $\mathfrak{A}' \upharpoonright_\mathcal{L} \vDash \sigma'$ \\ <Definition>
				---- If $\sigma \in \Sigma'$, then $\mathfrak{A}' \upharpoonright_\mathcal{L} \vDash \sigma'$ \\ <Abbreviate>
				---- If $\sigma \inot \in \Sigma'$, then \\ 
					----- $\forall x \alpha \inot \in \Sigma'$ \\
					----- $\lnot \forall x \alpha \in \Sigma'$ \\ <Expansion by chosen enumerated axioms is maximal>
					----- $\exists x \lnot \alpha \in \Sigma'$ \\ <Definition>
					----- There exists $c_{(i, j)}$, $(\exists x \lnot \alpha \implies \sub{\lnot \alpha}{x}{c_{(i, j)}}) \in \Sigma'$ \\ <Definition>
					----- $\seq{\exists x \lnot \alpha, \exists x \lnot \alpha \implies \sub{\lnot \alpha}{x}{c_{(i, j)}}, \sub{\lnot \alpha}{x}{c_{(i, j)}}} \in PC$
					----- $\Sigma' \vdash \sub{\lnot \alpha}{x}{c_{(i, j)}}$ \\ <PC>
					----- $\sub{\lnot \alpha}{x}{c_{(i, j)}} \in \Sigma'$ \\ <Expansion by chosen enumerated axioms is deductively closed>
					----- $\mathfrak{A}' \upharpoonright_\mathcal{L} \vDash \sub{\lnot \alpha}{x}{c_{(i, j)}}$ \\ <Inductive hypothesis>
					----- There exists $s$, there exists $[t] \in A'$, \\
						------ $\mathfrak{A}' \upharpoonright_\mathcal{L} \vDash \sub{\lnot \alpha}{x}{c_{(i, j)}}[s]$ \\ <Definition>
						------ $Subbable(c_{(i, j)}, x, \lnot \alpha)$ \\ <Definition>
						------ $\mathfrak{A}' \upharpoonright_\mathcal{L} \vDash (\lnot \alpha)[s[x|\extend{s}(c_{(i, j)})]]$ \\ <Substitution and modification identity on models>
						------ $\extend{s}(c_{(i, j)} = [c_{(i, j)}]$ \\ <VFT-universe assignment in Henkin structure reduct>
						------ $\mathfrak{A}' \upharpoonright_\mathcal{L} \vDash \lnot \alpha[s[x|[c_{(i, j)}]]]$ \\
						------ $\mathfrak{A}' \upharpoonright_\mathcal{L} \inot \vDash \alpha[s[x|[c_{(i, j)}]]]$ \\ <Definition>
						------ $[t] = [c_{(i, j)}]$ \\
						------ $\mathfrak{A}' \upharpoonright_\mathcal{L} \inot \vDash \alpha[s[x|[t]]]$ \\
					----- There exists $s$, there exists $[t] \in A'$, $\mathfrak{A}' \upharpoonright_\mathcal{L} \inot \vDash \alpha[s[x|[t]]]$ \\ <Abbreviate>
					----- There exists $s$, not for any $[t] \in A'$, $\mathfrak{A}' \upharpoonright_\mathcal{L} \vDash \alpha[s[x|[t]]]$ \\
					----- There exists $s$, not $\mathfrak{A}' \upharpoonright_\mathcal{L} \vDash (\forall x \in \alpha)[s]$ \\
					----- Not for any $s$, $\mathfrak{A}' \upharpoonright_\mathcal{L} \vDash (\forall x \in \alpha)[s]$ \\
					----- $\mathfrak{A}' \upharpoonright_\mathcal{L} \inot \vDash (\forall x \in \alpha)$ \\
					----- $\mathfrak{A}' \upharpoonright_\mathcal{L} \inot \vDash \sigma'$ \\
				---- If $\sigma' \inot \in \Sigma'$, then $\mathfrak{A}' \upharpoonright_\mathcal{L} \inot \vDash \sigma'$ \\ <Abbreviate>
				---- If $\mathfrak{A}' \upharpoonright_\mathcal{L} \vDash \sigma'$, then $\sigma' \in \Sigma'$ \\ <Contraposition>
				---- $\sigma' \in \Sigma'$ iff $\mathfrak{A}' \upharpoonright_\mathcal{L} \vDash \sigma'$ \\ <Conjunction>
			--- $\sigma \in \Sigma'$ iff $\mathfrak{A}' \upharpoonright_\mathcal{L} \vDash \sigma$ \\ <Induction>
		-- If $\sigma \in Sent(\mathcal{L})$, $\sigma \in \Sigma'$ iff $\mathfrak{A}' \upharpoonright_\mathcal{L} \vDash \sigma$ \\ <Abbreviate>
	===================================================================================================================
\subsection{(Metatheorem) Completeness of First-order Logic: Lemma (I)}
	- If $UC(\Sigma) \inot \vdash \contr$, then there exists $\mathfrak{A}$, $\mathfrak{A} \vDash UC(\Sigma)$ \\
	- Proof: \\
		-- If $UC(\Sigma) \inot \vdash \contr$, then \\
			--- $UC(\Sigma) \subseteq Sent(\mathcal{L})$ \\
			--- $UC(\Sigma)'$ is consistent, deductively closed, maximal \\
			--- $\mathfrak{B}' \vDash UC(\Sigma)'$ \\ <Henkin structure models Henkin theory>
			--- $\mathfrak{B}' \upharpoonright_\mathcal{L} \vDash UC(\Sigma)$ \\ <Henkin structure reduct models consistent theory>
			--- $\mathfrak{A} = \mathfrak{B}' \upharpoonright_\mathcal{L}$ \\
			--- There exists $\mathfrak{A}$, $ \mathfrak{A} \vDash UC(\Sigma)$ \\
		-- If $UC(\Sigma) \inot \vdash \contr$, then there exists $\mathfrak{A}$, $\mathfrak{A} \vDash UC(\Sigma)$ \\ <Abbreviate>
	===================================================================================================================
\subsection{(Metatheorem) Completeness for uncountable language} %=3.2.1.6
	- Countable language assumption only affects Henkin theory construction
	- TODO VERIFY: ANNOTATIONS!!! %% https://proofwiki.org/wiki/Definition:Upper_Bound_of_Set https://proofwiki.org/wiki/Definition:Maximal/Ordered_Set
	- If $\mathcal{L}$ is uncountable, then \\
		-- $\mathcal{L}'$ is uncountable \\
		-- $\hat{\Sigma}$ is uncountable \\
		-- $\Sigma'$ is uncountable \\ TODO: FIX WHY COUNTABLE
	- $\Sigma_{all} = \set{\hat{\Sigma} \cup \Sigma_{ext}: \hat{\Sigma} \cup \Sigma_{ext} \inot \vdash \contr}$ \\
	- $Poset(\Sigma_{all}, \subseteq)$ \\
	- For any $T$, if $T \subseteq \Sigma_{all}$ and $Woset(T, \subseteq)$, then there exists $\Sigma_{ub}$, $UB(\Sigma_{ub}, T, \hat{\Sigma}, \subseteq)$ \\
		-- $\Sigma_{ub} = \hat{\Sigma} \cup \vdc{\Sigma_{ext}^t}{t \in T}{ }$ \\
	- There exists $\Sigma_{max}$, $Max(\Sigma_{max}, \Sigma_{all}, \subseteq)$ \\ <Zorn's lemma>
	- $\Sigma_{max}$ is consistent, deductively closed, maximal \\
	- $\mathfrak{A}_{max} \vDash \Sigma_{max}$ \\
	- $\mathfrak{A}_{max} \upharpoonright_\mathcal{L} \vDash \Sigma$ \\
	===================================================================================================================
\subsection{(Metatheorem) Contradiction explosion} %=3.2.1.1
	- If $\Gamma \vDash \contr$, then $\Gamma \vDash \phi$ \\
	- Proof: \\
		-- If $\Gamma \vDash \contr$, then \\
			--- $\contr \vDash_{PC} \phi$ \\
			--- $\contr \vdash \phi$ \\ <PC>
			--- $\Gamma \vdash \phi$ \\
	===================================================================================================================

\section{Compactness}
	===================================================================================================================
\subsection{(Metatheorem) Compactness theorem} %=3.3.1
	- $\Sigma \inot \vDash \contr$ iff for any $\Gamma$, if $\Gamma \subseteq \Sigma$ and $Finite(\Gamma)$, then $\Gamma \inot \vDash \contr$ \\
	- Proof: \\
		-- If $\Sigma \inot \vDash \contr$, then \\
			--- There exists $\mathfrak{A}$, $\mathfrak{A} \vDash \Sigma$ \\
			--- For any $\Gamma$, if $\Gamma \subseteq \Sigma$ and $Finite(\Gamma)$, then \\
				---- $\mathfrak{A} \vDash \Gamma$ \\ <Definition>
				---- $\Gamma \inot \vDash \contr$ \\ <Definition>
			--- For any $\Gamma$, if $\Gamma \subseteq \Sigma$ and $Finite(\Gamma)$, then $\Gamma \inot \vDash \contr$ \\ <Abbreviate>
		-- If $\Sigma \vDash \contr$, then \\
			--- $\Sigma \vdash \contr$ \\ <Completeness theorem>
			--- There exists $\Sigma_{fin}$, $\Sigma_{fin} \subseteq \Sigma$ and $Finite(\Sigma_{fin})$ and $\Sigma_{fin} \vdash \contr$ \\ <DEDUCTIONS ARE FINITE>
			--- $\Sigma_{fin} \vDash \contr$ \\ <Soundness theorem>
			--- $\Gamma = \Sigma_{fin}$ \\
			--- There exists $\Gamma$, ($\Gamma \subseteq \Sigma$ and $Finite(\Gamma)$) and $\Gamma \vDash \contr$ \\
			--- Not for any $\Gamma$, not (($\Gamma \subseteq \Sigma$ and $Finite(\Gamma)$) and $\Gamma \vDash \contr$) \\
			--- Not for any $\Gamma$, not ($\Gamma \subseteq \Sigma$ and $Finite(\Gamma)$) or not $\Gamma \vDash \contr$ \\
			--- Not for any $\Gamma$, if $\Gamma \subseteq \Sigma$ and $Finite(\Gamma)$, then not $\Gamma \vDash \contr$ \\
			--- Not for any $\Gamma$, if $\Gamma \subseteq \Sigma$ and $Finite(\Gamma)$, then $\Gamma \inot \vDash \contr$ \\
		-- If $\Sigma \vDash \contr$, then not for any $\Gamma$, if $\Gamma \subseteq \Sigma$ and $Finite(\Gamma)$, then $\Gamma \inot \vDash \contr$ \\ <Abbreviate>
		-- If for any $\Gamma$, if $\Gamma \subseteq \Sigma$ and $Finite(\Gamma)$, then $\Gamma \inot \vDash \contr$, then $\Sigma \inot \vDash \contr$ \\ <Contraposition>
		-- $\Sigma \inot \vDash \contr$ iff for any $\Gamma$, if $\Gamma \subseteq \Sigma$ and $Finite(\Gamma)$, then $\Gamma \inot \vDash \contr$ <Conjunction>
	===================================================================================================================
\subsection{(Metatheorem) Logical implication takes finite hypotheses} %=3.3.2
	- $\Sigma \vDash \phi$ iff there exists $\Sigma_{fin}$, $Finite(\Sigma_{fin})$ and $\Sigma_{fin} \subseteq \Sigma$ and $\Sigma_{fin} \vDash \phi$ \\
	- Proof: \\
		-- $\Sigma \vDash \phi$ iff \\
			--- $\Sigma \vdash \phi$ iff \\ <Completeness theorem, Soundness theorem>
			--- There exists $\Sigma_{fin}$, $Finite(\Sigma_{fin})$ and $\Sigma_{fin} \subseteq \Sigma$ and $\Sigma_{fin} \vdash \phi$ iff \\ <DEDUCTION ARE FINITE>
			--- There exists $\Sigma_{fin}$, $Finite(\Sigma_{fin})$ and $\Sigma_{fin} \subseteq \Sigma$ and $\Sigma_{fin} \vDash \phi$ iff \\ <Soundness theorem, Completeness theorem>
	===================================================================================================================
\subsection{(Definition) Theory of a structure} %=3.3.4
	- The theory of the $\mathcal{L}$-structure $\mathfrak{A}$ is $Th(\mathfrak{A}) = \set{\phi \in \mathcal{L}: \mathfrak{A} \vDash \phi}$ \\
	===================================================================================================================
\subsection{(Definition) Elementary equivalent structures} %=3.3.4
	- The $\mathcal{L}$-structures $\mathfrak{A}, \mathfrak{B}$ are elementary equivalent ($\mathfrak{A} =_E \mathfrak{B}$) iff $Th(\mathfrak{A}) = Th(\mathfrak{B})$ \\
	===================================================================================================================

\section{Substructures and the Lowenheim-Skolem theorems}
	===================================================================================================================
\subsection{(Definition) Function restriction} %=3.4.1
	- The function $f \upharpoonright_A: A \rightarrow C$ is a restriction of the function $f: A \cup B \rightarrow C$ iff \\
		-- For any $a \in A$, $f \upharpoonright_A(a) = f(a)$ \\
	===================================================================================================================
\subsection{(Definition) Substructure} %=3.4.1
	- The $\mathcal{L}$-structure $\mathfrak{A}$ is a substructure of the $\mathcal{L}$-structure $\mathfrak{B}$ ($\mathfrak{A} \subseteq \mathfrak{B}$) iff \\
		-- $A \subseteq B$ and \\
		-- For any $c \in Const$, $\means{c}{\mathfrak{A}} = \means{c}{\mathfrak{B}}$ and \\
		-- For any $f \in Func$, $\means{f}{\mathfrak{A}} = \means{f}{\mathfrak{B}} \upharpoonright_{A^{Arity(f)}}$ and \\
		-- For any $P \in Rel$, $\means{P}{\mathfrak{A}} = \means{P}{\mathfrak{B}} \cap A^{Arity(P)}$ and \\
		-- $\mathfrak{A}$ is an $\mathcal{L}$-structure \\
	===================================================================================================================
\subsection{(Metatheorem) Stronger substructure} %=3.4.1
	- If $\emptyset \neq A \subset B$ and for any $c \in Const$, $\means{c}{\mathfrak{B}} \in A$ and for any $f \in Func$, $\means{f}{\mathfrak{B}} \upharpoonright_{A^{Arity(f)}}: A^{Arity(f)} \rightarrow A$, then $\mathfrak{A}_{A, \mathfrak{B}} \subseteq \mathfrak{B}$ \\
	- Proof: definition \\
	===================================================================================================================
\subsection{(Definition) Elementary substructure} %=3.4.4
	- The $\mathcal{L}$-structure $\mathfrak{A}$ is an elementary substructure of the $\mathcal{L}$-structure $\mathfrak{B}$ ($\mathfrak{A} \prec \mathfrak{B}$) iff \\
		-- $\mathfrak{A} \subseteq \mathfrak{B}$ and \\
		-- For any $\phi \in Form(\mathcal{L})$, for any $s: Var \rightarrow A$, $\mathfrak{A} \vDash \phi[s]$ iff $\mathfrak{B} \vDash \phi[s]$ \\
	- %% TODO: Chaff: Notice that if we want to prove A ≺ B, we need only prove A |= φ[s] → B |= φ[s], since once we have done that, the other direction comes for free by using the contrapositive and negations.
	===================================================================================================================
\subsection{(Metatheorem) Elementary substructure property} %=3.4.5
	- If $\mathfrak{A} \prec \mathfrak{B}$, then for any $\phi \in Sent(\mathcal{L})$, $\mathfrak{A} \vDash \phi$ iff $\mathfrak{B} \vDash \phi$ \\
		-- Proof: \\
			--- If $\mathfrak{A} \prec \mathfrak{B}$, then \\
				---- For any $\chi \in Form(\mathcal{L})$, for any $s: Var \rightarrow A$, $\mathfrak{A} \vDash \chi[s]$ iff $\mathfrak{B} \vDash \chi[s]$ \\ <Definition>
				---- $\phi \in Form(\mathcal{L})$ \\
				---- For any $s: Var \rightarrow A$, $\mathfrak{A} \vDash \phi[s]$ iff $\mathfrak{B} \vDash \phi[s]$ \\
				---- $\mathfrak{A} \vDash \phi$ iff \\
					----- For any $s: Var \rightarrow A$, $\mathfrak{A} \vDash \phi[s]$ iff \\ <Definition>
					----- For any $s: Var \rightarrow A$, $\mathfrak{B} \vDash \phi[s]$ iff \\
					----- For any $s: Var \rightarrow B$, $\mathfrak{B} \vDash \phi[s]$ iff \\ <Sentences have fixed truth>
					----- $\mathfrak{B} \vDash \phi$ \\ <Definition>
				---- $\mathfrak{A} \vDash \phi$ iff $\mathfrak{B} \vDash \phi$ <Abbreviate>
	===================================================================================================================
\subsection{(Metatheorem) Stronger elementary substructure} %=3.4.7
	- If ($\mathfrak{A} \subseteq \mathfrak{B}$ and for any $\gamma \in Form(\mathcal{L})$, for any $s: Var \rightarrow A$, if $\mathfrak{B} \vDash (\exists x \gamma)[s]$, then there exists $a \in A$, $\mathfrak{B} \vDash \gamma[s[x|a]]$), then $\mathfrak{A} \prec \mathfrak{B}$ \\
	- Proof: \\
		-- If ($\mathfrak{A} \subseteq \mathfrak{B}$ and for any $\gamma \in Form(\mathcal{L})$, for any $s: Var \rightarrow A$, if $\mathfrak{B} \vDash (\exists x \gamma)[s]$, then there exists $a \in A$, $\mathfrak{B} \vDash \gamma[s[x|a]]$), then \\
			--- $\mathfrak{A} \subseteq \mathfrak{B}$ \\ <Hypothesis>
			--- $A \subseteq B$ <(1)> \\ <Definition>
			--- If $s: Var \rightarrow A$, then $s: Var \rightarrow B$ <(2)> \\ <Definition>
			--- If $P \in Rel$, then $\means{P}{\mathfrak{A}} = \means{P}{\mathfrak{B}} \cap A^{Arity(P)}$ <(3)> \\ <Definition>
			--- For any $\gamma \in Form(\mathcal{L})$, for any $s: Var \rightarrow A$, if $\mathfrak{B} \vDash (\exists x \gamma)[s]$, then there exists $a \in A$, $\mathfrak{B} \vDash \gamma[s[x|a]]$ <(4)> \\ <Hypothesis>
			--- If $\phi \is t_1 \equiv t_2$, then \\
				---- For any $s: Var \rightarrow A$, \\
					----- $\mathfrak{A} \vDash \phi[s]$ iff \\
						------ $\mathfrak{A} \vDash (t_1 \equiv t_2)[s]$ iff \\ <Definition>
						------ $\extend{s}(t_1) = \extend{s}(t_2)$ iff \\ <Definition>
						------ $\mathfrak{B} \vDash (t_1 \equiv t_2)[s]$ iff \\ <(2)>
						------ $\mathfrak{B} \vDash \phi[s]$ \\
					----- $\mathfrak{A} \vDash \phi[s]$ iff $\mathfrak{B} \vDash \phi[s]$ \\ <Abbreviate>
				---- For any $s: Var \rightarrow A$, $\mathfrak{A} \vDash \phi[s]$ iff $\mathfrak{B} \vDash \phi[s]$ \\ <Abbreviate>
			--- If $\phi \is P \vdc{t_i}{i=1}{Arity(P)}$, then \\
				---- For any $s: Var \rightarrow A$, \\
					----- $\mathfrak{A} \vDash \phi[s]$ iff \\
						------ $\mathfrak{A} \vDash (P \vdc{t_i}{i=1}{Arity(P)})[s]$ iff \\ <Definition>
						------ $\seq{\vdc{\extend{s}(t_i)}{i=1}{Arity(P)}} \in \means{P}{\mathfrak{A}}$ \\ <Definition>
						------ $\seq{\vdc{\extend{s}(t_i)}{i=1}{Arity(P)}} \in \means{P}{\mathfrak{B}}$ \\ <(3)>
						------ $\mathfrak{B} \vDash (P \vdc{t_i}{i=1}{Arity(P)})[s]$ iff \\ <Definition>
						------ $\mathfrak{B} \vDash \phi[s]$ \\ <Definition>
					----- $\mathfrak{A} \vDash \phi[s]$ iff $\mathfrak{B} \vDash \phi[s]$ \\ <Abbreviate>
				---- For any $s: Var \rightarrow A$, $\mathfrak{A} \vDash \phi[s]$ iff $\mathfrak{B} \vDash \phi[s]$ \\ <Abbreviate>
			--- If $\phi \is \lnot \alpha$ and $\set{\alpha} \subseteq \set{\zeta: \txtforall{s: Var \rightarrow A}{\mathfrak{A} \vDash \zeta[s] \txtiff \mathfrak{B} \vDash \zeta[s]}}$, then \\
				---- For any $s: Var \rightarrow A$, \\
					----- $\mathfrak{A} \vDash \phi[s]$ iff \\
						------ $\mathfrak{A} \vDash (\lnot \alpha)[s]$ iff \\ <Definition>
						------ $\mathfrak{A} \inot \vDash \alpha[s]$ iff \\ <Definition>
						------ $\mathfrak{B} \inot \vDash \alpha[s]$ iff \\ <Inductive hypothesis>
						------ $\mathfrak{B} \vDash (\lnot \alpha)[s]$ iff \\ <Definition>
						------ $\mathfrak{B} \vDash \phi[s]$ \\ <Definition>
					----- $\mathfrak{A} \vDash \phi[s]$ iff $\mathfrak{B} \vDash \phi[s]$ \\ <Abbreviate>
				---- For any $s: Var \rightarrow A$, $\mathfrak{A} \vDash \phi[s]$ iff $\mathfrak{B} \vDash \phi[s]$ \\ <Abbreviate>
			--- If $\phi \is \alpha \lor \beta$ and $\set{\alpha, \beta} \subseteq \set{\zeta: \txtforall{s: Var \rightarrow A}{\mathfrak{A} \vDash \zeta[s] \txtiff \mathfrak{B} \vDash \zeta[s]}}$, then \\
				---- For any $s: Var \rightarrow A$, \\
					----- $\mathfrak{A} \vDash \phi[s]$ iff \\
						------ $\mathfrak{A} \vDash (\alpha \lor \beta)[s]$ iff \\ <Definition>
						------ $\mathfrak{A} \vDash \alpha[s]$ or $\mathfrak{A} \vDash \beta[s]$ iff \\ <Definition>
						------ $\mathfrak{B} \vDash \alpha[s]$ or $\mathfrak{B} \vDash \beta[s]$ iff \\ <Inductive hypothesis>
						------ $\mathfrak{B} \vDash (\alpha \lor \beta)[s]$ iff \\ <Definition>
						------ $\mathfrak{B} \vDash \phi[s]$ \\ <Definition>
					----- $\mathfrak{A} \vDash \phi[s]$ iff $\mathfrak{B} \vDash \phi[s]$ \\ <Abbreviate>
				---- For any $s: Var \rightarrow A$, $\mathfrak{A} \vDash \phi[s]$ iff $\mathfrak{B} \vDash \phi[s]$ \\ <Abbreviate>
			--- If $\phi \is \exists x \alpha$ and $\set{\alpha}	\subseteq \set{\zeta: \txtforall{s: Var \rightarrow A}{\mathfrak{A} \vDash \zeta[s] \txtiff \mathfrak{B} \vDash \zeta[s]}}$, then \\
				---- For any $s: Var \rightarrow A$, \\
					----- If $\mathfrak{A} \vDash \phi[s]$, then \\
						------ $\mathfrak{A} \vDash (\exists x \alpha)[s]$ \\
						------ There exists $a \in A$, $\mathfrak{A} \vDash \alpha[s[x|a]]$ \\ <Definition>
						------ $\mathfrak{B} \vDash \alpha[s[x|a]]$ \\ <Inductive hypothesis>
						------ $a \in B$ \\ <(I)>
						------ There exists $a \in B$, $\mathfrak{B} \vDash \alpha[s[x|a]]$ \\ <Conjunction>
						------ $\mathfrak{B} \vDash (\exists x \alpha)[s]$ \\ <Definition>
						------ $\mathfrak{B} \vDash \phi[s]$ \\
					----- If $\mathfrak{A} \vDash \phi[s]$, then $\mathfrak{B} \vDash \phi[s]$ \\ <Abbreviate>
					----- If $\mathfrak{B} \vDash \phi[s]$, then \\
						------ There exists $a \in A$, $\mathfrak{B} \vDash \alpha[s[x|a]]$ \\ <(4)>
						------ $\mathfrak{A} \vDash \alpha[s[x|a]]$ \\ <Inductive hypothesis>
						------ There exists $a \in A$, $\mathfrak{A} \vDash \alpha[s[x|a]]$ \\ <Conjunction>
						------ $\mathfrak{A} \vDash (\exists x \alpha)[s]$ \\ <Definition>
						------ $\mathfrak{A} \vDash \phi[s]$ \\
					----- If $\mathfrak{B} \vDash \phi[s]$, then $\mathfrak{A} \vDash \phi[s]$ \\ <Abbreviate>
					----- $\mathfrak{A} \vDash \phi[s]$ iff $\mathfrak{B} \vDash \phi[s]$ \\ <Conjunction>
				---- For any $s: Var \rightarrow A$, $\mathfrak{A} \vDash \phi[s]$ iff $\mathfrak{B} \vDash \phi[s]$ \\ <Abbreviate>
			--- For any $s: Var \rightarrow A$, $\mathfrak{A} \vDash \phi[s]$ iff $\mathfrak{B} \vDash \phi[s]$ \\ <Induction>
			--- $\mathfrak{A} \prec \mathfrak{B}$ \\ <Definition>
	===================================================================================================================
\subsection{(Definition) TODO Countable/finite/infinite notations}
	- $Finite(X)$ iff $|X| \in \mathbb{N}$ \\
	- $Infinite(X)$ iff not $Finite(X)$ \\
	- $Countable(X)$ iff there exists $f$, $Bij(f, X, \mathbb{N})$ \\
	- $Countable_L(\mathcal{L})$ iff $Countable(Form(\mathcal{L}))$ \\
	- $Countable_S(\mathfrak{A})$ iff $Countable(Universe(\mathfrak{A}))$ \\
	- Cardinal = set cardinality \\
	===================================================================================================================
\subsection{(Metatheorem) Downward Lowenheim-Skolem theorem} %=3.4.8
	- If $Countable_L(\mathcal{L})$ and $\mathfrak{B}$ is an $\mathcal{L}$-structure, then there exists $\mathfrak{A}$, $\mathfrak{A} \prec \mathfrak{B}$ and $Countable_S(\mathfrak{A})$ \\
	- Proof: TODO ABSTRACTED \\
	===================================================================================================================
\subsection{(Metatheorem) PLACEHOLDER} %=3.4.10
	- If $\kappa$ is an infinite cardinal and there exists $\mathfrak{A}$, $\mathfrak{A} \vDash \Sigma$ and $Infinite_S(\mathfrak{A})$, then there exists $\mathfrak{B}$, $\mathfrak{B} \vDash \Sigma$ and $|B| \geq \kappa$ \\
	- Proof: TODO ABSTRACTED \\
	===================================================================================================================
\subsection{(Metatheorem) PLACEHOLDER} %=3.4.11
	- If $\kappa$ is an infinite cardinal and $Countable_L(\mathcal{L})$ and $\Sigma \subseteq Form(\mathcal{L})$ and there exists $\mathfrak{A}$, $\mathfrak{A} \vDash \Sigma$ and $Infinite_S(\mathfrak{A})$, then there exists $\mathfrak{B}$, $\mathfrak{B} \vDash \Sigma$ and $|B| = \kappa$ \\
	- Proof: TODO ABSTRACTED \\
	======================================================================================================
\subsection{(Metatheorem) PLACEHOLDER} %=3.4.12
	- If $Infinite_S(\mathfrak{A})$, then not there exists $\Sigma$, $\mathfrak{B} \vDash \Sigma$ iff $\mathfrak{A} \cong \mathfrak{B}$ \\
	- Proof: TODO ABSTRACTED \\
	======================================================================================================
\subsection{(Metatheorem) Upward Lowenheim-Skolem theorem} %=3.4.13
	- If $Countable_L(\mathcal{L})$ and $Infinite_S(\mathfrak{A})$ and $\kappa$ is a cardinal, then there exists $\mathfrak{B}$, $\mathfrak{A} \prec \mathfrak{B}$ and $|B| \geq \kappa$ \\
	- Proof: TODO ABSTRACTED \\
	======================================================================================================


\chapter{Incompleteness From Two Points of View}
	===================================================================================================================
\section{Introduction}
	- $\mathcal{L}$ is cool and all, but how about $\mathcal{L}_{NT}$ and $\mathfrak{N}$? \\
	- Can we find some way for any $\phi \in Form(\mathcal{L}_{NT})$, if $\mathfrak{N} \vDash \phi$, then $\Sigma \vdash \phi$ (complete) such that $\Sigma$ is consistent and decidable? \\
	===================================================================================================================
\subsection{(Definition) Axiomatic completeness} %=4.1.1
	- $\Sigma$ is axiomatically complete iff for any $\sigma \in Form(\mathcal{L})$, $\Sigma \vdash \sigma$ or $\Sigma \vdash \lnot \sigma$ \\
	===================================================================================================================
\subsection{(Definition) Axiomatization} %=4.1.2
	- $\Sigma$ is an axiomatization of $Th(\mathfrak{A})$ iff for any $\sigma \in Th(\mathfrak{A})$, $\Sigma \vdash \sigma$ \\
	- Promise: Given any complete, consistent, and decidable axiomatization for $\mathfrak{N}$ ($\Sigma$), we are going to find a sentence $\sigma$ such that $\mathfrak{N} \vDash \sigma$ but $\Sigma \inot \vdash \sigma$ \\
	===================================================================================================================

\section{Complexity of Formulas}
	- We will find this Godel sentence via complexity of formulas \\
	===================================================================================================================
\subsection{(Definition) Bounded quantifiers} %=4.2.1
	- If $\pnot{\occurs{x}{t}}$, then the following are bounded quantifiers: \\
		-- $(\forall x \leq t) \phi \is \forall x (x \leq t \implies t)$ \\
		-- $(\exists x \leq t) \phi \is \exists x (x \leq t \land t)$ \\
	===================================================================================================================
\subsection{(Definition) Sigma-formulas} %=4.2.2
	- $\Sigma_{Form}$ is defined as the smallest set of $\mathcal{L}_{NT}$ formulas that contains: \\
		-- Atomic formulas \\
		-- If $\alpha \in \Sigma_{Form}$, then $\lnot \alpha \in \Sigma_{Form}$ \\
		-- If $\set{\alpha, \beta} \subseteq \Sigma_{Form}$, then $\set{\alpha \lor \beta, \alpha \land \beta} \subseteq \Sigma_{Form}$ \\
		-- If $\alpha \in \Sigma_{Form}$ and $\pnot{\occurs{x}{t}}$, then $\set{(\forall x < t) \alpha, (\exists x < t) \alpha} \subseteq \Sigma_{Form}$ \\
		-- If $\alpha \in \Sigma_{Form}$ and $x \in Var$, then $\exists x \alpha \in \Sigma_{Form}$ \\
	- There are closed under bounded quantification + unbounded existential quantification \\
	- These are not complicated enough to establish incompleteness \\
	===================================================================================================================
\subsection{(Definition) Pi-formulas} %=4.2.3
	- $\Pi_{Form}$ is defined as the smallest set of $\mathcal{L}_{NT}$ formulas that contains: \\
		-- Atomic formulas \\
		-- If $\alpha \in \Sigma_{Form}$, then $\lnot \alpha \in \Sigma_{Form}$ \\
		-- If $\set{\alpha, \beta} \subseteq \Sigma_{Form}$, then $\set{\alpha \lor \beta, \alpha \land \beta} \subseteq \Sigma_{Form}$ \\
		-- If $\alpha \in \Sigma_{Form}$ and $\pnot{\occurs{x}{t}}$, then $\set{(\forall x < t) \alpha, (\exists x < t) \alpha} \subseteq \Sigma_{Form}$ \\
		-- If $\alpha \in \Sigma_{Form}$ and $x \in Var$, then $\forall	x \alpha \in \Sigma_{Form}$ \\
	- There are closed under bounded quantification + unbounded universal quantification \\
	- These are complicated enough to establish incompleteness \\
	===================================================================================================================
\subsection{(Definition) Delta-formulas} %=4.2.4
	- $\Delta_{Form} = \Sigma_{Form} \cap \Pi_{Form}$ \\
	===================================================================================================================
	TODO: REMARKS, EXERCISES

\section{The Roadmap to Incompleteness}
	- Key idea: use numbers to encode deductions, then construct a self-reference paradoxical deduction \\
	- It is easy to encode, decode, validate numbers into deductions and vice versa \\
	- Promise: fix our coding scheme, prove that the coding is nice, use the coding scheme in order to construct the formula $\sigma$, and then prove that $\sigma$ is both true and not provable \\
	===================================================================================================================

\section{An Alternate Route}
	- Instead of looking at formulas and deductions, we can look at computations \\
	- In this route, we will still encode computations are numbers \\
	===================================================================================================================

\section{How to Code a Sequence of Numbers}
	- We will use prime numbers with non-zero exponents \\
	===================================================================================================================
\subsection{Prime number function} %=4.5.1
	- The function $p: \mathbb{N} \rightarrow \mathbb{N}$ is defined as $p(k)$ is the $k$th prime number \\
	- $p(0) = 1, p(1) = 2, p(2) = 3, p(3) = 4, ..., p_i = p(i)$ \\
	===================================================================================================================
\subsection{Set of finite sequences of natural numbers} %=4.5.2
	- The set $\mathbb{N}^{<\mathbb{N}}$ is the set of all finite sequences of natural numbers \\
	===================================================================================================================
\subsection{Encoding function} %=4.5.3
	- The encoding function $enc: \mathbb{N}^{<\mathbb{N}} \rightarrow \mathbb{N}$ is defined as: \\
		-- If $k > 0$, then $enc(\vdc{a_i}{i=1}{k}) = \Pi_{i=1}^{k}(p_i^{a_i + 1})$ \\
		-- Otherwise, then $enc() = 1$ \\
	===================================================================================================================
\subsection{Code numbers} %=4.5.4
	- The set code numbers $C$ is defined as $C = \set{enc(s): s \in \mathbb{N}^{<\mathbb{N}}}$ \\
	- This is easy to check \\
	===================================================================================================================
\subsection{Decoding function} %=4.5.4
	- The decoding function $dec: \mathbb{N} \rightarrow \mathbb{N}^{<\mathbb{N}}$ is defined as: \\
		-- If $a \in C$, then \\
			--- There exists $\vdc{a_i}{i=1}{k}$, $a = enc(\vdc{a_i}{i=1}{k})$ \\ <Fundamental theorem of arithmetic + Definition>
			--- $dec(a) = \seq{\vdc{a_i}{i=1}{k}}$ \\
		-- Otherwise, then $dec(a) = \seq{}$ \\
	===================================================================================================================
\subsection{Length function} %=4.5.5
	- The length function $len: \mathbb{N} \rightarrow \mathbb{N}$ is defined as: \\
		-- If $a \in C$, then \\
			--- There exists $\vdc{a_i}{i=1}{k}$, $a = enc(\vdc{a_i}{i=1}{k})$ \\ <Fundamental theorem of arithmetic + Definition>
			--- $len(a) = k$ \\
		-- Otherwise, then $len(a) = 0$ \\
	- The Fundamental theorem of arithmetic ensures that for any positive integer, there exists is a unique prime factorization \\
	===================================================================================================================
\subsection{Index function} %=4.5.6
	- The index function $idx: \mathbb{N}^2 \rightarrow \mathbb{N}$ is defined as: \\
		-- If $a \in C$, then \\
			--- There exists $\vdc{a_i}{i=1}{k}$, $a = enc(\vdc{a_i}{i=1}{k})$ \\ <Fundamental theorem of arithmetic + Definition>
			--- If $1 \leq i \leq k$, then $idx(a, i) = a_i$ \\
			--- Otherwise, $idx(a, i) = 0$ \\
		-- Otherwise, then $idx(a, i) = 0$ \\
	===================================================================================================================
\subsection{Concatenate function} %=4.5.7
	- The concatenate function $cat: \mathbb{N}^2 \rightarrow \mathbb{N}$ is defined as: \\
		-- If $a \in C$ and $b \in C$, then \\
			--- There exists $\vdc{a_i}{i=1}{k}$, $a = enc(\vdc{a_i}{i=1}{k})$ \\ <Fundamental theorem of arithmetic + Definition>
			--- There exists $\vdc{b_i}{i=1}{k}$, $b = enc(\vdc{b_i}{i=1}{k})$ \\ <Fundamental theorem of arithmetic + Definition>
			--- $cat(a, b) = enc(\vdc{a_i}{i=1}{k_a}), \vdc{b_i}{i=1}{k_b}))$ \\
		-- Otherwise, then $cat(a, b) = 0$ \\
	===================================================================================================================

\section{An Old Friend}
	- $N$ is strong enough to prove every true sentence in $\Sigma_{Form}$, but it is not strong enough to prove every true sentence in $\Pi_{Form}$ \\
		- Proof: TODO ABSTRACTED \\
	===================================================================================================================

\subsection{(Definition) Goden numbering function} %=5.7.1
	- $GN: String(\mathcal{L}_{NT}) \rightarrow \mathbb{N}$ is defined as: \\
	-- If $s \in Form(\mathcal{L}_{NT})$ and $s \is \lnot \alpha$,       then $GN(s) = enc(1, GN(\alpha))$ \\
	-- If $s \in Form(\mathcal{L}_{NT})$ and $s \is \alpha \lor \beta$,  then $GN(s) = enc(3, GN(\alpha), GN(\beta))$ \\
	-- If $s \in Form(\mathcal{L}_{NT})$ and $s \is \forall v_i \alpha$, then $GN(s) = enc(5, GN(v_i), GN(\alpha))$ \\
	-- If $s \in Form(\mathcal{L}_{NT})$ and $s \is \equiv t_1 t_2$,     then $GN(s) = enc(7, GN(t_1), GN(t_2))$ \\
	-- If $s \in Form(\mathcal{L}_{NT})$ and $s \is < t_1 t_2$,          then $GN(s) = enc(19, GN(t_1), GN(t_2))$ \\
	-- If $s \in Term(\mathcal{L}_{NT})$ and $s \is S(t)$,               then $GN(s) = enc(11, GN(t))$ \\
	-- If $s \in Term(\mathcal{L}_{NT})$ and $s \is + t_1 t_2$,          then $GN(s) = enc(13, GN(t_1), GN(t_2))$ \\
	-- If $s \in Term(\mathcal{L}_{NT})$ and $s \is \centerdot t_1 t_2$, then $GN(s) = enc(15, GN(t_1), GN(t_2))$ \\
	-- If $s \in Term(\mathcal{L}_{NT})$ and $s \is E t_1 t_2$,          then $GN(s) = enc(17, GN(t_1), GN(t_2))$ \\
	-- If $s \in Var(\mathcal{L}_{NT})$ and $s \is v_i$,                 then $GN(s) = enc(2i)$ \\
	-- If $s \in Const(\mathcal{L}_{NT})$ and $s \is 0$,                 then $GN(s) = enc(9)$ \\
	-- Otherwise,                                                             $GN(s) = 3$ \\
	===================================================================================================================


\chapter{Computability Theory}
	===================================================================================================================
\section{The Origin of Computability Theory} %=7.1
	- Computability theory formalizes the notion of algorithms and computations \\
	- The goal is to create formal models of computation and study its limitations \\
	- Several models of note: Herbrand-Godel equations, Church's lambda-calculus, Kleene recursion, Turing machines \\
	- It's easy to see that if a function is computable in these models, then it is computable in the real-world, but the converse is not so clear \\
	- Turing machines model computation similar to how we do computations in the real-world, so maybe the converse holds (Church-Turing thesis) \\
	- All models mentioned induce the same class of computable functions \\
	===================================================================================================================
\section{The Basics} %=7.2
		- We will use Kleene recursion because it is easy to use in proofs \\
	===================================================================================================================
\subsection{(Definition) Computable functions} %=7.2.1
	- The set of computable functions $\mu$ is defined by: \\
		-- Zero function: If $\mathcal{O}: \emptyset \rightarrow \set{0}$ and $O() = 0$, then $O \in \mu$ \\
		-- Successor function: If $S: \mathbb{N} \rightarrow \mathbb{N}$ and $S(x) = x + 1$, then $S \in \mu$ \\
		-- Projection function: If $1 \leq i \leq n$ and $\mathcal{I}_i^n: \mathbb{N}^n \rightarrow \mathbb{N}$ and $\mathcal{I}_i^n(\vdc{x_j}{j=1}{n}) = x_i$, then $\mathcal{I}_i^n \in \mu$ \\
		-- Composition: If $h: \mathbb{N}^m \rightarrow \mathbb{N}$ and for any $i \in \set{\vdc{j}{j=1}{n}}$, $g_i: \mathbb{N}^n \rightarrow \mathbb{N}$ and $\set{h, \vdc{g_i}{i=1}{n}} \subseteq \mu$ and $f: \mathbb{N}^n \rightarrow \mathbb{N}$ and $f(\vdc{x_j}{j=1}{n}) = h(\vdc{g_i(\vdc{x_j}{j=1}{n})}{i=1}{m})$, then $f \in \mu$ \\
		-- Primitive recursion: If $g: \mathbb{N}^n \rightarrow \mathbb{N}$ and $h: \mathbb{N}^{n+2} \rightarrow \mathbb{N}$ and $\set{g, h} \subseteq \mu$ and $f: \mathbb{N}^{n+1} \rightarrow \mathbb{N}$ and $f(\vdc{x_i}{i=1}{n}, 0) = g(\vdc{x_i}{i=1}{n})$ and $f(\vdc{x_i}{i=1}{n}, y+1) = h(\vdc{x_i}{i=1}{n}, y, f(\vdc{x_i}{i=1}{n}, y))$, then $f \in \mu$ \\
		-- Minimalization: If ($g: \mathbb{N}^{n+1} \rightarrow \mathbb{N}$ and $g \in \mu$ and $\mu_{UBS}(g): \mathbb{N}^n \rightarrow \mathbb{N}$ and if (there exists $z$, $g(\vdc{x_i}{i=1}{n}, z) = 0$ and for any $z_- < z$, $g(\vdc{x_i}{i=1}{n}, z_-) \inot = 0$), then $\mu_{US}(g)(\vdc{x_i}{i=1}{n}) = z$), then $\mu_{US} \in \mu$ \\
	- Projection and composition can simulate arbitrary function arities \\
	- Minimalization is also called unbounded search and it can possibly be undefined which introduces partial functions \\
	- Partial functions are important in computability theory \\
	- When we claim that an algorithm computes a partial function $f: \mathbb{N}^n \rightarrow \mathbb{N}$, we claim that $f(\vdc{x_i}{i=1}{n})$ is defined iff the algorithm terminates on the inputs and returns the correct output \\
	===================================================================================================================
\subsection{(Definition) Primitive recursive functions} %=7.2.2
	- The set of primitive recursive $PR$ is defined by the definition of computable functions without Minimalization \\
	===================================================================================================================
\subsection{(Definition) Characteristic function} %=7.2.3
	- The characteristic function $\chi_{A(\placeholder)}: \mathbb{N}^n \rightarrow \set{0, 1}$ for $A \subseteq \mathbb{N}^n$ and $n>1$ is defined as: \\
		-- If $\seq{\vdc{x_i}{i=1}{n}} \in A$, then $\chi_{A(\placeholder)}(\vdc{x_i}{i=1}{n}) = 0$ \\
		-- If $\seq{\vdc{x_i}{i=1}{n}} \inot \in A$, then $\chi_{A(\placeholder)}(\vdc{x_i}{i=1}{n}) = 1$ \\
	- $\placeholder$ is a place holder or an abbreviation for exactly the same input arguments if it is defined \\
	===================================================================================================================
\subsection{(Definition) Computable set/relation} %=7.2.4
	- The set/relation $A$ is computable iff its characteristic function $\chi_{A(\placeholder)}$ is computable \\
	- The set/relation $A$ is primitive recursive iff its characteristic function $\chi_{A(\placeholder)}$ is recursive \\
	===================================================================================================================

\subsection{(Metatheorem) Constant function is primitive recursive} %=7.3.1
	- The constant function $c_i^n(\vdc{x_j}{j=1}{n}) = i$ is primitive recursive \\
	- Proof: \\
		-- If $i = 0$, then \\
			--- $c_0^n(\vdc{x_j}{j=1}{n}) = 0 = \mathcal{O}()$ \\ <Zero function>
			--- $c_0^n \in PR$ \\ <Composition> 
		-- If $i > 0$ and $c_i^n \in PR$, then \\
			--- $S \in PR$ \\ <Successor function>
			--- $c_{i+1}^n(\vdc{x_j}{j=1}{n}) = S(c_i^n(\vdc{x_j}{j=1}{n}))$ \\
			--- $c_{i+1}^n(\vdc{x_j}{j=1}{n}) \in PR$ \\ <Composition>
		-- $c_i^n \in PR$ \\ <Induction>
		-- $c_i^n(\vdc{x_i}{i=1}{n}) = i$ \\ <Definition>
	- The approach is not a construction via primitive recursion because $i$ is not treated as a function argument \\
	===================================================================================================================
\subsection{(Metatheorem) Standard addition, multiplication, exponentiaion are primitive recursive} %=7.3.2
	- The functions $+, \centerdot, E$ from the standard number theory ($\mathcal{N}$) are primitive recursive \\
	- $+ \in PR$ \\
	- Proof: \\
		-- $I_1^1(x) = x$ \\
		-- $I_1^1 \in PR$ \\ <Projection function>
		-- $S \in PR$ \\ <Successor function>
		-- $S_1^3(x, y, z) = S(I_1^3(x, y, z))$ \\
		-- $S_1^3 \in PR$ \\ <Composition>
		-- $+(x, 0) = I_1^1(x)$ \\
		-- $+(x, y+1) = S_1^3(x, y, +(x, y))$ \\
		-- $+ \in PR$ \\ <Primitive recursion>
	- $\centerdot \in PR$ \\
	- Proof: \\
		-- $c_0^1(x) = 0$ \\ <Definition>
		-- $c_0^1 \in PR$ \\ <Constant function is primitive recursive>
		-- $+ \in PR$ \\ <Standard addition, multiplication, exponentiaion are primitive recursive>
		-- $+_1^3(x, y, z) = +(I_1^3(x, y, z), I_3^3(x, y, z))$ \\
		-- $+_1^3 \in PR$ \\ <Composition>
		-- $\centerdot(x, 0) = c_0^1(x)$ \\
		-- $\centerdot(x, y+1) = +_1^3(x, y, \centerdot(x, y))$ \\
		-- $\centerdot \in PR$ \\ <Primitive recursion>
	- $E \in PR$ \\
	- Proof: \\
		-- $c_1^1(x) = 1$ \\ <Definition>
		-- $c_1^1 \in PR$ \\ <Constant function is primitive recursive>
		-- $\centerdot \in PR$ \\ <Standard addition, multiplication, exponentiaion are primitive recursive>
		-- $\centerdot_1^3(x, y, z) = \centerdot(I_1^3(x, y, z), I_3^3(x, y, z))$ \\
		-- $\centerdot_1^3 \in PR$ \\ <Composition>
		-- $E(x, 0) = c_1^1(x)$ \\
		-- $E(x, y+1) = \centerdot_1^3(x, y, E(x, y))$ \\
		-- $E \in PR$ \\ <Primitive recursion>
	===================================================================================================================
\subsection{(Metatheorem) Modified subtraction is primitive recursive} %=7.3.3
	- The modified subtraction function $\dot{-}$ is defined as: \\
		-- If $y > x$, then $x \dot{-} y = 0$ \\
		-- If $y \inot > x$, then	$x \dot{-} y = x - y$ \\
	- $\dot{-} \in PR$ \\
	- Proof: \\ 
		-- $\mathcal{O} \in PR$ \\
		-- $I_1^2 \in PR$ \\
		-- $P(0) = \mathcal{O}()$ \\
		-- $P(y+1) = I_1^2(y, P(y))$ \\
		-- $P \in PR$ \\ <Primitive recursion>
		-- $I_1^1 \in PR$ \\
		-- $P_1^3(x, y, z) = P(I_1^3(x, y, z))$ \\
		-- $P_1^3 \in PR$ \\ <Composition>
		-- $\dot{-}(x, 0) = I_1^1(x)$ \\
		-- $\dot{-}(x, y+1) = P_1^3(x, y, \dot{-}(x, y))$ \\
		-- $\dot{-} \in PR$ \\ <Primitive recursion>
	===================================================================================================================
\subsection{(Metatheorem) Standard logic connectives are closed under the primitive recursion} %=7.3.4
	- The relations $\lnot, \lor$ from the standard propositional logic ($\mathcal{PL}$) are closed under primitive recursion \\
	- For any $\set{\chi_{U(\placeholder)}, \chi_{V(\placeholder)}} \subseteq PR$, $\set{\chi_{\lnot U(\placeholder)}, \chi_{U(\placeholder) \lor V(\placeholder)}} \subseteq PR$ \\
	- Proof: \\
		-- For any $\chi_{U(\placeholder)} \in PR$, \\
			--- $\dot{-} \in PR$ \\ <Modified subtraction is primitive recursive>
			--- $Conj(x) = \dot{-}(c_1^1(x), I_1^1(x))$ \\
			--- $Conj \in PR$ \\ <Composition>
			--- $\chi_{\lnot U(\placeholder)}(\vdc{x_i}{i=1}{Arity(U)}) = Conj(\chi_{U(\placeholder)}(\vdc{x_i}{i=1}{Arity(U)}))$ \\
			--- $\chi_{\lnot U(\placeholder)} \in PR$ \\
		-- For any $\set{\chi_{U(\placeholder)}, \chi_{V(\placeholder)}} \subseteq PR$, \\
			--- $\centerdot \in PR$ \\ <Standard addition, multiplication, exponentiaion are primitive recursive>
			--- $\chi_{U(\placeholder)}'(\vdc{r_i}{i=1}{Arity(U)}, \vdc{s_i}{i=1}{Arity(V)}) = \chi_{U(\placeholder)}(\vdc{I_j^{Arity(U)+Arity(V)}(\vdc{r_i}{i=1}{Arity(U)}, \vdc{s_i}{i=1}{Arity(V)})}{j=1}{Arity(U)})$ \\
			--- $\chi_{U(\placeholder)}' \in PR$ \\ <Composition>
			--- $\chi_{V(\placeholder)}'(\vdc{r_i}{i=1}{Arity(U)}, \vdc{s_i}{i=1}{Arity(V)}) = \chi_{V(\placeholder)}(\vdc{I_j^{Arity(U)+Arity(V)}(\vdc{r_i}{i=1}{Arity(U)}, \vdc{s_i}{i=1}{Arity(V)})}{j=Arity(U)+1}{Arity(U)+Arity(V)})$ \\
			--- $\chi_{V(\placeholder)}' \in PR$ \\ <Composition>
			--- $\chi_{U(\placeholder) \lor V(\placeholder)}(\vdc{r_i}{i=1}{Arity(U)}, \vdc{s_i}{i=1}{Arity(V)}) = \centerdot(\chi_{U(\placeholder)}'(\vdc{r_i}{i=1}{Arity(U)}, \vdc{s_i}{i=1}{Arity(V)}), \chi_{V(\placeholder)}'(\vdc{r_i}{i=1}{Arity(U)}, \vdc{s_i}{i=1}{Arity(V)}))$ \\
			--- $\chi_{U(\placeholder) \lor V(\placeholder)} \in PR$ \\ <Composition>
	===================================================================================================================
\subsection{(Metatheorem) Standard ordering relations are primitive recursive} %=7.3.5
	- The relations $\chi_{\leq(\placeholder)}, \chi_{<(\placeholder)}, \chi_{=(\placeholder)}$ from the standard number theory ($\mathcal{N}$) are primitive recursive \\
	- $\chi_{\leq(\placeholder)} \in PR$ \\
	- Proof: \\
		-- $\seq{c_1^2, \dot{-}, +, \chi_{\lnot \leq(\placeholder)}, \chi_{\leq(\placeholder) \land \leq(\placeholder)}} \in PR$ \\ <Misc. theorems>
		-- $\chi_{x \leq y}(x, y) = 1 \dot{-} ((y+1) \dot{-} x)$ \\ <Informal>
		-- $\chi_{\leq(\placeholder)} \in PR$ \\
	- $\chi_{<(\placeholder)} \in PR$ \\
	- Proof: \\
		-- $\chi_{x < y}(x, y) = \chi_{\lnot (y \leq x)}$ \\ <Informal>
		-- $\chi_{<(\placeholder)} \in PR$ \\
	- $\chi_{<(\placeholder)} \in PR$ \\
	- Proof: \\
		-- $\chi_{x = y}(x, y) = \chi_{x \leq y \land y \leq x}(x, y)$ \\ <Informal>
		-- $\chi_{=(\placeholder)} \in PR$ \\
	===================================================================================================================
\subsection{(Metatheorem) Bounded sums and products are closed under the primitive recursion} %=7.3.6
	- If $f: \mathbb{N}^{n+1} \rightarrow \mathbb{N} \in PR$, then $Sum(f): \mathbb{N}^{n+1} \rightarrow \mathbb{N} \subseteq PR$ \\
	- Proof: \\
		-- If $f \in PR$, then \\
			--- $Sum(f)(\vdc{x_i}{i=1}{n}, 0) = f(\vdc{x_i}{i=1}{n}, 0)$ \\ <Informal>
			--- $Sum(f)(\vdc{x_i}{i=1}{n}, y+1) = f(\vdc{x_i}{i=1}{n}, y+1) + Sum(f)(\vdc{x_i}{i=1}{n}, y)$ \\ <Informal>
			--- $Sum(f) \in PR$ \\ <Primitive recursion>
	- If $f: \mathbb{N}^{n+1} \rightarrow \mathbb{N} \in PR$, then $Prod(f): \mathbb{N}^{n+1} \rightarrow \mathbb{N} \subseteq PR$ \\
	- Proof: \\
		-- If $f \in PR$, then \\
			--- $Prod(f)(\vdc{x_i}{i=1}{n}, 0) = f(\vdc{x_i}{i=1}{n}, 0)$ \\ <Informal>
			--- $Prod(f)(\vdc{x_i}{i=1}{n}, y+1) = f(\vdc{x_i}{i=1}{n}, y+1) \centerdot Prod(f)(\vdc{x_i}{i=1}{n}, y)$ \\ <Informal>
			--- $Prod(f) \in PR$ \\ <Primitive recursion>
	===================================================================================================================
\subsection{(Metatheorem) Bounded quantifiers are closed under the primitive recursion} %=7.3.7
	- If $\chi_{P(\placeholder)}: \mathbb{N}^{n+1} \rightarrow \mathbb{N} \in PR$, then $\chi_{(\exists i \leq m) P(\placeholder)}: \mathbb{N}^{n+1} \rightarrow \mathbb{N} \in PR$ \\
	- Proof: \\
		-- $Prod_{\chi_{P(\placeholder)}} \in PR$ \\ <Bounded sums and products are closed under the primitive recursion>
		-- $\chi_{(\exists i \leq m) P(\placeholder)}(\vdc{x_j}{j=1}{n}, m) = Prod_{\chi_{P(\placeholder)}}(\vdc{x_j}{j=1}{n}, m)$ \\ <Informal>
		-- $\chi_{(\exists i \leq m) P(\placeholder)} \in PR$ \\ <Composition>
	- If $\chi_{P(\placeholder)} \in PR$, then $\chi_{(\forall i \leq m) P(\placeholder)}: \mathbb{N}^{n+1} \rightarrow \mathbb{N} \in PR$ \\
	- Proof: \\
		-- $\chi_{(\exists i \leq m) P(\placeholder)} \in PR$ \\ <Bounded quantifiers are closed under the primitive recursion>
		-- $\chi_{\lnot (\exists i \leq m) \lnot P(\placeholder)} \in PR$ \\ <Standard logic connectives are closed under the primitive recursion>
		-- $\chi_{(\forall i \leq m) P(\placeholder)}(\vdc{x_j}{j=1}{n}, m) = \chi_{\lnot (\exists i \leq m) \lnot P(\placeholder)}(\vdc{x_j}{j=1}{n}, y)$ \\ <Informal>
		-- $\chi_{(\forall i \leq m) P(\placeholder)} \in PR$ \\ <Composition>
	===================================================================================================================
\subsection{(Definition) Definition by cases} %=7.3.8
	- The function $f: \mathbb{N}^n \rightarrow \mathbb{N}$ is defined by cases using the functions $h, g_1, g_2: \mathbb{N}^n \rightarrow \mathbb{N}$ iff \\
		-- If $h(\vdc{x_i}{i=1}{n}) = 0$, then $f(\vdc{x_i}{i=1}{n}) = g_1(\vdc{x_i}{i=1}{n})$ and \\
		-- Otherwise, $f(\vdc{x_i}{i=1}{n}) = g_2(\vdc{x_i}{i=1}{n})$ \\
	===================================================================================================================
\subsection{(Metatheorem) Definition by cases is closed under the primitive recursion} %=7.3.9
	- If $\set{h, g_1, g_2} \subseteq PR$ and $f$ is defined by cases using $h, g_1, g_2$, then $f \in PR$ \\
	- Proof: \\
		-- If $\set{h, g_1, g_2} \subseteq PR$ and $f$ is defined by cases using $h, g_1, g_2$, then \\
			--- $\set{\chi_{h(\placeholder) = 0}, Conj, \centerdot, +} \subseteq PR$ \\ <Misc. theorems>
			--- $f(\vdc{x_i}{i=1}{n}) = Conj(\chi_{h(\placeholder) = 0}(\vdc{x_i}{i=1}{n})) \centerdot g_1(\vdc{x_i}{i=1}{n}) + \chi_{h(\placeholder) = 0}(\vdc{x_i}{i=1}{n}) \centerdot g_2(\vdc{x_i}{i=1}{n})$ \\ <Informal>
			--- $f \in PR$ \\ <Composition>
	===================================================================================================================
\subsection{(Definition) Bounded minimalization} %=7.3.10
	- The function $\mu_{BS}(g): \mathbb{N}^{n+1} \rightarrow \mathbb{N}$ is a bounded minimalization using the function $g: \mathbb{N}^{n+1} \rightarrow \mathbb{N}$ iff \\
		-- If there exists $i \leq y$, $g(\vdc{x_j}{j=1}{n}, i) = 0$ and for any $j < i$, $g(\vdc{x_j}{j=1}{n}, j) \inot = 0$, then $\mu_{BS}(g)(\vdc{x_j}{j=1}{n}, y) = i$ \\
		-- Otherwise, $\mu_{BS}(g)(\vdc{x_j}{j=1}{n}, y) = y+1$ \\
	===================================================================================================================
\subsection{(Metatheorem) Bounded minimalization is closed under the primitive recursion} %=7.3.11
	- If $g: \mathbb{N}^{n+1} \rightarrow \mathbb{N} \in PR$, then $\mu_{BS}(g) \in PR$ \\
	- Proof: \\
		-- If $g: \mathbb{N}^{n+1} \rightarrow \mathbb{N} \in PR$, then \\
			--- $\set{\chi_{(\exists i \leq y) (g(\placeholder) = 0)}, Sum(\chi_{(\exists i \leq y) (g(\placeholder) = 0)})} \subseteq PR$ \\ <Misc. theorems>
			--- $\mu_{BS}(g)(\vdc{x_j}{j=1}{n}, y) = Sum(\chi_{(\exists i \leq y) (g(\placeholder) = 0)})(\vdc{x_j}{j=1}{n}, y)$ \\ <Informal>
			--- $\mu_{BS} \in PR$ \\ <Composition>
	- $\mu_{BS}(g)(\vdc{x_j}{j=1}{n}, y) = Sum(\chi_{(\exists i \leq y) (g(\placeholder) = 0)})(\vdc{x_j}{j=1}{n}, y)$ \\
	- Proof: \\
		-- If there exists $i \leq y$, $g(\vdc{x_j}{j=1}{n}, i) = 0$ and for any $j < i$, $g(\vdc{x_j}{j=1}{n}, j) \inot = 0$, then \\
			--- For any $a < i$, $\chi_{(\exists i \leq y) (g(\placeholder) = 0)}(\vdc{x_j}{j=1}{n}, a) = 1$ \\
			--- For any $i \leq b \leq y$, $\chi_{(\exists i \leq y) (g(\placeholder) = 0)}(\vdc{x_j}{j=1}{n}, b) = 0$ \\
			--- $Sum(\chi_{(\exists i \leq y) (g(\placeholder) = 0)})(\vdc{x_i}{i=1}{n}, y) = \sum_{z=0}^{i-1}(1) + \sum_{z=i}^{y}(0) = i$ \\
		-- Otherwise, \\
			--- $Sum(\chi_{(\exists i \leq y) (g(\placeholder) = 0)})(\vdc{x_i}{i=1}{n}, y) = \sum_{z=0}^{y}(1) = y+1$ \\
	- Note that the occurrence of $y$ in $\chi_{(\exists i \leq y)}$ also varies with $y \in Sum$ \\
	===================================================================================================================
\subsection{(Metatheorem) Prime number function is the primitive recursive} %=7.3.12
	- The prime number function $p \in PR$ \\
	- Proof: \\
		-- $NotPrime(x)$ iff $\lnot(2 \leq x \land (\forall y \leq x) (\forall z \leq x) ((y+2) \centerdot (z+2) \inot = x))$ \\
		-- $NumPrimesLeq(x) = Sum(\chi_{NotPrime(x)})(x)$ \\
		-- $p(n)$ as definition by cases: \\
			--- If $I_1^1(n) = 0$, then $p(n) = 1$ \\
			--- Otherwise, $p(n) = \mu_{BS}(\chi_{NumPrimesLeq(\placeholder) = n})(2^{2^n})$ \\ <N-th prime is bounded by 2\string^(2\string^n)>
		-- $p \in PR$ \\ <Misc. theorems>
	===================================================================================================================
\subsection{(Definition) Prime factor index function} %=7.3.13
	- The prime factor index function $\pi_i$ returns the exponent of the $i$th prime factor in its unique prime factorization \\
	- $\pi_i(n)$ as definition by cases: \\
		-- If $\chi_{n \leq 1}(n) = 0$, $\pi_i(n) = 0$ \\
		-- Otherwise, $\pi_i(n) = \mu_{BS}(\chi_{(\exists x \leq n) (x \centerdot p(i) E \placeholder = n) \land (\forall x \leq n) (x \centerdot p(i) E (\placeholder+1) \inot = n)})(n)$ \\
	===================================================================================================================
\subsection{(Metatheorem) Prime factor index function is primitive recursive} %=7.3.13
	- For any $i > 0$, $\pi_i: \mathbb{N} \rightarrow \mathbb{N} \in PR$ \\
	- Proof: all functions used are in PR or closed under PR \\ <Misc. theorems>
	===================================================================================================================
\subsection{(Metatheorem) SingleDec, length, isCodeFor functions are primitive recursive} %=7.3.14
	- For any $\seq{\vdc{a_i}{i=1}{n}}$, there exists $a \in \mathbb{N}$, there are the following primitive recursive functions: \\
		-- $len(a) = n$ \\
		-- $singleDec_j(a) = a_j$ \\
		-- $\set{len, singleDec_j} \subseteq PR$ \\
	- $isCodeFor(a, \vdc{a_i}{i=1}{n})$ iff $len(a) = n$ and for any $1 \leq j \leq n$, $singleDec_j(a) = a_j$ and $\chi_{isCodeFor} \in PR$ \\
	- Proof: all utilized functions and relations of prime numbers are in PR or closed under PR \\ <Misc. theorems>
	===================================================================================================================
\subsection{(Metatheorem) IsCode, empty, singleEnc, concatenate functions are primitive recursive} %=7.3.15
	- $isCode(a)$ iff there exists $\seq{\vdc{a_i}{i=1}{len(a)}}$, $isCodeFor(a, \seq{\vdc{a_i}{i=1}{len(a)}})$ and $\chi_{isCode} \in PR$ \\
	- $len(empty()) = 0$ and $empty \in PR$ \\
	- If $len(a) = 1$, then there $singleEnc(a) = p(1) E (singleDec_1(a) + 1)$ and $singleEnc \in PR$ \\
	- If $isCodeFor(a, \vdc{a_i}{i=1}{n})$ and $isCodeFor(b, \vdc{b_j}{j=1}{m})$, then $isCodeFor(cat(a, b), \vdc{a_i}{i=1}{n}, \vdc{b_j}{j=1}{m})$ and $cat \in PR$ \\
	- Proof: all utilized functions and relations of prime numbers are in PR or closed under PR \\ <Misc. theorems>
	===================================================================================================================
\subsection{(Metatheorem) Enc, dec are primitive recursive} %=7.3.15
	- $enc(\vdc{x_i}{i=1}{n}) = cat(\vdc{singleEnc(x_i)}{i=1}{n})$ and $enc \in PR$ \\
	- $dec(x) = \seq{\vdc{singleDec_i(x)}{i=1}{n}}$ and $dec \in PR$ \\
	- Alternative definitions could be formed using bounded products and the prime number function \\
	- Proof: all utilized functions and relations of prime numbers are in PR or closed under PR \\ <Misc. theorems>
	===================================================================================================================
\subsection{(Metatheorem) Coding is monotonic} %=7.3.16
	- For any $\seq{\vdc{a_i}{i=1}{n}}$, for any $1 \leq m \leq n$, $enc(\vdc{a_i}{i=1}{n}) < enc(\vdc{a_h}{h=1}{m-1}, a_m+1, \vdc{a_t}{t=m+1}{n})$ \\
	- For any $\seq{\vdc{a_i}{i=1}{n+1}}$, $enc(\vdc{a_i}{i=1}{n}) < enc(\vdc{a_i}{i=1}{n+1})$ \\
	- These monotonicity properties guarantee that: \\
		-- All the numbers in the sequence encoded by the number $x$ will be smaller than $x$ \\
		-- The code for a subsequence of a sequence will be smaller than the code for the sequence itself \\
		-- This makes it easy to find primitive recursive definitions of predicates and functions dealing with encoded sequences \\
	- Proof: TODO ABSTRACTED \\
	===================================================================================================================
\subsection{(Metatheorem) Subbed Godel numbering function is the primitive recursive} %=7.3.17
	- If $\phi \in Form(\mathcal{L}_{NT})$ and $\free{x}{\phi}$, then there exists $f_\phi(a) = GN(\sub{\phi}{x}{\overleftarrow{a}})$ and $f_\phi \in PR$ \\
	- Proof: \\
		-- If $t \is 0$,                  then $g_t(a) = enc(9)$ \\
		-- If $t \is v_i$,                then $g_t(a) = enc(2i)$ \\
		-- If $t \is S t_1$,              then $g_t(a) = enc(11, g_{t_1}(a))$ \\
		-- If $t \is + t_1 t_2$,          then $g_t(a) = enc(13, g_{t_1}(a), g_{t_2}(a))$ \\
		-- If $t \is \centerdot t_1 t_2$, then $g_t(a) = enc(15, g_{t_1}(a), g_{t_2}(a))$ \\
		-- If $t \is E t_1 t_2$,          then $g_t(a) = enc(17, g_{t_1}(a), g_{t_2}(a))$ \\
		-- If $\phi \is \equiv t_1 t_2$,     then $f_\phi(a) = enc(7, g_{t_1}(a), g_{t_2}(a))$ \\
		-- If $\phi \is < t_1 t_2$,          then $f_\phi(a) = enc(19, g_{t_1}(a), g_{t_2}(a))$ \\
		-- If $\phi \is \lnot \alpha$,       then $f_\phi(a) = enc(1, f_\alpha(a))$ \\
		-- If $\phi \is \alpha \lor \beta$,  then $f_\phi(a) = enc(3, f_\alpha(a), f_\beta(a))$ \\
		-- If $\phi \is \forall v_i \alpha$, then $f_\phi(a) = enc(5, g_{v_i}(a), f_\alpha(a))$ \\
		-- $f_\phi = GN(\sub{\phi}{x}{\overleftarrow{\placeholder}})$ \\ <Induction>
		-- $f_\phi \in PR$ \\ <Misc. theorems>
	===================================================================================================================
\subsection{(Definition) Ackermann function} %=7.3.1.9
	- The Ackermann function $A: \mathbb{N}^2 \rightarrow \mathbb{N}$ is defined as: \\
		-- $A(0, y) = y+1$ \\
		-- $A(x+1, 0) = A(x, 1)$ \\
		-- $A(x+1, y+1) = A(x, A(x+1, y))$ \\
	===================================================================================================================
\subsection{(Definition) Majorization} %=7.3.1.9
	- The function $h: \mathbb{N}^n \rightarrow \mathbb{N}$ is majorized by the function $f: \mathbb{N}^2 \rightarrow \mathbb{N}$ ($Majorized(h, f)$) iff there exists $b$, for any $\set{\vdc{a_i}{i=1}{n}} \subseteq \mathbb{N}$, $h(\vdc{a_i}{i=1}{n}) < f(b, max(\vdc{a_i}{i=1}{n}))$ \\
	===================================================================================================================
\subsection{(Metatheorem) Binary functions cannot majorize themselves} %=7.3.1.9
	- $f: \mathbb{N}^2 \rightarrow \mathbb{N}$ and $\pnot{Majorized(f, f)}$ \\
	- Proof: \\
		-- If $Majorized(f, f)$, then \\
			--- There exists $b$, for any $x, y$, $f(x, y) < f(b, max(x, y))$ \\
			--- $f(b, max(x, y, b)) < f(b, max(x, y, b)) = f(b, max(x, y, b))$ \\
			--- CONTRADICTION \\
		-- $\pnot{Majorized(f, f)}$ \\ <Metaproof by contradiction>
	===================================================================================================================
\subsection{(Definition) Majorized by the Ackermann function} %=7.3.1.9
	- The set $\mathcal{A}$ is defined by $\mathcal{A} = \set{h: Majorized(h, A)}$ \\
	===================================================================================================================
\subsection{(Metatheorem) Primitive recursive functions are majorized by the Ackermann function} %=7.3.1.9
	- $PR \subseteq \mathcal{A}$ \\
	- Proof: TODO: ABSTRACTED \\ https://planetmath.org/ackermannfunctionisnotprimitiverecursive
		-- $a_{max} = max(\vdc{a_i}{i=1}{k})$ \\
		-- If $f = \mathcal{O}$, then \\
			--- $f(a) = 0 < a+1 = A(0, a_{max})$ \\
			--- $f \in \mathcal{A}$ \\
		-- If $f = S$, then \\
			--- $f(a) = a+1 < a+2 = A(1, a_{max})$ \\
			--- $f \in \mathcal{A}$ \\
		-- If $f = \mathcal{I}_j^m$, then \\
			--- $f(\vdc{a_i}{i=1}{Arity(f)}) = a_j \leq a_{max} < a_{max}+1 = A(0, a_{max})$ \\
			--- $f \in \mathcal{A}$ \\
		-- If $f(\vdc{a_i}{i=1}{Arity(f)}) = h(\vdc{g_j(\vdc{a_i}{i=1}{Arity(f)})}{j=1}{Arity(h)})$ and $\set{h, \vdc{g_j}{j=1}{Arity(h)}} \subseteq \mathcal{A}$, then \\
			--- For any $1 \leq j \leq Arity(h)$, there exists $r_{g_j}$, $g_j(\vdc{a_i}{i=1}{Arity(f)})	< A(r_{g_j}, a_{max})$ \\ <Inductive hypothesis>
			--- There exists $r_h$, $h(\vdc{a_i}{i=1}{Arity(h)}) < A(r_h, a_{max})$ \\ <Inductive hypothesis>
			--- $f(\vdc{a_i}{i=1}{Arity(f)}) = h(\vdc{g_j(\vdc{a_i}{i=1}{Arity(f)})}{j=1}{Arity(h)}) < A(r_h, {g_j}_{max})$ \\ <Inductive hypothesis>
			--- $A(r_h, {g_j}_{max}) < A(r_h, A(r_{g_j}, a_{max}))$ \\ <Monotonic property>
			--- $A(r_h, A(r_{g_j}, a_{max})) < A(b, a_{max})$ \\ <Branch is primitive recurisve property>
			--- $f \in \mathcal{A}$ \\
		-- If $f ... primitive recursion$, then $f \in \mathcal{A}$ \\
		-- $PR \subseteq \mathcal{A}$ \\ <Induction>
	===================================================================================================================
\subsection{(Metatheorem) Ackermann function is not primitive recursive} %=7.3.1.9
	- $A \inot \in PR$ \\
	- Proof: \\
		-- If $f \in PR$, then $f \in \mathcal{A}$ \\ <Primitive recursive functions are majorized by the Ackermann function>
		-- If $f \inot \in \mathcal{A}$, then $f \inot \in PR$ \\ <Contrapositive>
		-- $A \inot \in \mathcal{A}$ \\ <Binary functions cannot majorize themselves>
		-- $A \inot \in PR$ \\ <Conjunction>
	===================================================================================================================

\subsection{(Definition) Computable index} %=7.4.1
	- The natural number $e$ is a computable index for the function $f$ ($CI(e_f, f)$) iff: \\
		-- If $f = S$, then $e_f = enc(0)$ \\
		-- If $f = \mathcal{I}_i^n$, then $e_f = enc(1, n, i)$ \\
		-- If $f = \mathcal{O}$, then $e_f = enc(2)$ \\
		-- If $f(\vdc{x_i}{i=1}{n}) = h(\vdc{g_j(\vdc{x_i}{i=1}{n})}{j=1}{Arity(h)})$ and for any $1 \leq j \leq Arity(h)$, $CI(e_{g_j}, g_j)$ and $CI(e_h, h)$, then $e_f = enc(3, n, \vdc{e_{g_j}}{j=1}{Arity(h)}, e_h)$ \\
		-- If $f(\vdc{x_i}{i=1}{n}, 0) = g(\vdc{x_i}{i=1}{n})$ and $f(\vdc{x_i}{i=1}{n}, y+1) = h(\vdc{x_i}{i=1}{n}, y, f(\vdc{x_i}{i=1}{n}, y))$ and $CI(e_g, g)$ and $CI(e_h, h)$, then $e_f = enc(4, n, e_g, e_h)$ \\
		-- If $f(\vdc{x_i}{i=1}{n}) = \mu_{US}(g)(\vdc{x_i}{i=1}{n})$ and $CI(e_g, g)$, then $e_f = enc(5, n, e_g)$ \\
	- $e_f$ is like a computer program / source code for $f$ \\
		-- ALTERNATIVE TRASH
		-- $e_f = enc(0)$ and $f = S$ \\
		-- $e_f = enc(1, n, i)$ and $f = \mathcal{I}_i^n$ \\
		-- $e_f = enc(2)$ and $f = \mathcal{O}$ \\
		-- $e_f = enc(3, n, \vdc{e_{g_j}}{j=1}{Arity(h)}, e_h)$ and $f(\vdc{x_i}{i=1}{n}) = h(\vdc{g_j(\vdc{x_i}{i=1}{n})}{j=1}{Arity(h)})$ and for any $1 \leq j \leq Arity(h)$, $CI(e_{g_j}, g_j)$ and $CI(e_h, h)$ \\
		-- $e_f = enc(4, n, e_g, e_h)$ and $f(\vdc{x_i}{i=1}{n}, 0) = g(\vdc{x_i}{i=1}{n})$ and $f(\vdc{x_i}{i=1}{n}, y+1) = h(\vdc{x_i}{i=1}{n}, y, f(\vdc{x_i}{i=1}{n}, y))$ and $CI(e_g, g)$ and $CI(e_h, h)$ \\
		-- $e_f = enc(5, n, e_g)$ and $f(\vdc{x_i}{i=1}{n}) = \mu_{US}(g)(\vdc{x_i}{i=1}{n})$ and $CI(e_g, g)$ \\
	===================================================================================================================
\subsection{(Metatheorem) Padding lemma} %=7.4.2
	- If $f \in \mu$, then there exists $E$, $InfiniteSet(E)$ and for any $e \in E$, $CI(e, f)$ \\
	- Proof: \\
		-- By definition of computable index, $CI(e_f, f)$ \\
		-- Let $I_1^1(f(x)) = f(x)$, so $CI(e_{I_1^1(f)}, f)$, and so on ... \\
		-- TODO ABSTRACTED \\
	===================================================================================================================
\subsection{(Definition) Computations} %=7.5.1
	- The collection of computations $\mathcal{C}$ is defined by: \\
		-- If $C = \seq{}$, then $C \in \mathcal{C}$ \\
		-- If $C = \Gamma \cup \seq{enc(e_S, a, b)}$ and $\Gamma \in \mathcal{C}$ and \\
			--- $CI(e_S, S)$ and $b = S(a)$, then \\
			--- $C \in \mathcal{C}$ \\
		-- If $C = \Gamma \cup \seq{enc(e_{\mathcal{I}_i^n}, enc(\vdc{a_i}{i=1}{n}), b)}$ and $\Gamma \in \mathcal{C}$ and \\
			--- $CI(e_{\mathcal{I}_i^n}, \mathcal{I}_i^n)$ and \\
			--- $1 \leq i \leq n$ and $b = a_i$, then \\
			--- $C \in \mathcal{C}$ \\
		-- If $C = \Gamma \cup \seq{enc(e_\mathcal{O}, enc(), 0)}$ and $\Gamma \in \mathcal{C}$ and \\
			--- $CI(e_\mathcal{O}, \mathcal{O})$, then \\
			--- $C \in \mathcal{C}$ \\
		-- If $C = \Gamma \cup \seq{enc(e_f, enc(\vdc{a_i}{i=1}{n}), b)}$ and $\Gamma \in \mathcal{C}$ and \\
			--- $f(\vdc{a_i}{i=1}{n}) = h(\vdc{g_j(\vdc{a_i}{i=1}{n})}{j=1}{Arity(h)})$ and $CI(e_f, f)$ and \\
			--- There exists $\set{\vdc{v_j}{j=1}{Arity(h)}} \subseteq \mathbb{N}$, (
				---- For any $1 \leq l \leq Arity(h)$, $enc(e_{g_l}, enc(\vdc{a_i}{i=1}{n}), v_l) \in \Gamma$ and \\
				---- $enc(e_h, \vdc{v_j}{j=1}{Arity(h)}, b) \in \Gamma$), then \\
			--- $C \in \mathcal{C}$ \\
		-- If $C = \Gamma \cup \seq{enc(e_f, enc(\vdc{a_i}{i=1}{n}, c), b)}$ and $\Gamma \in \mathcal{C}$ and \\
			--- $f(\vdc{a_i}{i=1}{n}, 0) = g(\vdc{a_i}{i=1}{n})$ and $f(\vdc{a_i}{i=1}{n}, y+1) = h(\vdc{a_i}{i=1}{n}, y, f(\vdc{a_i}{i=1}{n}, y))$ and $CI(e_f, f)$ and \\
			--- There exists $\set{\vdc{v_j}{j=0}{c}} \subseteq \mathbb{N}$, ( \\
				---- $enc(e_g, enc(\vdc{a_i}{i=1}{n}), v_0) \in \Gamma$ and \\
				---- For any $1 \leq l \leq c$, $enc(e_h, enc(\vdc{a_i}{i=1}{n}, l, v_{l-1}), v_l)$), then \\
			--- $C \in \mathcal{C}$ \\
		-- If $C = \Gamma \cup \seq{e_f, enc(\vdc{a_i}{i=1}{n}), b}$ and $\Gamma \in \mathcal{C}$ and \\
			--- $f(\vdc{a_i}{i=1}{n}) = \mu_{US}(g)(\vdc{a_i}{i=1}{n})$ and $CI(e_f, f)$ and \\
			--- There exists $\set{\vdc{v_j}{j=0}{b}} \subseteq \mathbb{N}$, ( \\
				---- For any $0 \leq l \leq b$, $enc(e_g, enc(\vdc{a_i}{i=1}{n}, l), v_l) \in \Gamma$ and \\
				---- If $i < b$, then $v_i \inot = 0$ and $v_b = 0$), then \\
			--- $C \in \mathcal{C}$ \\
	===================================================================================================================
\subsection{(Metatheorem) Computation iff computable indexable} %=7.5.2
	- If $CI(e_f, f)$, then $f(\vdc{a_i}{i=1}{n}) = b$ iff there exists $\Gamma \in \mathcal{C}$, $enc(e_f, enc(\vdc{a_i}{i=1}{n}), b) \in \Gamma$ \\
	- Proof: \\
		-- If $CI(e_f, f)$, then \\
			--- If $e_f = enc(0)$, then \\	
				---- $f = S$ \\ <Definition>
				---- If $f(a) = b$, then \\
					----- $f(a) = S(a) = a+1 = b$ \\
					----- There exists $\Omega \in \mathcal{C}$ \\
					----- $\Gamma = \Omega \cup \seq{e_f, enc(a), b} \in \mathcal{C}$ \\ <Definition>
					----- There exists $\Gamma \in \mathcal{C}$, $enc(e_f, enc(\vdc{a_i}{i=1}{n}), b) \in \Gamma$ \\
				---- If there exists $\Gamma \in \mathcal{C}$, $enc(e_f, enc(\vdc{a_i}{i=1}{n}), b) \in \Gamma$, then \\
					----- $b = a+1$ \\ <Definition>
					----- $b = a+1 = S(a) = f(a)$ \\
				---- $f(a) = b$ iff there exists $\Gamma \in \mathcal{C}$, $enc(e_f, enc(\vdc{a_i}{i=1}{n}), b) \in \Gamma$ \\ <Conjunction>
			--- If $e_f = enc(1, n, i)$, then \\
				---- $f = \mathcal{I}_i^n$ \\ <Definition>
				---- If $f(\vdc{a_i}{i=1}{n}) = b$, then \\
					---- $f(\vdc{a_i}{i=1}{n}) = \mathcal{I}_i^n(\vdc{a_i}{i=1}{n}) = a_i = b$ \\
					----- There exists $\Omega \in \mathcal{C}$ \\
					----- $\Gamma = \Omega \cup \seq{e_f, enc(\vdc{a_i}{i=1}{n}), b} \in \mathcal{C}$ \\ <Definition>
					----- There exists $\Gamma \in \mathcal{C}$, $enc(e_f, enc(\vdc{a_i}{i=1}{n}), b) \in \Gamma$ \\
				---- If there exists $\Gamma \in \mathcal{C}$, $enc(e_f, enc(\vdc{a_i}{i=1}{n}), b) \in \Gamma$, then \\
					----- $b = a_i$ \\ <Definition>
					----- $b = a_i =  \mathcal{I}_i^n(\vdc{a_i}{i=1}{n}) = f(\vdc{a_i}{i=1}{n})$ \\
				---- $f(a) = b$ iff there exists $\Gamma \in \mathcal{C}$, $enc(e_f, enc(\vdc{a_i}{i=1}{n}), b) \in \Gamma$ \\ <Conjunction>
			--- TODO ABSTRACTED \\
	===================================================================================================================
\subsection{(Metatheorem) Computation iff computable indexable corollary} %=7.5.3
	- If $CI(e_f, f)$, then $f(\vdc{a_i}{i=1}{n})	= b$ iff there exists $\Gamma \in \mathcal{C}$, $\Gamma = \Omega \cup \seq{enc(e_f, enc(\vdc{a_i}{i=1}{n}), b)}$ \\
	- Proof: \\ TODO: ABSTRACTED
	===================================================================================================================
\subsection{(Notation) Indexed abbreviations} %=7.5.3
	- $dec_{a, b}(t) = singleDec_b(singleDec_a(t))$ \\
	===================================================================================================================
\subsection{(Metatheorem) IsComputation is primitive recursive} %=7.5.4
	- The predicate $isComputation$ is defined as $isComputation(t)$ iff $t = enc(\vdc{c_i}{i=1}{k})$ and $k \geq 1$ and $\seq{\vdc{c_i}{i=1}{k}} \in \mathcal{C}$ \\
	- $\chi_{isComputation} \in PR$ \\
	- Proof: \\ TODO: ABSTRACTED
	===================================================================================================================
\subsection{(Metatheorem) T-predicate is primitive recursive} %=7.5.5
	- The predicate $\mathcal{T}_n$ is defined as $\mathcal{T}_n(e, \vdc{x_i}{i=1}{n}, t)$ iff $isComputation(t)$ and $dec_{len(t), 1}(t) = e$ and $len(dec_{len(t), 2}(t)) = n$ and $\seq{\vdc{dec_{len(t), 2, i}(t)}{i=1}{n}} = \seq{\vdc{x_i}{i=1}{n}}$ \\
	- $\mathcal{T}_n(e_f, \vdc{x_i}{i=1}{n}, t)$ states that the number $t$ encodes an execution of the program given by the index $e$ on the inputs $\vdc{x_i}{i=1}{n}$ \\
	- $\chi_{\mathcal{T}_n} \in PR$ \\
	- Proof: \\ TODO: ABSTRACTED
	===================================================================================================================
\subsection{(Metatheorem) U is primitive recursive} %=7.5.5
	- The function $\mathcal{U}$ is defined as $\mathcal{U}(t) = dec_{len(t), 3}(t)$ \\
	- $\mathcal{U}(t)$ picks the output from the computation encoded by $t$ \\
	- $\mathcal{U} \in PR$ \\
	- Proof: \\ TODO: ABSTRACTED
	===================================================================================================================
\subsection{(Metatheorem) Kleene's Normal Form theorem***} %=7.4.3/7.5.5
	- For any $f \in \mu$, if $CI(e_f, f)$, then $f(\vdc{x_i}{i=1}{n}) = \mathcal{U}(\mu_{US}(\chi_{\mathcal{T}_n})(e_f, \vdc{x_i}{i=1}{n}))$ \\
	- Proof: \\ TODO: ABSTRACTED
		-- $f(\vdc{x_i}{i=1}{n}) = b$ is defined iff \\
			--- There exists $\Gamma \in \mathcal{C}$, $\Gamma = \Omega \cup \seq{enc(e_f, enc(\vdc{a_i}{i=1}{n}), b)}$ iff \\ <Computation iff computable indexable corollary>
			---- $\mu_{US}(\chi_{\mathcal{T}_n})(e_f, \vdc{x_i}{i=1}{n})$ is defined iff \\
			---- $\mathcal{U}(\mu_{US}(\chi_{\mathcal{T}_n})(e_f, \vdc{x_i}{i=1}{n})) = b$ \\ <Definition>
	===================================================================================================================
\subsection{(Definition) Computable function by index} %=7.4.4
	- The $e$-th $N$-ary computable function $\{e\}^n$ is defined as $\{e\}^n(\vdc{x_i}{i=1}{n}) = \mathcal{U}(\mu_{US}(\chi_{\mathcal{T}_n})(e, \vdc{x_i}{i=1}{n}))$ \\
	- If $CI(e, f)$, then $\{e\}^n = f$ \\
	- Otherwise, $\{e\}^n$ is undefined everywhere \\
	===================================================================================================================
\subsection{(Metatheorem) Enumeration theorem***} %=7.4.5
	- For any $f \in \mu$, there exists $e$, $f(\vdc{x_i}{i=1}{n}) = \{e\}^n((\vdc{x_i}{i=1}{n}))$ \\
	- Proof: \\
		-- For any $f \in \mu$, \\
			--- There exists $e$, $CI(e, f)$ \\ <Padding lemma>
			--- $f(\vdc{x_i}{i=1}{n}) = \mathcal{U}(\mu_{US}(\chi_{\mathcal{T}_n})(e, \vdc{x_i}{i=1}{n}))$ \\ <Kleene's Normal Form theorem>
			--- $f(\vdc{x_i}{i=1}{n}) = \{e\}^n((\vdc{x_i}{i=1}{n}))$ \\ <Definition>
	- The function $g$ is defined as $g(y, \vdc{x_i}{i=1}{n}) = \mathcal{U}(\mu_{US}(\chi_{\mathcal{T}_n})(y, \vdc{x_i}{i=1}{n}))$ \\
	- $g$ outputs the computable function indexed by $y$ \\
	- $g \in \mu$	and for any $y \in \mathbb{N}$, $g(y, \vdc{x_i}{i=1}{n}) = \{y\}^n(\vdc{x_i}{i=1}{n})$ \\
	- Proof: \\ TODO: ABSTRACTED
		-- $\mathcal{U} \in \mu$ \\ <U is primitive recursive>
		-- $\chi_{\mathcal{T}_n} \in \mu$ \\ <T-predicate is primitive recursive>
		-- $g \in \mu$ \\ <Misc. theorems>
		-- For any $y \in \mathbb{N}$, \\
			--- $\{y\}^n(\vdc{x_i}{i=1}{n}) = \mathcal{U}(\mu_{US}(\chi_{\mathcal{T}_n})(y, \vdc{x_i}{i=1}{n}))$ \\ <Definition>
			--- $g(y, \vdc{x_i}{i=1}{n}) = \{y\}^n(\vdc{x_i}{i=1}{n})$ \\
	===================================================================================================================
\subsection{(Metatheorem) Universal function theorem} %=7.4.6
	- The computable function $u$ is defined as $u(y, enc(\vdc{x_i}{i=1}{n})) = \{y\}^1(enc(\vdc{x_i}{i=1}{n}))$ \\
	- $u$ is the universal function \\
	- For any $f \in \mu$, there exists $t \in \mathbb{N}$, $u(t, enc(\vdc{x_i}{i=1}{n})) = f(\vdc{x_i}{i=1}{n})$ \\
	- Proof: \\ TODO: ABSTRACTED
		-- There exists $f_0 \in \mu$, $f_0(enc(\vdc{x_i}{i=1}{n})) = f(\vdc{x_i}{i=1}{n})$ \\ <Misc. theorems>
		-- There exists $y$, $CI(y, f_0)$ \\
		-- $u(y, enc(\vdc{x_i}{i=1}{n})) = $ \\
			--- $\{y\}^1(enc(\vdc{x_i}{i=1}{n})) = $ \\ <Definition>
			--- $\mathcal{U}(\mu_{US}(\chi_{\mathcal{T}_n})(y, enc(\vdc{x_i}{i=1}{n}))) = $ \\ <Definition>
			--- $f_0(enc(\vdc{x_i}{i=1}{n})) = $ \\ <Kleene's Normal Form theorem>
			--- $f(\vdc{x_i}{i=1}{n})$ \\
	===================================================================================================================
\subsection{(Metatheorem) Diagonal functions are non-computable} %=7.4.6
	- For simplicity, consider functions that are only $1$-ary \\
	- The diagonal function $d$ is defined as $d(i) \inot = \{i\}^1(i)$ \\
	- $d \inot \in \mu$ \\
	- Proof: TODO ABSTRACTED \\
		-- For any $f \in \mu$, \\
			--- There exists $e_f$, $CI(e_f, f)$ \\
			--- $d(e_f) \inot = \{e_f\}^1(e_f)$ \\ <Definition>
			--- $\{e_f\}^1(e_f) = $ \\
				---- $\mathcal{U}(\mu_{US}(\chi_{\mathcal{T}_n})(e_f, e_f)) = $ \\ <Definition>
				---- $f(e_f)$ \\ <Kleene's Normal Form theorem>
			--- $d(e_f) \inot = f(e_f)$ \\ <Conjunction>
			--- $d \inot = f$ \\
		-- For any $f \in \mu$, $d \inot = f$ \\ <Abbreviate>
		-- $d \inot \in \mu$ \\
	===================================================================================================================
\subsection{(Metatheorem) Total diagonal functions are non-computable} %=7.4.7
	- One simple total example of $d^*$ can be defined as: \\
		-- If $\{x\}^1(x)$ is defined, then $d^*(x) = \{x\}^1(x) + 1$ \\
		-- Otherwise, $d^*(x) = 0$ \\
	- $Total(d^*)$ and $d^*$ satisfies the properties of the diagonal function, thus $d^* \inot \in \mu$ \\
	===================================================================================================================
\subsection{(Metatheorem) Undecidability of the Halting Problem} %=7.4.8
	- The halting predicate $H$ is defined as $H(y, x)$ iff $u(y, x)$ is defined \\
	- $\chi_H \inot \in \mu$ \\
	- Proof: \\
		-- If $\chi_H \in \mu$, then \\
			--- If $\chi_H(x, x) = 0$, then $d'(x) = \{x\}^1(x) + 1$ and otherwise, $d'(x) = 0$ \\
			--- $d' \in \mu$ \\ <Definition by cases are closed under primitive recursion>
			--- $d' \inot \in \mu$ \\ <Total diagonal functions are non-computable>
			--- CONTRADICTION !!
		-- $\chi_H \inot \in \mu$ \\ <Metaproof by contradiction>
	===================================================================================================================
\subsection{(Metatheorem) S-m-n theorem} %=7.4.9
	- There exists $S^m_n \in PR$, $\{S^m_n(e, \vdc{x_i}{i=1}{n})\}^m(\vdc{y_j}{j=1}{m}) = \{e\}^{n+m}(\vdc{x_i}{i=1}{n}, \vdc{y_j}{j=1}{m})$ \\
	- TODO Something about the combination of two functions \\
	- Proof: TODO ABSTRACTED \\
	===================================================================================================================

\section{(Notation) Computability notations} %=7.6
	- $\mathcal{T}(e, x, t)$ abbreviates $\mathcal{T}_1(e, x, t)$ \\
	- $\{e\}(x)$ abbreviates $\{e\}^1(x)$ \\
	- $\{e\}(x) = \{e\}^1(x) = \mathcal{U}(\mu_{US}(\chi_{\mathcal{T}_1})(e, x)) = \mathcal{U}(\mu_{US}(\chi_{\mathcal{T}})(e, x))$ \\
	- If $f \in \mu$ and $f: \mathbb{N} \rightarrow \mathbb{N}$, then $dom(f) = \set{x \in \mathbb{N}: \txtexists{y \in \mathbb{N}}{f(x) = y}}$ \\
	- If $f \in \mu$ and $f: \mathbb{N} \rightarrow \mathbb{N}$, then $rng(f) = \set{y \in \mathbb{N}: \txtexists{x \in \mathbb{N}}{f(x) = y}}$ \\
	===================================================================================================================
\subsection{(Definition) Semi-computable set} %=7.6.1
	- The set $A$ is semi-computable ($A \in SC$) iff there exists $f \in \mu$, $A = dom(f)$ \\
	- There exists an algorithm that confirms membership, but not necessarily decide membership \\
	===================================================================================================================
\subsection{(Definition) Computably enumerable set} %=7.6.2
	- The set $A$ is computable enumerable ($A \in CE$) iff there exists $f \in \mu$, $Total(f)$ and $A = rng(f)$ \\
	- Alternative definition: $A \in CE$ iff $Finite(A)$ or there exists $f \in \mu$, $Bijection(f)$ and $Total(f)$ and $A = rng(f)$ \\
	- There exists an algorithm that can list down all the elements of the set \\
	===================================================================================================================
\subsection{(Metatheorem) Equivalent definition for domain} %=7.6.3
	- If $f \in \mu$ and $CI(e, f)$, then $dom(f) = \set{x: \txtexists{t}{\mathcal{T}(e, x, t)}}$ \\
	- Proof: \\ <Kleene's Normal Form theorem>
		-- If $x \in dom(f)$, then \\
			--- There exists $y \in \mathbb{N}$, $f(x) = y$ \\ <Definition>
			--- There exists $\Gamma \in \mathcal{C}$, $\Gamma = \Omega \cup \seq{enc(e, enc(x), y)}$ \\ <Computation iff computable indexable corollary>
			--- $t = enc(\vdc{\omega_i}{i=1}{|\Omega|},	enc(e, enc(x), y))$ \\ <IsComputation is primitive recursive>
			--- There exists $t$, $\mathcal{T}(e, enc(x), t)$ \\ <T-predicate is primitive recursive>
		-- $dom(f) \subseteq \set{x: \txtexists{t}{\mathcal{T}(e, x, t)}}$ \\
		-- If $x \in \set{x: \txtexists{t}{\mathcal{T}(e, x, t)}}$, then \\
			--- $f(\vdc{x_i}{i=1}{n}) = \mathcal{U}(t) = y$ \\ <Kleene's Normal Form theorem>
			--- There exists $y \in \mathbb{N}$, $f(x) = y$ \\
			--- $x \in dom(f)$ \\ <Definition>
		-- $\set{x: \txtexists{t}{\mathcal{T}(e, x, t)}} \subseteq dom(f)$ \\
		-- $dom(f) = \set{x: \txtexists{t}{\mathcal{T}(e, x, t)}}$ \\
	===================================================================================================================
\subsection{(Metatheorem) Computable sets are semi-computable} %=7.6.4
	- If $\chi_A \in \mu$, then $A \in SC$ \\
	- Proof: \\
		-- If $\chi_A \in \mu$, then \\
			--- $f(x) = \mu_{US}(\chi_{x \in A \land (\placeholder = \placeholder)})(x)$ \\
			--- $f \in \mu$ \\ <Misc. theorems>
			--- $dom(f) = A$ \\
			--- There exists $f \in \mu$, $A = dom(f)$ \\
			--- $A \in SC$ \\
	===================================================================================================================
\subsection{(Metatheorem) SC iff CE property} %=7.6.5
	- If $A \subseteq \mathbb{N}$, then \\
		-- $A \in SC$ iff \\
		-- $A = \set{}$ or there exists $f \in PR$, $rng(f) = A$ iff \\
		-- $A \in CE$ \\
	- Proof: If $A \in SC$, then $A = \set{}$ or there exists $f \in PR$, $rng(f) = A$ \\
		-- If $A \in SC$ and $A = \set{}$, then $A = \set{}$ or there exists $f \in PR$, $rng(f) = A$ \\
		-- If $A \in SC$, and $A \inot = \set{}$, then \\
			--- There exists $f$, $A = dom(f)$ \\ <Definition>
			--- There exists $e$, $CI(e, f)$ \\
			--- $A = \set{x: \txtexists{t}{\mathcal{T}(e, x, t)}}$ <(I)> \\ <Equivalent definition for domain>
			--- There exists $a$, $a \in A$ \\
			--- There exists $g_a$, if $\chi_{\mathcal{T}(e, singleDec_1(x), singleDec_2(x))} = 0$, then $g_a(x) = singleDec_1(x)$ and otherwise, $g_a(x) = a$ <(II)> \\
			--- $g_a \in PR$ \\ <Misc. theorems>
			--- If $b \in A$, then \\
				---- There exists $t$, $\mathcal{T}(e, b, t)$ \\ <(I)>
				---- $g_a(enc(b, t)) = b$ \\ <(II)>
				---- $b \in rng(g_a)$ \\ <Definition>
			--- $A \subseteq rng(g_a)$ \\
			--- If $b \in rng(g_a)$, then \\
				---- If $b = a$, then $b \in A$ \\
				---- If $b \inot = a$, then \\
					----- There exists $t$, $\mathcal{T}(e, b, t)$ \\ <(II)>
					----- $b \in A$ \\ <(I)>
				---- $b \in A$ \\ <Conjunction>
			--- $rng(g_a) \subseteq A$ \\
			--- $rng(g_a) = A$ \\ <Conjunction>
			--- There exists $f \in PR$, $rng(f) = A$ or $A = \set{}$ \\
		-- If $A \in SC$, then $A = \set{}$ or there exists $f \in PR$, $rng(f) = A$ \\ <Conjunction>
	- Proof: If $A = \set{}$ or there exists $f \in PR$, $rng(f) = A$, then $A \in CE$ \ \\
		-- If $A = \set{}$ or there exists $f \in PR$, $rng(f) = A$, then \\
			--- If $Finite(A)$, then $A \in CE$ \\ <Definition>
			--- If $\pnot{Finite(A)}$, then \\
				---- There exists $NextHasOccurred_f$, if $\chi_{(\forall j \leq x)(f(j) \inot = f(x+1))}(x) = 0$, then $NextHasOccurred_f = 1$ and otherwise, $NextHasOccurred_f = 0$ \\
				---- $NextHasOccurred_f \in PR$ \\ <Misc. theorems>
				---- $NumOfUniqueOutputsLeq(n) = Sum(NextHasOccurred_f)(n)+1$ \\
				---- $NumOfUniqueOutputsLeq \in PR$ \\ <Misc. theorems>
				---- There exists $g$, if $\mathcal{I}_1^1(x) = 0$, then $g(x) = f(0)$ and otherwise, $g(x) = f(\mu_{US}(\chi_{NumOfUniqueOutputsLeq(\placeholder)-1 = x}))$	<(I)> \\
				---- $g \in \mu$ \\ <Misc. theorems>
				---- $Total(g)$ and $Bijection(g)$ and $rng(g) = A$ <(I)> \\
				---- There exists $g \in \mu$, $Bijection(f)$ and $Total(f)$ and $A = rng(f)$ \\ 
				---- $A \in CE$ \\
			--- $A \in CE$ \\ <Conjunction>
		-- If $A = \set{}$ or there exists $f \in PR$, $rng(f) = A$, then $A \in CE$ \\ <Conjunction>
	- Proof: If $A \in CE$, then $A \in SC$ \\
		-- If $A \in CE$, then \\
			--- $Finite(A)$ or there exists $f \in \mu$, $Bijection(f)$ and $Total(f)$ and $A = rng(f)$ \\
			--- If $Finite(A)$, then \\
				---- There exists $g$, $f(x) = \mu_{US}(\chi_{(\lor \vdc{a_i = \placeholder}{i=1}{|A|}) \land (x = \placeholder)})(x)$ \\
				---- $f \in \mu$ \\ <Misc. theorems>
				---- $dom(g) = A$ \\
				---- There exists $g \in \mu$, $A = dom(g)$ \\
				---- $A \in SC$ \\
			--- If $Finite(A)$, then $A \in SC$ \\ <Abbreviate>
			--- If $\pnot{Finite(A)}$ and there exists $f \in \mu$, $Bijection(f)$ and $Total(f)$ and $A = rng(f)$, then \\
				---- There exists $g$, $g(x) = \mu_{US}(\chi_{f(\placeholder) = x})(x)$ \\
				---- $g \in \mu$ \\ <Misc. theorems>
				---- $dom(g) = A$ \\
				---- There exists $g \in \mu$, $dom(g) = A$ \\
				---- $A \in SC$ \\
		 --- If $\pnot{Finite(A)}$ and there exists $f \in \mu$, $Bijection(f)$ and $Total(f)$ and $A = rng(f)$, then $A \in SC$ \\ <Abbreviate>
			--- $A \in SC$ \\ <Conjunction>
	- Proof: $A \in SC$ iff $A = \set{}$ or there exists $f \in PR$, $rng(f) = A$ iff $A \in CE$ \\ <Conjunction>
	===================================================================================================================
\subsection{(Definition) N-complement} %=7.6.5
	- The set $\bar{A}$ is the N-complement of the $A$ iff $\bar{A} = \mathbb{N} \setminus A$ \\
	===================================================================================================================
\subsection{(Metatheorem) Computable iff CE property} %=7.6.6
	- $\chi_A \in \mu$ iff $A \in CE$ and $\bar{A} \in CE$ \\
	- Proof: \\
		-- If $\chi_A \in \mu$, then \\
			--- $\chi_{\bar{A}} = Conj(\chi_A)$ \\
			--- $\chi_{\bar{A}} \in \mu$ \\ <Misc. theorems>
			--- $A \in SC$ and $\bar{A} \in SC$ \\ <Computable sets are semi-computable>
			--- $A \in CE$ and $\bar{A} \in CE$ \\ <SC iff CE property>
		-- If $\chi_A \in \mu$, then $A \in CE$ and $\bar{A} \in CE$ \\ <Abbreviate>
		-- If $A \in CE$ and $\bar{A} \in CE$, then \\
			--- If $A = \set{}$, then \\
				---- $\chi_{A}(x) = c_1^1(x) = 1$ \\
				---- $\chi_{A} \in \mu$ \\ <Misc. theorems>
			--- If $\bar{A} = \set{}$, then \\
				---- $\chi_{A}(x) = c_0^1(x) = 0$ \\
				---- $\chi_{A} \in \mu$ \\ <Misc. theorems>
			--- If $A \inot = \set{}$, then \\
				---- There exists $f_0 \in PR$, $Total(f_0)$ and $rng(f_0) = A$ iff \\ <SC iff CE property>
				---- There exists $f_1 \in PR$, $Total(f_1)$ and $rng(f_1) = \bar{A}$ iff \\ <SC iff CE property>
				---- There exists $inFind$, $inFind(x) = \mu_{US}(\chi_{(f_0(\placeholder) = x) \lor (f_1(\placeholder) = x)})(x)$ \\
				---- $inFind \in \mu$ \\ <Misc. theorems>
				---- $Total(f_0)$ and $Total(f_1)$ and $rng(f) = rng(f_0) \cup rng(f_1) = \mathbb{N}$ <(I)> \\ <Disjunction>
				---- $Total(inFind)$ \\ <(I)>
				---- There exists $\chi$, if $f_0(inFind(x)) = x$, then $\chi(x) = 0$, and otherwise $\chi(x) = 1$ \\
				---- $\chi \in \mu$ \\ <Misc. theorems>
				---- $\chi_{A}(x) = \chi(x) = 0$ iff $x \in A$ \\
				---- $\chi_{A} \in \mu$ \\
			--- $\chi_{A} \in \mu$ \\ <Conjunction>
		-- If $A \in CE$ and $\bar{A} \in CE$, then $\chi_A \in \mu$ \\ <Abbreviate>
		-- $\chi_A \in \mu$ iff $A \in CE$ and $\bar{A} \in CE$ \\ <Conjunction>
	- The case $A = \set{}$ is required because a function can't be total if $rng(f) = \set{}$ \\
	===================================================================================================================
\subsection{(Metatheorem) Computable iff SC property} %=7.6.7
	- $\chi_A \in \mu$ iff $A \in SC$ and $\bar{A} \in SC$ \\
	- Proof: \\
		-- $A \in SC$ and $\bar{A} \in SC$ iff $A \in CE$ and $\bar{A} \in CE$ \\ <SC iff CE property>
		-- $A \in CE$ and $\bar{A} \in CE$ iff $\chi_A \in \mu$ \\ <Computable iff CE property>
		-- $\chi_A \in \mu$ iff $A \in SC$ and $\bar{A} \in SC$ \\
	===================================================================================================================
\subsection{(Definition) Semi-computable set by index} %=7.6.8
	- The $e$-th semi-computable set $\mathcal{W}_e$ is defined as $\mathcal{W}_e = dom(\{e\})$ \\
	===================================================================================================================
\subsection{(Metatheorem) SC iff SC indexed} %=7.6.9
	- $A \in SC$ iff there exists $e$, $A = \mathcal{W}_e$ \\
	- Proof: \\
		-- $A \in SC$ iff \\
			--- There exists $f \in \mu$, $A = dom(f)$ iff \\ <Definition>
			--- There exists $e$, $A = dom(\{e\})$ iff \\ <Enumeration theorem>
			--- There exists $e$, $A = \mathcal{W}_e$ iff \\ <Definition>
	===================================================================================================================
\subsection{(Definition) K} %=7.6.10
	- The set $\mathcal{K}$ is defined as $\mathcal{K} = \set{x: x \in \mathcal{W}_x}$ \\
	- $\mathcal{K}$ stands for kool \\
	===================================================================================================================
\subsection{(Metatheorem) N-complement of K is not semi-computable} %=7.6.11
	- $\bar{\mathcal{K}} \inot \in SC$ \\
	- Proof: \\
		-- If $\bar{\mathcal{K}} \in SC$, then \\
			--- There exists $m$, $\bar{\mathcal{K}} = \mathcal{W}_m$ <(I)> \\ <SC iff SC indexed>
			--- $m \in \bar{\mathcal{K}}$ iff $m \in \mathcal{W}_m$ \\ <(I)>
			--- $m \in \mathcal{W}_m$ iff $m \in \mathcal{K}$ \\ <Definition>
			--- $m \in \mathcal{K}$ iff $m \inot \in \bar{\mathcal{K}}$ \\ <Definition>
			--- $m \in \bar{\mathcal{K}}$ iff $m \inot \in \bar{\mathcal{K}}$ \\ <Conjunction>
			--- CONTRADICTION !!
		-- $\bar{\mathcal{K}} \in SC$ \\ <Metaproof by contradiction>
	===================================================================================================================
\subsection{(Metatheorem) K is semi-computable} %=7.6.12
	- $\mathcal{K} \in SC$ \\
	- Proof: \\
		-- $x \in \mathcal{K}$ iff \\
			--- $x \in \mathcal{W}_x$ iff \\ <Definition>
			--- $x \in dom(\{x\})$ iff \\ <Definition>
			--- There exists $t$, $\mathcal{T}(x, x, t)$ \\ <Equivalent definition for domain>
		-- $x \in \mathcal{K}$ iff there exists $t$, $\mathcal{T}(x, x, t)$ <(I)> \\ <Abbreviate>
		-- There exists $f$, $f(x) = \mu_{US}(\chi_{\mathcal{T}(x, x, \placeholder)})(x, x)$ \\
		-- $f \in \mu$ \\ <Misc. theorems>
		-- $dom(f) = \mathcal{K}$ \\ <(I)>
		-- $\mathcal{K} \in SC$ \\
	===================================================================================================================
\subsection{(Metatheorem) K is not computable} %=7.6.12
	- $\chi_{\mathcal{K}} \inot \in \mu$ \\
	- Proof: \\
		-- If $\chi_{\mathcal{K}} \in \mu$, then \\
			--- $\mathcal{K} \in SC$ and $\bar{\mathcal{K}} \in SC$ \\ <Computable iff SC property>
			--- $\bar{\mathcal{K}} \inot \in SC$ \\ <N-complement of K is not semi-computable>
			--- $\bar{\mathcal{K}} \in SC$ and $\bar{\mathcal{K}} \inot \in SC$ \\ <Conjunction>
			--- CONTRADICTION !! \\
		-- $\chi_{\mathcal{K}} \inot \in \mu$ \\ <Metaproof by contradiction>
	===================================================================================================================
\subsection{(Metatheorem) SC subset of N-complement of K contains a nonSC element} %=7.6.13
	- For any $e$, if $\mathcal{W}_e \subseteq \bar{\mathcal{K}}$, then $e \in \bar{\mathcal{K}} \setminus \mathcal{W}_e$	\\
	- Proof: \\
		-- For any $e$, if $\mathcal{W}_e \subseteq \bar{\mathcal{K}}$, then \\
			--- If $a \in \mathcal{W}_e$, then $a \in \bar{\mathcal{K}}$ <(I)> \\ <Hypothesis>
			--- $b \in \bar{\mathcal{K}}$ iff $b \inot \in \mathcal{W}_b$ <(II)> \\ <Definition>
			--- If $e \in \mathcal{W}_e$, then \\
				--- $e \in \bar{\mathcal{K}}$ \\ <(I)>
				--- $e \inot \in \mathcal{W}_e$ \\ <(II)>
				--- CONTRADICTION !!
			--- $e \inot \in \mathcal{W}_e$ \\ <Metaproof by contradiction>
			--- $e \in \bar{\mathcal{K}}$ \\ <(II)>
			--- $e \in \bar{\mathcal{K}} \setminus \mathcal{W}_e$ \\ <Conjunction>
	===================================================================================================================

\subsection{(Metatheorem) Sigma formulas can emulate computable functions} %=7.7.1
	- For any $f \in \mu$, there exists $\phi(\vdc{x_i}{i=1}{Arity(f)}, y) \in \Sigma_{Form}$, $f(\vdc{a_i}{i=1}{Arity(f)}) = b$ iff $\mathfrak{N} \vDash \sub{\phi}{\vdc{x_i}{i=1}{Arity(f)}, y}{\vdc{\overleftarrow{a_i}}{i=1}{Arity(f)}, \overleftarrow{b}}$ \\ <TODO CLEANUP>
	- Proof: \\
		-- If $f = S$, then \\
			--- There exists $\phi(x, y) \in \Sigma_{Form}$, $\phi(x, y) \is S(x) \equiv y$ \\ <Definition>
			--- $f(a) = b$ iff \\
				---- $b = S(a)$ iff \\
				---- $b = a + 1$ iff \\
				---- $\mathfrak{N} \vDash S(\overleftarrow{a}) \equiv \overleftarrow{b}$ iff \\
				---- $\mathfrak{N} \vDash \sub{\phi}{x, y}{\overleftarrow{a}, \overleftarrow{b}}$ \\
			--- There exists $\phi(x, y) \in \Sigma_{Form}$, $f(a) = b$ iff $\mathfrak{N} \vDash \sub{\phi}{x, y}{\overleftarrow{a}, \overleftarrow{b}}$ \\
		-- If $f = \mathcal{I}_i^n$, then \\
			--- There exists $\phi(\vdc{x_j}{j=1}{n}, y) \in \Sigma_{Form}$, $\phi(\vdc{x_j}{j=1}{n}, y) \is \land(\vdc{x_j \equiv x_j}{j=1}{n}) \land (x_i \equiv y)$ \\ <Definition>
			--- $f(\vdc{a_j}{j=1}{n}) = b$ iff \\
				---- $\mathcal{I}_i^n(\vdc{a_j}{j=1}{n}) = b$ iff \\
				---- $b = a_i$ iff \\
				---- $\mathfrak{N} \vDash \land(\vdc{x_j \equiv x_j}{j=1}{n}) \land (\overleftarrow{a_i} \equiv \overleftarrow{b})$ iff \\
				---- $\mathfrak{N} \vDash \sub{\phi}{x, y}{\overleftarrow{a}, \overleftarrow{b}}$ \\
		-- If $f = \mathcal{O}$, then \\
			--- There exists $\phi(y) \in \Sigma_{Form}$, $\phi(y) \is y \equiv 0$ \\ <Definition>
			--- $f() = b$ iff \\
				---- $b = \mathcal{O}()$ iff \\
				---- $b = 0$ iff \\
				---- $\mathfrak{N} \vDash \overleftarrow{b} \equiv \overleftarrow{0}$ iff \\
			---- $\mathfrak{N} \vDash \sub{\phi}{x, y}{\overleftarrow{a}, \overleftarrow{b}}$ \\
		-- If $f(\vdc{x_i}{i=1}{n}) = h(\vdc{g_j(\vdc{x_i}{i=1}{n})}{j=1}{m})$ and \\ TODO CLEAN UP
			--- For any $z \in \set{h, \vdc{g_j}{j=1}{m}}$, there exists $\phi_z(\vdc{x_i}{i=1}{Arity(z)}, y) \in \Sigma_{Form}$, \\
			--- $z(\vdc{a_{z, i}}{i=1}{Arity(z)}) = b_z$ iff $\mathfrak{N} \vDash \sub{\phi_z}{\vdc{x_i}{i=1}{Arity(z)}, y}{\vdc{\overleftarrow{a_{z, i}}}{i=1}{Arity(z)}, \overleftarrow{b_z}}$), then \\
				---- There exists $\phi(\vdc{x_i}{i=1}{n}, y) \in \Sigma_{Form}$, $\phi(\vdc{x_i}{i=1}{n}, y) \is \vdc{(\exists y_j)}{j=1}{m}(\land(\vdc{\phi_{g_j}(\vdc{x_i}{i=1}{n}, y_j)}{j=1}{m}) \land \phi_h(\vdc{y_j}{j=1}{m}, y))$ \\ <Definition>
				---- $f(\vdc{a_i}{i=1}{n}) = b$ iff \\
					----- $b = b_h = h(\vdc{b_{g_j}}{j=1}{m})$ and $\vdc{b_{g_j} = g_j(\vdc{a_i}{i=1}{n})}{j=1}{m}$ iff \\ <Definition>
					----- $\mathfrak{N} \vDash \sub{\phi_{h}}{\vdc{x_i}{i=1}{m}, y}{\vdc{\overleftarrow{a_{h, i}}}{i=1}{m}, \overleftarrow{b_h}}$ and $\vdc{\mathfrak{N} \vDash \sub{\phi_{g_j}}{\vdc{x_i}{i=1}{n}, y}{\vdc{\overleftarrow{a_{g_j, i}}}{i=1}{n}, \overleftarrow{b_{g_j}}}}{j=1}{m}$ iff \\ <Inductive hypothesis>
					----- $\mathfrak{N} \vDash \vdc{(\exists y_j)}{j=1}{m}(\land(\vdc{\phi_{g_j}(\vdc{\overleftarrow{a_i}}{i=1}{n}, y_j)}{j=1}{m}) \land \phi_h(\vdc{y_j}{j=1}{m}, \overleftarrow{b}))$ iff \\ <Substitution and modification identity on models>
					----- $\mathfrak{N} \vDash \sub{\phi}{\vdc{x_i}{i=1}{n}, y}{\vdc{\overleftarrow{a_i}}{i=1}{Arity(n)}, \overleftarrow{b}}$ \\
		-- If $f(\vdc{x_i}{i=1}{n}, 0) = g(\vdc{x_i}{i=1}{n})$ and $f(\vdc{x_i}{i=1}{n}, y+1) = h(\vdc{x_i}{i=1}{n}, y, f(\vdc{x_i}{i=1}{n}, y))$ and \\
			--- $a = (t)_i$ iff $\mathfrak{N} \vDash \sub{IE}{x, y, z}{\overleftarrow{a}, \overleftarrow{i}, \overleftarrow{t}}$ \\ <TODO 5.6>
			--- For any $z \in \set{g, h}$, there exists $\phi_z(\vdc{x_{z, i}}{i=1}{Arity(z)}, y_z) \in \Sigma_{Form}$, \\
			--- $z(\vdc{a_{z, i}}{i=1}{Arity(z)}) = b_z$ iff $\mathfrak{N} \vDash \sub{\phi_z}{\vdc{x_{z, i}}{i=1}{Arity(z)}, y_z}{\vdc{\overleftarrow{a_{z, i}}}{i=1}{Arity(z)}, \overleftarrow{b_z}}$), then \\
				---- There exists $\phi(\vdc{x_i}{i=1}{n}, z, y) \in \Sigma_{Form}$, $\phi(\vdc{x_i}{i=1}{n}, z, y) \is \exists t(IE(y, S(y), t) \land \\ \exists y_0(IE(y_0, S(0), t) \land \sub{\phi_g}{\vdc{x_{g, i}}{i=1}{n}, y_g}{\vdc{x_i}{i=1}{n}, y_0}) \land \\ (\forall i < z)(\exists u, v)(IE(u, S(i), t) \land IE(v, S(S(i)), t) \land \sub{\phi_h}{\vdc{x_{g, i}}{i=1}{n+2}, y_h}{\vdc{x_i}{i=1}{n}, i, v, u}))$ \\ <Definition>
				---- $f(\vdc{a_i}{i=1}{n}, c+1) = b$ iff \\
					----- $b = h(\vdc{a_i}{i=1}{n}, c, f(\vdc{a_i}{i=1}{n}, c))$ iff \\ <Definition>
					----- $\mathfrak{N} \vDash \exists t(IE(y, S(y), t) \land \\ \exists y_0(IE(y_0, S(0), t) \land \sub{\phi_g}{\vdc{x_{g, i}}{i=1}{n}, y_g}{\vdc{x_i}{i=1}{n}, y_0}) \land \\ (\forall i < z)(\exists u, v)(IE(u, S(i), t) \land IE(v, S(S(i)), t) \land \sub{\phi_h}{\vdc{x_{g, i}}{i=1}{n+2}, y_h}{\vdc{x_i}{i=1}{n}, i, v, u}))$ iff \\ <Induction>
					----- $\mathfrak{N} \vDash \sub{\phi}{\vdc{x_i}{i=1}{n}, y}{\vdc{\overleftarrow{a_i}}{i=1}{Arity(n)}, \overleftarrow{b}}$ \\
		-- If $f(\vdc{x_i}{i=1}{n}) = \mu_{US}(g)(\vdc{x_i}{i=1}{n})$ and there exists $\phi_g(\vdc{x_{g, i}}{i=1}{n+1}, y_g) \in \Sigma_{Form}$, $g(\vdc{a_{g, i}}{i=1}{n+1}) = b_g$ iff $\mathfrak{N} \vDash \sub{\phi_g}{\vdc{x_{g, i}}{i=1}{n+1}, y_g}{\vdc{\overleftarrow{a_{g, i}}}{i=1}{n+1}, \overleftarrow{b_g}}$), then \\
			--- There exists $\phi(\vdc{x_i}{i=1}{n}, y) \in \Sigma_{Form}$, $\phi(\vdc{x_i}{i=1}{n}, y) \is (\forall i < y)(\phi_g(\vdc{x_i}{i=1}{n}, y, 0) \land \exists u(\phi_g(\vdc{x_i}{i=1}{n}, i, u) \land \lnot(u \equiv 0))$ \\ <Definition>
			--- $f(\vdc{a_i}{i=1}{n}) = b$ iff \\
				---- $g(\vdc{a_i}{i=1}{n}, b) = 0$ and for any $b_< < b$, $g(\vdc{a_i}{i=1}{n}, b_<) > 0$ iff \\ <Definition>
				---- $\mathfrak{N} \vDash \sub{\phi}{\vdc{x_i}{i=1}{n}, y}{\vdc{\overleftarrow{a_i}}{i=1}{n}, \overleftarrow{b}}$ \\ <Misc. Semantics>
		-- For any $f \in \mu$, there exists $\phi(\vdc{x_i}{i=1}{Arity(f)}, y) \in \Sigma_{Form}$, $f(\vdc{a_i}{i=1}{Arity(f)}) = b$ iff $\mathfrak{N} \vDash \sub{\phi}{\vdc{x_i}{i=1}{Arity(f)}, y}{\vdc{\overleftarrow{a_i}}{i=1}{Arity(f)}, \overleftarrow{b}}$ \\ <Induction>
	===================================================================================================================
\subsection{(Metatheorem) Sigma formulas can emulate SC sets} %=7.7.1
	- For any $A \in SC$, there exists $\theta(x) \in \Sigma_{Form}$, $\mathfrak{N} \vDash \sub{\theta}{x}{\overleftarrow{a}}$ iff $a \in A$ \\
	- Proof: \\
		-- For any $A \in SC$, \\
			--- There exists $f' \in \mu$, $dom(f') = A$ \\ <Definition>
			--- There exists $g' \in \mu$, $g'(x) = 0 \dot{-} f'(x)$ <(I)> \\ <Misc. theorems>
			--- $g'(x) = 0$ iff $x \in dom(f')$ iff $x \in A$ \\ <(I)>
			--- There exists $\phi'(x, y) \in \Sigma_{Form}$, $g'(a) = 0$ iff $\mathfrak{N} \vDash \sub{\phi'}{x, y}{\overleftarrow{a}, \overleftarrow{0}}$ \\ <Sigma formulas can emulate computable functions>
			--- $\mathfrak{N} \vDash \sub{\phi'}{x, y}{\overleftarrow{a}, \overleftarrow{0}}$ iff $g'(a) = 0$ iff $a \in A$ <(II)> \\ <(I)>
			--- There exists $\theta(x)$, $\theta \is \sub{\phi'}{y}{\overleftarrow{0}}$ \\
			--- There exists $\theta(x) \in \Sigma_{Form}$, $\sub{\theta}{x}{\overleftarrow{a}}$ iff $a \in A$ \\ <(II)>
	- Basically, emulate the characteristic of the domain \\
	===================================================================================================================
\subsection{(Metatheorem) Sigma formulas can emulate K} %=7.7.1
	- There exists $\theta(x) \in \Sigma_{Form}$, $\mathfrak{N} \vDash \sub{\theta}{x}{\overleftarrow{a}}$ iff $a \in \mathcal{K}$ \\
	- Proof: \\ <Sigma formulas can emulate SC sets>
	===================================================================================================================
\subsection{(Metatheorem) Pi formulas can emulate N-complement of K} %=7.7.1
	- There exists $\psi(x) \in \Pi_{Form}$, $\mathfrak{N} \vDash \sub{\psi}{x}{\overleftarrow{a}}$ iff $a \in \bar{\mathcal{K}}$ and \\
		-- There exists $\theta(x) \in \Sigma_{Form}$, $\mathfrak{N} \vDash \sub{\theta}{x}{\overleftarrow{a}}$ iff $a \in \mathcal{K}$ and \\
		-- $\sub{\psi}{x}{\overleftarrow{a}} \vDash \lnot \sub{\theta}{x}{\overleftarrow{a}}$ and $\lnot \sub{\theta}{x}{\overleftarrow{a}} \vDash \sub{\psi}{x}{\overleftarrow{a}}$ \\
	- Proof: \\ <Sigma formulas can emulate K, De Morgan's>
	===================================================================================================================
\subsection{(Definition) Weak number theory conjunction} %=7.7.2
	- The formula $N_{\land}$ is defined as $N_{\land} = \land(\vdc{\phi}{\phi \in N}{ })$ \\
	===================================================================================================================
\subsection{(Metatheorem) Undecidability of the Entscheidungsproblem} %=7.7.2
	- The set of all valid formulas $\mathcal{E}$ is defined as $\mathcal{E} = \set{GN(\phi): \phi \in Form(\mathcal{L}_{NT}) \txtand \vDash \phi}$ \\
	- $\chi_{\mathcal{E}} \inot \in \mu$ \\
	- Proof: \\
		-- $a \in \mathcal{K}$ iff \\
			--- There exists $\phi(x) \in \Sigma_{Form}$, $\mathfrak{N} \vDash \sub{\phi}{x}{\overleftarrow{a}}$ iff \\
			--- $N \vdash \sub{\phi}{x}{\overleftarrow{a}}$ iff \\ <TODO 5.3.13>
			--- $\vdash N_{\land} \implies \sub{\phi}{x}{\overleftarrow{a}}$ iff \\ <Deduction theorem>
			--- $\vDash N_{\land} \implies \sub{\phi}{x}{\overleftarrow{a}}$ \\ <Completeness theorem>
		-- There exists $\phi(x) \in \Sigma_{Form}$, $a \in \mathcal{K}$ iff $\vDash N_{\land} \implies \sub{\phi}{x}{\overleftarrow{a}}$ \\ <Abbreviate>
		-- There exists $g \in \mu$, $g(n) = GN(N_{\land} \implies \sub{\phi}{x}{\overleftarrow{n}})$ \\ <Misc. theorems>
		-- If $\chi_{\mathcal{E}} \in \mu$, then \\
			--- There exists $f \in \mu$, $f(x) = \chi_{\mathcal{E}}(g(x))$ \\ <Misc. theorems>
			--- $f(n) = 0$ iff \\
				---- $\chi_{\mathcal{E}}(g(n)) = 0$ iff \\
				---- $\vDash N_{\land} \implies \sub{\phi}{x}{\overleftarrow{n}}$ iff \\
				---- $n \in \mathcal{K}$ \\
			--- $f = \chi_{\mathcal{K}}$ \\
			--- $\chi_{\mathcal{K}} \inot \in \mu$ \\ <K is not computable>
			--- $\chi_{\mathcal{K}} \in \mu$ and $\chi_{\mathcal{K}} \inot \in \mu$ \\ <Conjunction> 
			--- CONTRADICTION !!
		-- $\chi_{\mathcal{E}} \inot \in \mu$ \\ <Metaproof by contradiction>
	===================================================================================================================
\subsection{(Metatheorem) SC axioms yields SC theorems} %=7.7.3
	- If $\set{\phi(x)} \cup A \subseteq Form(\mathcal{L}_{NT})$ and $\set{GN(\eta): A \vdash \eta} \in SC$, then $\set{a: A \vdash \sub{\phi}{x}{\overleftarrow{a}}} \in SC$ \\
	- Proof: \\
		-- If $\phi(x) \in Form(\mathcal{L}_{NT})$ and $A \subseteq Form(\mathcal{L}_{NT}$ and $\set{GN(\eta): A \vdash \eta} \in SC$, then \\
			--- There exists $f \in \mu$, $dom(f) = \set{GN(\eta): A \vdash \eta}$ <(I)> \\ <Definition>
			--- There exists $g \in \mu$, $g(n) = GN(\sub{\phi}{x}{\overleftarrow{n}})$ \\ <Misc. theorems>
			--- There exists $h \in \mu$, $h(m) = f(g(m))$ \\ <Misc. theorems>
			--- $a \in dom(h)$ iff \\
				---- There exists $b \in \mathbb{N}$, $h(a) = b$ iff \\ <Definition>
				---- There exists $b \in \mathbb{N}$, $f(GN(\sub{\phi}{x}{\overleftarrow{a}})) = b$ iff \\
				---- $A \vdash \sub{\phi}{x}{\overleftarrow{a}}$ \\ <(I)>
			--- There exists $h \in \mu$, $dom(h) = \set{a: A \vdash \sub{\phi}{x}{\overleftarrow{a}}} \in SC$ \\
			--- $\set{a: A \vdash \sub{\phi}{x}{\overleftarrow{a}}} \in SC$ \\ <Definition>
	===================================================================================================================
\subsection{(Metatheorem) Incompleteness theorem version I} %=7.7.4
	- If $A \subseteq Form(\mathcal{L}_{NT})$ and $\mathfrak{N} \vDash A$ and $\set{GN(\eta): A \vdash \eta} \in SC$, then there exists $\theta \in \Pi_{Form}$, $\mathfrak{N} \vDash \theta$ and $A \inot \vdash \theta$ \\
	- Proof: \\
		-- If $A \subseteq Form(\mathcal{L}_{NT})$ and $\mathfrak{N} \vDash A$ and $\set{GN(\eta): A \vdash \eta} \in SC$, then \\
			--- There exists $\psi(x) \in \Pi_{Form}$, $\mathfrak{N} \vDash \sub{\psi}{x}{\overleftarrow{a}}$ iff $a \in \b	r{\mathcal{K}}$ \\ <Pi formulas can emulate N-complement of K>
			--- $\bar{\mathcal{K}} = \set{a: \mathfrak{N} \vDash \sub{\psi}{x}{\overleftarrow{a}}}$ \\
			--- $\set{a: A \vDash \sub{\psi}{x}{\overleftarrow{a}}} \subseteq \set{a: \mathfrak{N} \vDash \sub{\psi}{x}{\overleftarrow{a}}}$ \\ <Hypothesis>
			--- $\set{a: A \vdash \sub{\psi}{x}{\overleftarrow{a}}} \subseteq \set{a: A \vDash \sub{\psi}{x}{\overleftarrow{a}}}$ \\ <Soundness theorem>
			--- $\set{a: A \vdash \sub{\psi}{x}{\overleftarrow{a}}} \subseteq \bar{\mathcal{K}}$ \\ <Conjunction>
			--- $\set{a: A \vdash \sub{\psi}{x}{\overleftarrow{a}}} \in SC$ \\ <SC axioms yields SC theorems>
			--- $\bar{\mathcal{K}} \inot \in SC$ \\ <N-complement of K is not semi-computable>
			--- $\bar{\mathcal{K}} \inot = \set{a: A \vdash \sub{\psi}{x}{\overleftarrow{a}}}$ \\ <Conjunction>
			--- There exists $\theta \in \bar{\mathcal{K}} \setminus \set{a: A \vdash \sub{\psi}{x}{\overleftarrow{a}}}$, \\
				---- $\theta \in \bar{\mathcal{K}}$ and $\theta \inot \in \set{a: A \vdash \sub{\psi}{x}{\overleftarrow{a}}}$ \\
				---- $\mathfrak{N} \vDash \sub{\theta}{x}{\overleftarrow{a}}$ and $A \inot \vdash \sub{\theta}{x}{\overleftarrow{a}}$ \\
	===================================================================================================================
\subsection{(Definition) Theory extension} %=7.7.5
	- The theory $A$ extends the theory $B$ ($extends(A, B)$) iff for any $\phi \in \mathcal{L}$, if $B \vdash \phi$, then $A \vdash \phi$ \\
	- Alternative definition: $extends(A, B)$ iff $A \vdash B$ \\
	===================================================================================================================	
\subsection{(Metatheorem) Incompleteness theorem version II} %=7.7.5
	- If $A \subseteq Form(\mathcal{L}_{NT})$ and $A \inot \vdash \contr$ and $\set{GN(\eta): A \vdash \eta} \in SC$, then there exists $\theta \in \Pi_{Form}$, $\mathfrak{N} \vDash \theta$ and $A \inot \vdash \theta$ \\
	- Proof: \\
		-- If $\pnot{extends(A, N)}$, then \\
			--- $A \inot \vdash N$ \\ <Definition>
			--- There exists $\alpha \in N$, $A \inot \vdash \alpha$ \\ <Definition>
			--- There exists $\theta \in \Pi_{Form}$, $\theta \is N_{\land}$ \\ <Definition>
			--- $\mathfrak{N} \vDash \theta$ \\
			--- $A \inot \vdash \theta$ \\ <PC>	
			--- There exists $\theta \in \Pi_{Form}$, $\mathfrak{N} \vDash \theta$ and $A \inot \vdash \theta$ \\
		-- If $extends(A, N)$, then \\
			--- $A \vdash N$ <(I)> \\ <Definition>
			--- There exists $\phi(x) \in \Sigma_{Form}$, $\mathfrak{N} \vDash \sub{\phi}{x}{\overleftarrow{a}}$ iff $a \in \mathcal{K}$ \\ <Sigma formulas can emulate K>
			--- $a \in \mathcal{K}$ iff \\
				---- $\mathfrak{N} \vDash \sub{\phi}{x}{\overleftarrow{a}}$ iff \\
				---- $N \vdash \sub{\phi}{x}{\overleftarrow{a}}$ \\ <TODO 5.3.13>
			--- $a \in \mathcal{K}$ iff $N \vdash \sub{\phi}{x}{\overleftarrow{a}}$ \\ <Abbreviate>
			--- If $N \vdash \sub{\phi}{x}{\overleftarrow{a}}$, then $A \vdash \sub{\phi}{x}{\overleftarrow{a}}$ \\ <(I)>
			--- If $a \in \mathcal{K}$, then $A \vdash \sub{\phi}{x}{\overleftarrow{a}}$ \\ <Conjunction>
			--- If $A \inot \vdash \sub{\phi}{x}{\overleftarrow{a}}$, then $a \inot \in \mathcal{K}$ <(II)> \\ <Contrapositive>
			--- If $A \vdash \lnot \sub{\phi}{x}{\overleftarrow{a}}$, then \\
				---- $A \inot \vdash \sub{\phi}{x}{\overleftarrow{a}}$ \\ <Hypothesis>
				---- $a \inot \in \mathcal{K}$ \\ <(II)>
				---- $a \in \bar{\mathcal{K}}$ \\ <Definition>
			--- If $A \vdash \lnot \sub{\phi}{x}{\overleftarrow{a}}$, then $a \in \bar{\mathcal{K}}$ <(III)> \\ <Abbreviate>
			--- There exists $\psi(x) \in \Pi_{Form}$, <(IV)> \\ <Pi formulas can emulate N-complement of K>
				---- $\sub{\psi}{x}{\overleftarrow{a}} \vDash \sub{\lnot \phi}{x}{\overleftarrow{a}}$ and \\
				---- $\sub{\lnot \phi}{x}{\overleftarrow{a}} \vDash \sub{\psi}{x}{\overleftarrow{a}} \vDash$ and \\
				---- $\mathfrak{N} \vDash \sub{\psi}{x}{\overleftarrow{a}}$ iff $a \in \bar{\mathcal{K}}$ \\
			--- $A \vdash \sub{\psi}{x}{\overleftarrow{a}}$ iff \\
				---- $A \vDash \sub{\psi}{x}{\overleftarrow{a}}$ iff \\ <Soundness theorem>
				---- $A \vDash \sub{\lnot \phi}{x}{\overleftarrow{a}}$ iff \\ <(IV)>
				---- $A \vdash \sub{\lnot \phi}{x}{\overleftarrow{a}}$ \\ <Completeness theorem>
			--- $A \vdash \sub{\psi}{x}{\overleftarrow{a}}$ iff $A \vdash \sub{\lnot \phi}{x}{\overleftarrow{a}}$ <(V)> \\ <Abbreviate>
			--- If $A \vdash \sub{\psi}{x}{\overleftarrow{a}}$, then \\
				---- $A \vdash \sub{\lnot \phi}{x}{\overleftarrow{a}}$ \\ <(V)>
				---- $a \in \bar{\mathcal{K}}$ \\ <(III)>
				---- $\mathfrak{N} \vDash \sub{\psi}{x}{\overleftarrow{a}}$ \\ <(IV)>
			--- If $A \vdash \sub{\psi}{x}{\overleftarrow{a}}$, then $\mathfrak{N} \vDash \sub{\psi}{x}{\overleftarrow{a}}$ \\ <Abbreviate>
			--- $\set{a: A \vdash \sub{\psi}{x}{\overleftarrow{a}}} \subseteq \set{a: \mathfrak{N} \vDash \sub{\psi}{x}{\overleftarrow{a}}}$ \\
			--- $\bar{\mathcal{K}} = \set{a: \mathfrak{N} \vDash \sub{\psi}{x}{\overleftarrow{a}}}$ \\ <(IV)>
			% DUPLICATE LOGIC FROM 7.5.4 ===================================================================================================================
			--- $\set{a: A \vdash \sub{\psi}{x}{\overleftarrow{a}}} \subseteq \set{a: A \vDash \sub{\psi}{x}{\overleftarrow{a}}}$ \\ <Soundness theorem>
			--- $\set{a: A \vdash \sub{\psi}{x}{\overleftarrow{a}}} \subseteq \bar{\mathcal{K}}$ \\ <Conjunction>
			--- $\set{a: A \vdash \sub{\psi}{x}{\overleftarrow{a}}} \in SC$ \\ <SC axioms yields SC theorems>
			--- $\bar{\mathcal{K}} \inot \in SC$ \\ <N-complement of K is not semi-computable>
			--- $\bar{\mathcal{K}} \inot = \set{a: A \vdash \sub{\psi}{x}{\overleftarrow{a}}}$ \\ <Conjunction>
			--- There exists $\theta \in \bar{\mathcal{K}} \setminus \set{a: A \vdash \sub{\psi}{x}{\overleftarrow{a}}}$, \\
				---- $\theta \in \bar{\mathcal{K}}$ and $\theta \inot \in \set{a: A \vdash \sub{\psi}{x}{\overleftarrow{a}}}$ \\
				---- $\mathfrak{N} \vDash \sub{\theta}{x}{\overleftarrow{a}}$ and $A \inot \vdash \sub{\theta}{x}{\overleftarrow{a}}$ \\
			--- There exists $\theta \in \Pi_{Form}$, $\mathfrak{N} \vDash \theta$ and $A \inot \vdash \theta$ \\ <Abbreviate>
			% DUPLICATE LOGIC FROM 7.5.4 ===================================================================================================================
		-- There exists $\theta \in \Pi_{Form}$, $\mathfrak{N} \vDash \theta$ and $A \inot \vdash \theta$ \\ <Conjunction>
	===================================================================================================================
	===================================================================================================================
\subsection{(Remarks) Incompleteness theorem intuition} %=7.7.7
	- From an intuitive computability-theoretic point of view, the first Incompleteness Theorem is an inevitable consequence of the fact that we can define an undecidable set in the $\mathcal{L}_{NT}$-structure $\mathfrak{N}$. In other words, there exists $\phi(x) \in \mathcal{L}_{NT}$, $\mathfrak{N} \vDash \sub{\phi}{x}{\overleftarrow{a}}$ iff $a \in \mathcal{K}$. \\
	- Since we can define an undecidable set in $\mathfrak{N}$, no semi-decidable set of axioms of $\mathcal{L}_{NT}$ will be complete for $\mathfrak{N}$. \\
	- If there were such a set of axioms, we could decide membership in an undecidable set. Otherwise, we could decide if $a$ is a member of $\mathcal{K}$ by enumerating deductions until we encountered a proof or a refutation of $\sub{\phi}{x}{\overleftarrow{a}}$. \\
	- The expressive power (the standard interpretation) of the language $\mathcal{L}_{NT}$ is essential. To define an undecidable set like $\mathcal{K}$, we need an expressive language. \\
	===================================================================================================================



	===================================================================================================================
	===================================================================================================================
	===================================================================================================================



TODO:
	Lowenhiem Skolem + model theory
	Rice's theorem
	Lindström's theorem


FORMAT:
	- out of scope lemma: (I, II, III, ...)
	- inside of scope lemma: (1, 2, 3, ...)
	- annotations: <NEW REF> \\ <CAUSE REF>

TODO: add Incomleteness theorem III, Rice's theorem, others ???
TODO: OVERLEFT ARROW ABBREVIATES vdc{S}...
TODO: One liner theorems on comment header ??

?
TODO: assumption contexts
TODO: DEFINITIONS WITH: SATISFIES ANY OF THE FOLLOWING: <CONJUNCTIONS> IS MUCH CLEARER THAN IF X, THEN Y DEFINITIONS
TODO: RECURSION BY STAGE + RECURSION BY STRUCTURE
TODO: max largest biggest symbolic qualifier for sets like (set of all free variables contained in phi or something)
TODO: do decidable metatheorems: 1.8.1.7 / 2.4.3.1-2

TODO check mistakes: 
	- FIX BAD SMELL: IMPLICIT ASSUMPTIONS
	- re-write IF with IFF appropriate definitions like inferences
	- PC ONLY AFFECTS PROPOSITIONAL VAR, NOT ALPHABET VAR

\subsection{(Notation) Retarded notation - free occurrence} %=1.5.2/1.5.1.6
	- $\phi(x)$ means $x$ is free in $\phi$ \\
	- $\phi(t)$ means substitute $x$ by $t$ in $\phi$ \\
	===================================================================================================================

\begin{comment}
TODO:
- maybe dont abbreviate
-\txt formatting
-definitions should be iff
	===================================================================================================================

Hotkeys:
Ctrl+R: find reference
Ctrl+K, Ctrl+1: code fold all
Ctrl+K, Ctrl+J: code unfold all
Ctrl+Shift+[: code fold
Ctrl+Shift+]: code unfold
Ctrl+Shift+Right-click+Drag: block select
Ctrl+Space: display auto-complete
Tab: tab trigger auto-complete
Shift+Tab: force tab

=== math.sublime-completions ===
{ "completions": [

{ "trigger": "fl\t\\forall", "contents": "\\forall " },
{ "trigger": "es\t\\exists", "contents": "\\exists " },
{ "trigger": "ev\t\\equiv", "contents": "\\equiv " },
{ "trigger": "es\t\\implies", "contents": "\\implies " },
{ "trigger": "if\t\\iff", "contents": "\\iff " },
{ "trigger": "and\t\\land", "contents": "\\land " },
{ "trigger": "or\t\\lor", "contents": "\\lor " },
{ "trigger": "not\t\\lnot", "contents": "\\lnot " },
{ "trigger": "dr\t\\vdash", "contents": "\\vdash " },
{ "trigger": "md\t\\vDash", "contents": "\\vDash " },
{ "trigger": "cl\t\\mathcal", "contents": "\\mathcal{}" },
{ "trigger": "fk\t\\mathfrak", "contents": "\\mathfrak{}" },
{ "trigger": "mb\t\\mathbb", "contents": "\\mathbb{}" },
{ "trigger": "nit\t\\inot", "contents": "\\inot " },
{ "trigger": "cr\t\\contr", "contents": "\\contr " },
{ "trigger": "phr\t\\placeholder", "contents": "\\placeholder" },

{ "trigger": "vdc\t\\vdc{_i}{i=1}{Arity()}", "contents": "\\vdc{_i}{i=1}{Arity()}" },
{ "trigger": "sb\t\\sub{}{}{}", "contents": "\\sub{}{}{}" },
{ "trigger": "sq\t\\subseteq", "contents": "\\subseteq " },
{ "trigger": "cp\t\\cup", "contents": "\\cup " },
{ "trigger": "rw\t\\rightarrow", "contents": "\\rightarrow " },

{ "trigger": "a\t\\alpha", "contents": "\\alpha " },
{ "trigger": "b\t\\beta", "contents": "\\beta " },
{ "trigger": "g\t\\gamma", "contents": "\\gamma " },
{ "trigger": "f\t\\phi", "contents": "\\phi " },
{ "trigger": "s\t\\psi", "contents": "\\psi " },
{ "trigger": "m\t\\sigma", "contents": "\\sigma " },
{ "trigger": "t\t\\theta", "contents": "\\theta " },

{ "trigger": "A\t\\Alpha", "contents": "\\Alpha " },
{ "trigger": "B\t\\Beta", "contents": "\\Beta " },
{ "trigger": "G\t\\Gamma", "contents": "\\Gamma " },
{ "trigger": "F\t\\Phi", "contents": "\\Phi " },
{ "trigger": "S\t\\Psi", "contents": "\\Psi " },
{ "trigger": "M\t\\Sigma", "contents": "\\Sigma " },
{ "trigger": "T\t\\Theta", "contents": "\\Theta " },

] }
=== math.sublime-completions ===

\subsection{(Metatheorem) Reduct modellerers} % CUT AWAY
	- FUCK MODEL THEORY, PROVE FOR HENKIN
	- For any $\phi \in Form(\mathcal{L})$, $\mathfrak{A}^+ \vDash \phi$ iff $\mathfrak{A}^+ \upharpoonright_\mathcal{L} \vDash \phi$ \\
	- Proof: \\
		-- $\set{s: Var(\mathcal{L}^+) \rightarrow Universe(\mathfrak{A}^+)} = \set{s: Var(\mathcal{L}) \rightarrow Universe(\mathfrak{A})}$ \\  <Definition>
		-- If $\phi \is t_1 \equiv t_2$, then \\
			--- $\mathfrak{A}^+ \vDash \phi$ iff \\
				---- $\mathfrak{A}^+ \vDash t_1 \equiv t_2$ iff \\
				---- For any $s$, $\extend{s}(t_1) = \extend{s}(t_2)$ iff \\ <Definition>
				---- $\mathfrak{A}^+ \upharpoonright_\mathcal{L} \vDash t_1 \equiv t_2$ iff \\ <Definition>
				---- $\mathfrak{A}^+ \upharpoonright_\mathcal{L} \vDash \phi$ \\
			--- $\mathfrak{A}^+ \vDash \phi$ iff $\mathfrak{A}^+ \upharpoonright_\mathcal{L} \vDash \phi$ \\ <Abbreviate>
		-- If $\phi \is P \vdc{t_i}{i=1}{Arity(P)}$, then \\
			--- $\means{P}{\mathfrak{A}^+} = \means{P}{\mathfrak{A}^+ \upharpoonright_\mathcal{L}}$ \\ <Definition>
			--- $\mathfrak{A}^+ \vDash \phi$ iff \\
				---- $\mathfrak{A}^+ \vDash P \vdc{t_i}{i=1}{Arity(P)}$ iff \\
				---- For any $s$, $\seq{\vdc{\extend{s}(t_i)}{i=1}{Arity(P)}} \in \means{P}{\mathfrak{A}^+}$ iff \\ <Definition>
				---- For any $s$, $\seq{\vdc{\extend{s}(t_i)}{i=1}{Arity(P)}} \in \means{P}{\mathfrak{A}^+ \upharpoonright_\mathcal{L}}$ iff \\
				---- $\mathfrak{A}^+ \upharpoonright_\mathcal{L} \vDash P \vdc{t_i}{i=1}{Arity(P)}$ iff \\ <Definition>
				---- $\mathfrak{A}^+ \upharpoonright_\mathcal{L} \vDash \phi$ \\
			--- $\mathfrak{A}^+ \vDash \phi$ iff $\mathfrak{A}^+ \upharpoonright_\mathcal{L} \vDash \phi$ \\ <Abbreviate>
		-- If $\phi \is \lnot \alpha$ and $\set{\alpha} \subseteq \set{\zeta: \mathfrak{A}^+ \vDash \zeta \txtiff \mathfrak{A}^+ \upharpoonright_\mathcal{L} \vDash \zeta}$, then \\
			--- $\mathfrak{A}^+ \vDash \phi$ iff \\
				---- $\mathfrak{A}^+ \vDash \lnot \alpha$ iff \\
				---- $\mathfrak{A}^+ \inot \vDash \alpha$ iff \\ <Definition>
				---- $\mathfrak{A}^+ \upharpoonright_\mathcal{L} \inot \vDash \alpha$ iff \\ <Inductive hypothesis>
				---- $\mathfrak{A}^+ \upharpoonright_\mathcal{L} \vDash \lnot \alpha$ iff \\ <Definition>
				---- $\mathfrak{A}^+ \upharpoonright_\mathcal{L} \vDash \phi$ \\
			--- $\mathfrak{A}^+ \vDash \phi$ iff $\mathfrak{A}^+ \upharpoonright_\mathcal{L} \vDash \phi$ \\ <Abbreviate>
		-- If $\phi \is \alpha \lor \beta$ and $\set{\alpha, \beta} \subseteq \set{\zeta: \mathfrak{A}^+ \vDash \zeta \txtiff \mathfrak{A}^+ \upharpoonright_\mathcal{L} \vDash \zeta}$, then \\
			--- $\mathfrak{A}^+ \vDash \phi$ iff \\
				---- $\mathfrak{A}^+ \vDash \alpha \lor \beta$ iff \\
				---- $\mathfrak{A}^+ \vDash \alpha$ or $\mathfrak{A}^+ \vDash \beta$ iff \\ <Definition>
				---- $\mathfrak{A}^+ \upharpoonright_\mathcal{L} \vDash \alpha$ or $\mathfrak{A}^+ \upharpoonright_\mathcal{L} \vDash \beta$ iff \\ <Inductive hypothesis>
				---- $\mathfrak{A}^+ \upharpoonright_\mathcal{L} \vDash \alpha \lor \beta$ iff \\ <Definition>
				---- $\mathfrak{A}^+ \upharpoonright_\mathcal{L} \vDash \phi$ \\
			--- $\mathfrak{A}^+ \vDash \phi$ iff $\mathfrak{A}^+ \upharpoonright_\mathcal{L} \vDash \phi$ \\
		-- If $\phi \is \forall x \alpha$ and $Stage(Comp(\phi) - 1) \subseteq \set{\zeta: \mathfrak{A}^+ \vDash \zeta \txtiff \mathfrak{A}^+ \upharpoonright_\mathcal{L} \vDash \zeta}$, then \\
			--- $Subbable(BLABLA)$ \\ <Definition>
			--- $\mathfrak{A}^+ \vDash \phi$ iff \\
				---- $\mathfrak{A}^+ \vDash \forall x \alpha$ iff \\
				---- For any $s$, $\mathfrak{A}^+ \vDash (\forall x \alpha)[s]$ iff \\ <Definition>
				---- For any $s$, for any $[t] \in A$, $\mathfrak{A}^+ \vDash \alpha[s[x|[t]]]]$ iff \\ <Definition>
				---- For any $s$, for any $[t] \in A$, $\mathfrak{A}^+ \vDash \alpha[s[x|\extend{s}(t)]]]$ iff \\ <Definition>
				---- For any $s$, for any $[t] \in A$, $\mathfrak{A}^+ \vDash \sub{\alpha}{x}{t}[s]]$ iff \\ <Definition>
				---- .........

\subsection{(Definition) PR rank} %=7.3.1.8 TODO: is this useful??? 
	- The primitive recursive function $f$ is of rank $n$ ($rank(f, n)$) iff: \\
		-- If $f \in \set{S, I_i^n, \mathcal{O}}$, then $rank(f, 0)$ \\
		-- If $f(\vdc{x_i}{i=1}{Arity(f)}) = h(\vdc{g_j(\vdc{x_i}{i=1}{Arity(f)})}{j=1}{Arity(h)}$ and for any $f^* \in \set{h, \vdc{g_j}{j=1}{Arity(h)}}$, there exists $m \leq n$, $rank(f^*, m)$, then $rank(f, n)$ \\
		-- If $f(\vdc{x_i}{i=1}{n}, 0) = g(\vdc{x_i}{i=1}{n})$ and $f(\vdc{x_i}{i=1}{n}, y+1) = h(\vdc{x_i}{i=1}{n}, y, f(\vdc{x_i}{i=1}{n}, y))$ and for any $f^* \in \set{g, h}$, there exists $m \leq n$, $rank(f^*, m)$, then $rank(f, n+1)$ \\
	- USELESS COROLLARY: If $rank(f, n)$, then $rank(f, n+1)$ \\
	===================================================================================================================

\end{comment}

\end{document}


