\documentclass{book}
\usepackage[utf8]{inputenc}
\usepackage{verbatim} % for comments
\usepackage[margin=0.5in]{geometry} % for margins
\usepackage{amsmath} % for math
\usepackage{amssymb} % for math
\usepackage{latexsym} % for math
\usepackage{mathtools} % for math
\usepackage{mathrsfs} % for math
\usepackage{cancel} % for math

\usepackage{mleftright} % for left right
\mleftright

\title{Mathematical Logic Notes}
\author{John Paul Guzman}
\date{ }
\begin{document}
\maketitle
%\tableofcontents

\chapter{Trigonometric Identities}
	- Reference: https://www.themathpage.com/aTrig/trigonometric-identities.htm \\
	===================================================================================================================
\section{Fundamental Identities}
	- $x^2 + y^2 = r^2$ \\ % <Construction with 4 translated right triangles>
	- $tan(x) = \frac{sin(x)}{cos(x)}$ \\
	- $sin(-x) = -sin(x)$ \\
	- $cos(-x) = cos(x)$ \\
	===================================================================================================================
\section{Pythagorean identities}
	- $sin^2(x) + cos^2(x) = 1$ \\
	- $1 = sec^2(x) - tan^2(x)$ \\
	- $1 = csc^2(x) - cot^2(x)$ \\
	===================================================================================================================
\section{Sum and difference formulas}
	- $sin(a+b) = sin(a) cos(b) + cos(a) + sin(b)$ \\ % <Construction with 2 stacked right triangles>
	- $sin(a-b) = sin(a) cos(b) - cos(a) + sin(b)$ \\
	- $cos(a+b) = cos(a) cos(b) - sin(a) + sin(b)$ \\
	- $cos(a-b) = cos(a) cos(b) + sin(a) + sin(b)$ \\
	===================================================================================================================
\section{Double-angle and half-angle formulas}
	- $sin(2a) = 2 sin(a) cos(a)$ \\
	- $cos(2a) = cos^2(a) - sin^2(a)$ \\
	- $sin(a/2) = \sqrt{(1/2) (1 - cos(a))}$ \\
	- $cos(a/2) = \sqrt{(1/2) (1 + cos(a))}$ \\
	- $sin^2(a) = (1/2) (1 - cos(2a))$ \\
	- $cos^2(a) = (1/2) (1 + cos(2a))$ \\
	===================================================================================================================
\section{Tangent angle formulas}
	- $tan(a+b) = \frac{tan(a) + tan(b)}{1 - tan(a) tan(b)}$ \\
	- $tan(a-b) = \frac{tan(a) - tan(b)}{1 + tan(a) tan(b)}$ \\
	- $tan(2a) = \frac{2 tan(a)}{1 - tan^2(a)}$ \\
	- $tan(a/2) = \frac{1 - cos(a)}{sin(a)}$ \\
	===================================================================================================================
\section{Products as sums formulas}
	- $sin(a) cos(b) = (1/2) (sin(a+b) + sin(a-b))$ \\
	- $cos(a) sin(b) = (1/2) (sin(a+b) - sin(a-b))$ \\
	- $cos(a) cos(b) = (1/2) (cos(a+b) + cos(a-b))$ \\
	- $sin(a) sin(b) = (-1/2) (cos(a+b) - cos(a-b))$ \\
	===================================================================================================================
\section{Sums as products formulas}
	- $sin(a) + sin(b) = 2 sin((1/2) (a + b)) cos((1/2) (a - b))$ \\
	- $sin(a) - sin(b) = 2 sin((1/2) (a - b)) cos((1/2) (a + b))$ \\
	- $cos(a) + cos(b) = 2 cos((1/2) (a + b)) cos((1/2) (a - b))$ \\
	- $cos(a) - cos(b) = -2 sin((1/2) (a + b)) sin((1/2) (a - b))$ \\
	===================================================================================================================

\chapter{Calculus}
	- Reference: http://tutorial.math.lamar.edu/Classes/CalcI/CalcI.aspx \\
	===================================================================================================================
\section{Limits}
	- Given $a, L \in \mathbb{R}$: \\
	- $(\lim_{x \to a} f(x) = L) \iff \forall_{\epsilon > 0} \exists_{\delta > 0} (0 < |x - a| < \delta \implies |f(x) - L| < \epsilon)$ \\
	- $Continuous(f, a) \iff (\lim_{x \to a} f(x) = f(a) \land f(a) \in \mathbb{R})$ \\
	- $Continuous(f, A) \iff \forall_{a \in A} Continuous(f, a)$ \\
	===================================================================================================================
\subsection{Limit properties}
	- Given $a, L, M \in \mathbb{R}$, $(\lim_{x \to a} f(x) = L)$, $(\lim_{x \to a} g(x) = M)$: \\
	- $\lim_{x \to a} (f(x) \pm g(x)) = \lim_{x \to a} f(x) \pm \lim_{x \to a} g(x) = L + M$ \\ % $\epsilon = \epsilon_f + \epsilon_g, \delta = min(\delta_f, \delta_g)$
	- $\lim_{x \to a} c = c$ \\ % $\delta > 0$
	- $\lim_{x \to a} (c f(x)) = c \lim_{x \to a} f(x) = c L$ \\ % $\epsilon = c \epsilon_f$
	- $\lim_{x \to a} (f(x) g(x)) = \lim_{x \to a} f(x) \lim_{x \to a} g(x) = L M$ \\ % $\epsilon = \epsilon_{f - L} \epsilon{g - M}, \delta = min(\delta_{f - L}, _{g - M})$
	- $\lim_{x \to a} (f(x) / g(x)) = \lim_{x \to a} f(x) / \lim_{x \to a} g(x) = L / M$ \\ % $\epsilon_1 = |M| / 2, \epsilon_2 = |M|^2 \epsilon / 2, \delta
	===================================================================================================================
\section{Derivatives}
	- $\frac{df(x)}{dx} = \lim_{h \rightarrow 0} \frac{f(x+h) - f(x)}{h}$ \\
	- $Differentiable(f, a) \iff (\frac{df(x)}{dx}(a) \in \mathbb{R})$
	- $Differentiable(f, A) \iff \forall_{a \in A} Differentiable(f, a)$ \\
	===================================================================================================================
\subsection{Derivative properties} % <From derivative and limit properties>
	- $Differentiable(f, a) \implies Continuous(f, a)$ \\
	- $\frac{d(f(x) \pm g(x))}{dx} = \frac{df(x)}{dx} \pm \frac{dg(x)}{dx}$ \\
	- $\frac{dc}{dx} = 0$ \\
	- $\frac{d(c f(x))}{dx} = c \frac{df(x)}{dx}$ \\
	- $\frac{d(f(x) g(x))}{dx} = \frac{f(x)}{dx} g(x) + f(x) \frac{dg(x)}{dx}$ \\
	- $\frac{d(f(x) / g(x))}{dx} =  \frac{\frac{f(x)}{dx} g(x) - f(x) \frac{dg(x)}{dx}}{(g(x))^2}$ \\
	===================================================================================================================
\subsection{Derivative properties from chain rule} % http://tutorial.math.lamar.edu/Classes/CalcI/DerivativeProofs.aspx#Extras_DerPf_ChainRule ; http://math2.org/math/derivatives/tableof.htm
	- $\frac{df(g(x))}{dx} = \frac{df(g(x))}{dg(x)} \frac{dg(x)}{dx}$ \\ % use definitions, \bar{f} = \delta f or f, \bar{g} = \delta g or g to catch undefined cases 
	- $\frac{d(f^{-1}(x))}{dx} = \frac{1}{\frac{df(x)}{dx}(f^{-1}(x))}$ \\
	- $ln(e^x) = e^{ln(x)} = x$ \\
	- $\lim_{h \to 0} (\frac{e^h - 1}{h} = 1)$ \\
	- $\frac{d(x^{-1})}{dx} = ln|x|$ \\ % <From definition of e???>
	- $\frac{d(e^x)}{dx} = e^x$ \\
	- $\frac{d(x^c)}{dx} = c x^{c-1}$ \\
	===================================================================================================================
\section{Integrals}


\begin{comment}
TODO:
- maybe dont abbreviate
-\txt formatting
-definitions should be iff
	===================================================================================================================

Hotkeys:
Ctrl+R: find reference
Ctrl+K, Ctrl+1: code fold all
Ctrl+K, Ctrl+J: code unfold all
Ctrl+Shift+[: code fold
Ctrl+Shift+]: code unfold
Ctrl+Shift+Right-click+Drag: block select
Ctrl+Space: display auto-complete
Tab: tab trigger auto-complete
Shift+Tab: force tab

=== math.sublime-completions ===
{ "completions": [

{ "trigger": "fl\t\\forall", "contents": "\\forall " },
{ "trigger": "es\t\\exists", "contents": "\\exists " },
{ "trigger": "ev\t\\equiv", "contents": "\\equiv " },
{ "trigger": "es\t\\implies", "contents": "\\implies " },
{ "trigger": "if\t\\iff", "contents": "\\iff " },
{ "trigger": "and\t\\land", "contents": "\\land " },
{ "trigger": "or\t\\lor", "contents": "\\lor " },
{ "trigger": "not\t\\lnot", "contents": "\\lnot " },
{ "trigger": "dr\t\\vdash", "contents": "\\vdash " },
{ "trigger": "md\t\\vDash", "contents": "\\vDash " },
{ "trigger": "cl\t\\mathcal", "contents": "\\mathcal{}" },
{ "trigger": "fk\t\\mathfrak", "contents": "\\mathfrak{}" },
{ "trigger": "mb\t\\mathbb", "contents": "\\mathbb{}" },
{ "trigger": "nit\t\\inot", "contents": "\\inot " },
{ "trigger": "cr\t\\contr", "contents": "\\contr " },
{ "trigger": "phr\t\\placeholder", "contents": "\\placeholder" },

{ "trigger": "vdc\t\\vdc{_i}{i=1}{Arity()}", "contents": "\\vdc{_i}{i=1}{Arity()}" },
{ "trigger": "sb\t\\sub{}{}{}", "contents": "\\sub{}{}{}" },
{ "trigger": "sq\t\\subseteq", "contents": "\\subseteq " },
{ "trigger": "cp\t\\cup", "contents": "\\cup " },
{ "trigger": "rw\t\\rightarrow", "contents": "\\rightarrow " },

{ "trigger": "a\t\\alpha", "contents": "\\alpha " },
{ "trigger": "b\t\\beta", "contents": "\\beta " },
{ "trigger": "g\t\\gamma", "contents": "\\gamma " },
{ "trigger": "f\t\\phi", "contents": "\\phi " },
{ "trigger": "s\t\\psi", "contents": "\\psi " },
{ "trigger": "m\t\\sigma", "contents": "\\sigma " },
{ "trigger": "t\t\\theta", "contents": "\\theta " },

{ "trigger": "A\t\\Alpha", "contents": "\\Alpha " },
{ "trigger": "B\t\\Beta", "contents": "\\Beta " },
{ "trigger": "G\t\\Gamma", "contents": "\\Gamma " },
{ "trigger": "F\t\\Phi", "contents": "\\Phi " },
{ "trigger": "S\t\\Psi", "contents": "\\Psi " },
{ "trigger": "M\t\\Sigma", "contents": "\\Sigma " },
{ "trigger": "T\t\\Theta", "contents": "\\Theta " },

] }
=== math.sublime-completions ===

\end{comment}

\end{document}


