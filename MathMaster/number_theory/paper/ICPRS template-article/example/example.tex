% File example.tex
% Contact: simonnet@ecole.ensicaen.fr
%
% version 1.0 (July 24, 2009)
% version 1.1 (September 3, 2009)
% add using of the optional command: \secondauthoraddr

\documentclass[10pt]{article}

% File icdp2009.sty
% Preamble that you have to include to use the template  

% July 24, 2009
% Contact: simonnet@ecole.ensicaen.fr


\usepackage[a4paper,textwidth=18cm,textheight=24cm,top=2.85cm, bottom=2.85cm, left=1.5cm, right=1.5cm]{geometry}

\usepackage{../template/icdp2009}

% left justified caption
\makeatletter
\long\def\@makecaption#1#2{%
\vskip\abovecaptionskip
\sbox\@tempboxa{#1. #2}%
\ifdim \wd\@tempboxa >\hsize
#1. #2\par
\else
\global \@minipagefalse
\hb@xt@\hsize{\box\@tempboxa\hfil}%
\fi
\vskip\belowcaptionskip}
\makeatother




%other package

% vectorial font
\usepackage{lmodern}
\usepackage{graphicx}
\usepackage{times}
\usepackage{amsmath}

\newtheorem{theorem}{Theorem}[section]
\newtheorem{corollary}{Corollary}[theorem]
\newtheorem{lemma}[theorem]{Lemma}

\begin{document}
\noindent

% This should produce references in the order they appear
\bibliographystyle{ieeetr}

\title{The RSA Encryption Algorithm}

\authorname{John Paul S. Guzman}
\authoraddr{Mathematics Department, De La Salle University, 2401 Taft Ave., Manila 0922, Philippines}

\maketitle

% TODO:
% - signatures 

\abstract
Encryption is the process of scrambling information in such a way that it can only be read by select individuals. It is an essential tool as we rely on the internet in our activities such as online banking, transactions, and private messaging. In these activities, we are often required to communicate sensitive information like passwords, credit card information, social security number, and home addresses. Encryption prevents unwanted individuals from having access to these types of information. % WHERE IS RSA IN THIS ABSTRACT

\keywords
Cryptography, Internet security, Prime numbers.

\section{Introduction}


The RSA encryption algorithm is a widely used encryption algorithm that secures sensitive data from “man-in-the-middle“ attacks wherein malicious third-party eavesdrops on the communication between two parties with the intent of stealing sensitive information \cite{???}. It is one of the earliest implementations of public-key or asymmetric encryption which allows users to establish a private communication over a public channel \cite{???}. This is accomplished by having separate keys for encryption (public key) and decryption (private key). On the other hand, symmetric encryption uses the same key for encryption and decryption.

Another key feature of RSA is the ability to create electronic signatures. Similar to a physical signature, an electronic signature provides a way to prove that a message originated from the sender \cite{???}. It also provides the means to show that the message was not tampered with.

RSA uses Euler’s theorem to: (a) scramble or encrypt a message into ciphertext, (b) unscramble or decrypt ciphertext back to the original message, (c) generate electronic signatures.


\section{Cryptographic Operations}
In order to create the public key and private key, we need to choose two large distinct prime numbers $p$ and $q$ at random. Let $n$ be the product of $p$ and $q$. Next, pick a large random integer $d$ that is relatively prime to $\phi(n)$ where $\phi$ is the Euler totient function. Finally, we choose $e$ to be the multiplicative inverse of $d$ modulo $\phi(n)$.
\begin{equation*}
	e d \equiv 1 \mod \phi(n).
\end{equation*}

We know that such an $e$ is exists since $d$ is relatively prime to $\phi(n)$. The pair $(e, n)$ will serve as the public key, while the pair $(d, n)$ will serve as the private key.

Let $M$ be a plaintext message encoded as a number. We insist that $M < n$. In the case where $M \geq n$, we will split $M$ into multiple chunks and send them separately. Furthermore, we insist that $M$ is relatively prime to $n$. In practice, this is a safe assumption since it is unlikely that $M$ is a multiple of $p$ or $q$. Regardless, we could easily check for this case since we have access to $p$ and $q$. If so, we can simply split $M$ further or add some padding until it becomes relatively prime to $n$.

We generate the ciphertext $C$ by raising $M$ to $e$th power then taking modulo $n$. We can decipher $C$ to the plaintext $D$ by raising $C$ to the $d$th power then taking modulo $n$. Lastly, we can create the electronic signature $S$ by raising $M$ to the $d$th power then taking modulo $n$.
\begin{align*}
    C \equiv M^e \mod n.
 \\ D \equiv C^d \mod n.
 \\ S \equiv M^d \mod n.
\end{align*}


\section{Mathematical Foundations}
To illustrate the correctness of these of operations, we first need to discuss some results in number theory.

\begin{theorem}
\label{cong_thms}
    If $a \equiv b \mod n$, then $c a \equiv c b \mod n$ and $a^k \equiv b^k \mod n$ for any positive integer $k$.
\end{theorem}

\begin{theorem}
    If $a$ and $b$ are relatively prime with $a \mid c$ and $b  \mid c$, then $a b \mid c$.
\end{theorem}

\begin{corollary}
\label{cong_rel-prime}
    If $m$ and $n$ are relatively prime with $a \equiv b \mod m$ and $a \equiv b \mod n$, then $a \equiv b \mod m n$.
\end{corollary}

\begin{theorem}
\label{phi_prime}
    If $p$ is prime, then $\phi(p) = p - 1$.
\end{theorem}

\begin{theorem}
\label{phi_mult}
    If $a$ and $b$ are relatively prime, then $\phi(a b) = \phi(a) \phi(b)$.
\end{theorem}

\begin{theorem}[Euler, 1763 \cite{???}]
\label{euler}
    If $a$ is relatively prime to $b$, then $a^{\phi(b)} \equiv 1 \mod b$.
\end{theorem}

We can now show that decrypting the ciphertext does indeed yield the original message. By the choice of $e$ and $d$,
\begin{equation}
\label{ed_id}
    e d = k \phi(n) + 1 \text{ for some integer }k.
\end{equation}

\noindent Since $p$ and $q$ are distinct primes, we can use \ref{phi_prime} and \ref{phi_mult} to obtain
\begin{equation}
    \phi(n) = \phi(p q) = \phi(p) \phi(q) = (p - 1) (q - 1).
\end{equation}

\noindent Since $M$ and $p$ are relatively prime, we can apply \ref{euler} to obtain
\begin{equation}
    M^{\phi(p)} \equiv M^{p - 1} \equiv 1 \mod p.
\end{equation}

\noindent Furthermore, by \ref{cong_thms},
\begin{align}
    M^{k \phi(n)} \equiv (M^{p - 1})^{k (q - 1)} \equiv 1^{k (q - 1)} \equiv 1 \mod p.
 \\ \label{Mmodp}
    M^{k \phi(n) + 1} \equiv M (1) \equiv M \mod p.
\end{align}

\noindent We can apply similar arguments for $q$,
\begin{equation}
    \label{Mmodq}
    M^{k \phi(n) + 1} \equiv M \mod q.
\end{equation}

\noindent Since $p$ and $q$ are distinct primes, we can apply \ref{cong_rel-prime} to \ref{Mmodp} and \ref{Mmodq}.
\begin{equation}
    \label{Mmodn}
    M^{k \phi(n) + 1} \equiv M \mod n.
\end{equation}

\noindent The desired result is obtained by \ref{ed_id} and \ref{Mmodn}.
\begin{equation}
    \label{CD_id}
    D \equiv C^d \equiv (M^e)^d  \equiv M^{e d} \equiv M^{k \phi(n) + 1} \equiv M \mod n.
\end{equation}


\section{Private Communication}
Let us consider the following scenario. Suppose that Alice wants to send a message $M$ to Bob. Bob will generate his key pair $(e_B, n_B)$ and $(d_B, n_B)$, then publish the only public key. Alice would take Bob’s public key, use it to create the ciphertext $C \equiv M^{e_B} \mod n_B$, and send it over. Bob can then recover the original message by $M \equiv C^{d_B} \mod n_B$ as in \ref{CD_id}. Now suppose Eve was eavesdropping in their conversation. This means that she would have a copy of $C$ and the public key $(e_B, n_B)$. However, she will be unable to decipher $C$ since she does not have access to $d_B$. It has been conjectured in \cite{???} that finding $d_B$ from $(e_B, n_B)$ requires an exhaustive search, and so it is unfeasible for a sufficiently large value of $n_B$.

Hence, we are successful in establishing a private conversation over a public channel. This bypasses one of the biggest problems in symmetric encryption, namely, key distribution. This issue is due to the fact that two parties have to agree on which keys to use for encryption and decryption. If Alice and Bob were to use symmetric encryption, then Eve can make a copy of the key during the exchange and use it to decipher future communication.


\section{Electronic Signatures}


\section{Conclusions}


\bibliography{IEEEabrv, icdp2009}

\end{document}
