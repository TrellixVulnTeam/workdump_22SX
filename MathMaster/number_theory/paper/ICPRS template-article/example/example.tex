% File example.tex
% Contact: simonnet@ecole.ensicaen.fr
%
% version 1.0 (July 24, 2009)
% version 1.1 (September 3, 2009)
% add using of the optional command: \secondauthoraddr

\documentclass[10pt]{article}

\input{../template/preamble.tex}

%other package

% vectorial font
\usepackage{lmodern}
\usepackage{graphicx}
\usepackage{times}
\usepackage{amsmath}

\newtheorem{theorem}{Theorem}[section]
\newtheorem{corollary}{Corollary}[theorem]
\newtheorem{lemma}[theorem]{Lemma}

\begin{document}
\noindent

% This should produce references in the order they appear
\bibliographystyle{ieeetr}

\title{The RSA Encryption Algorithm}

\authorname{John Paul S. Guzman}
\authoraddr{Mathematics Department, De La Salle University, 2401 Taft Ave., Manila 0922, Philippines}

\maketitle

% TODO:
% - signatures 
% - How to solve for d with public data?
% -- It does this by having separate keys to do: discuss one way? trap doors?? security of factoring algos???

\abstract
Encryption is the process of scrambling information in such a way that it can only be read by select individuals. It is an essential tool as we rely on the internet in our 
 activities such as online banking, transactions, and private messaging. In these activities, we are often required to communicate sensitive information like passwords, credit card information, social security number, and home addresses. Encryption prevents unwanted individuals from having access to these types of information.

\keywords
Cryptography, Internet security, Prime numbers.

\section{Introduction}
The RSA encryption algorithm is a widely used encryption algorithm that secures sensitive data from “man-in-the-middle“ attacks wherein malicious third-party eavesdrops on the communication between two parties with the intent of stealing sensitive information [1]. It is one of the earliest implementations of public-key or asymmetric encryption which allows users to establish a private communication over a public channel [5]. This is accomplished by having separate keys for encryption (public key) and decryption (private key).

Another key feature of RSA is the ability to create electronic signatures. Similar to a physical signature, an electronic signature provides a way to prove that a message originated from the sender [5]. It also provides the means to show that the message was not tampered with.

RSA uses Euler’s theorem to: (a) scramble or encrypt a message into ciphertext, (b) unscramble or decrypt ciphertext back to the original message, (c) generate electronic signatures.


\section{Cryptographic Operations}
In order to create the public key and private key, we need to choose two large distinct prime numbers $p$ and $q$ at random. Let $n$ be the product of $p$ and $q$. Next, pick a large random integer $d$ that is relatively prime to $\phi(n)$ where $\phi$ is the Euler totient function. Finally, we choose $e$ to be the multiplicative inverse of $d$ modulo $\phi(n)$.
\begin{equation*}
	e d \equiv 1 \mod \phi(n).
\end{equation*}

We know that such an $e$ is exists since $d$ is relatively prime to $\phi(n)$. The pair $(e, n)$ will serve as the public key, while the pair $(d, n)$ will serve as the private key.

Let $M$ be a plaintext message encoded as a number. We insist that $M < n$. In the case where $M \geq n$, we will split $M$ into multiple chunks and send them separately. Furthermore, we insist that $M$ is relatively prime to $n$. In practice, this is a safe assumption since it is unlikely that $M$ is a multiple of $p$ or $q$. Regardless, we could easily check for this case since we have access to $p$ and $q$. If so, we can simply split $M$ further or add some padding until it becomes relatively prime to $n$.

We generate the ciphertext $C$ by raising $M$ to $e$th power then taking modulo $n$. We can decipher $C$ to the plaintext $D$ by raising $C$ to the $d$th power then taking modulo $n$. Lastly, we can create the electronic signature $S$ by raising $M$ to the $d$th power then taking modulo $n$.
\begin{align*}
    C \equiv M^e \mod n.
 \\ D \equiv C^d \mod n.
 \\ S \equiv M^d \mod n.
\end{align*}


\section{Mathematical Foundations}
To illustrate the correctness of these of operations, we first need to discuss some results in number theory.

\begin{theorem}
\label{cong_thms}
    If $a \equiv b \mod n$, then $c a \equiv c b \mod n$ and $a^k \equiv b^k \mod n$ for any positive integer $k$.
\end{theorem}

\begin{theorem}
    If $a$ and $b$ are relatively prime with $a \mid c$ and $b  \mid c$, then $a b \mid c$.
\end{theorem}

\begin{corollary}
\label{cong_rel-prime}
    If $m$ and $n$ are relatively prime with $a \equiv b \mod m$ and $a \equiv b \mod n$, then $a \equiv b \mod m n$.
\end{corollary}

\begin{theorem}
\label{phi_prime}
    If $p$ is prime, then $\phi(p) = p - 1$.
\end{theorem}

\begin{theorem}
\label{phi_mult}
    If $a$ and $b$ are relatively prime, then $\phi(a b) = \phi(a) \phi(b)$.
\end{theorem}

\begin{theorem}[Euler, 1763]
\label{euler}
    If $a$ is relatively prime to $b$, then $a^{\phi(b)} \equiv 1 \mod b$.
\end{theorem}

We can now show that decrypting the ciphertext does indeed yield the original message. By the choice of $e$ and $d$,
\begin{equation}
\label{ed_id}
    e d = k \phi(n) + 1 \text{ for some integer }k.
\end{equation}

\noindent Since $p$ and $q$ are distinct primes, we can use \ref{phi_prime} and \ref{phi_mult} to obtain
\begin{equation}
    \phi(n) = \phi(p q) = \phi(p) \phi(q) = (p - 1) (q - 1).
\end{equation}

\noindent Since $M$ and $p$ are relatively prime, we can apply \ref{euler} to obtain
\begin{equation}
    M^{\phi(p)} \equiv M^{p - 1} \equiv 1 \mod p.
\end{equation}

\noindent Furthermore, by \ref{cong_thms},
\begin{align}
    M^{k \phi(n)} \equiv (M^{p - 1})^{k (q - 1)} \equiv 1^{k (q - 1)} \equiv 1 \mod p.
 \\ \label{Mmodp}
    M^{k \phi(n) + 1} \equiv M (1) \equiv M \mod p.
\end{align}

\noindent We can apply similar arguments for $q$,
\begin{equation}
    \label{Mmodq}
    M^{k \phi(n) + 1} \equiv M \mod q.
\end{equation}

\noindent Since $p$ and $q$ are distinct primes, we can apply \ref{cong_rel-prime} to \ref{Mmodp} and \ref{Mmodq}.
\begin{equation}
    \label{Mmodn}
    M^{k \phi(n) + 1} \equiv M \mod n.
\end{equation}

\noindent The desired result is obtained by \ref{ed_id} and \ref{Mmodn}.
\begin{equation}
    \label{CD_id}
    D \equiv C^d \equiv (M^e)^d  \equiv M^{e d} \equiv M^{k \phi(n) + 1} \equiv M \mod n.
\end{equation}


\section{Security}
Let us consider the following scenario. Suppose that Alice wants to send a message $M$ to Bob. Alice would take Bob’s public key $(e_B, n_B)$, use it to create the ciphertext $C \equiv M^{e_B} \mod n_B$, and send it over. Bob can then recover the original message by $M \equiv C^{d_B} \mod n_B$ as in \ref{CD_id}.

Suppose Eve was eavesdropping in their conversation. This means that she would have a copy of $e_B$, $n_B$, and $C$. She is unable to deciper $C$ since she does not have access to $d_B$.





\section{2Introduction}
222456 Papers for ICPRS-2021 need to be submitted for review by the \textbf{15 November 2020}. The submission should be the format described here and should be \textbf{anonymous}. If your paper is accepted, a final camera-ready non-anonymous version should be submitted, using the same electronic submission system, no later than the \textbf{29 January 2021}. Papers received after that date will not be included in the Proceedings. Your final version should be prepared taking into account the comments made by the reviewers and available to authors via the submission system. The Proceedings produced for ICPRS-2021 will contain \textbf{all} the papers accepted \textbf{and presented} in the conference.  

\section{2Manuscript preparation}
Full papers must be typed in English. This instruction page is
an example of the format and font sizes to be used. MS Word
users can download from the conference site these
instructions in Word format. LaTeX is preferred as it is easier to change paper style and formatting.

These are detailed instructions valid for any word
processor. In the title of the paper the initial letters should be
capitalised in all words except articles and prepositions (e.g.:
in, a, an, and, the, there, their, do, on, of, from, with, at etc.).
E.g. "ErDoped Si Nanocrystals as a Candidate for Optical
Amplification" The type should be boldface 18pt and centred
on the page. The authors' names (in the final non-anonymous version) are typed in capital and lower
case bold letters and centred on the page. Directly under the
authors' names in capital and lower case letters and also
centred are the authors' affiliation(s), address(es), plus email
address(es) of (at least) the corresponding author. Manuscripts must be
typed single spaced using 10 point characters. Only Times,
Times Roman, Times New Roman and Symbol fonts are
accepted. The text must fall within a frame of 18 cm x 24 cm
centred on an A4 page (21 cm x 29.7 cm).Paragraphs are
separated by 6 points and with no indentation. The text of the
full papers is written in two columns and justified. Each
column has a width of 8.8 cm and the columns are separated
by a margin of 0.4 cm. The maximum length of the full paper
is 6 pages (min 4 pages). \textbf{Do not number the pages and avoid the use of footnotes}. The final format in
which the papers will appear on the Proceedings will be a
PDF file. Authors are required to upload a \textbf{PDF} file of their
final paper to be included directly in the Proceedings. \textbf{All
PDF files should NOT be locked and all fonts and
graphics should be embedded}.

\subsection{2Figures and tables}
Figures and tables should be centred in the column, numbered
consecutively throughout the text, and each should have a
caption underneath it (see for example Table 1). Care should
be taken that the lettering is not too small. All figures and
tables should be included in the electronic versions of the full
paper. We cannot guarantee that any printed version of the
proceedings will use colour.


\begin{figure}[h]
\centering
\includegraphics[width=1.5cm]{fig1}
\caption{\label{tab1}This is an example of a figure caption.} 
\end{figure}


\begin{table}[h]
\begin{center}

\begin{tabular}{c}
nn!1 \\
2 \\
31 \\
6 \\
\end{tabular}
\end{center}
\caption{\label{tab1}This is an example of a table caption.}
\end{table}

\subsection{2Equations}
Equations should be typed within the text, centred, and should
be numbered consecutively throughout the text. They should
be referred to in the text as Equation (n). Their numbers
should be typed in parentheses, flush right, as in the following
example.
\begin{equation}
	    PA + A'P - PBR^{-1}B'P + Q  =  0 \enspace.
\end{equation}

\section{Generating a {PDF} file}
The PDF format will be the final format under which the
papers will appear in the Proceedings. Therefore you are
required to submit your paper as a PDF document. If this is not
possible, Postscript format is also accepted as long as no fonts
other than the recommended fonts are used.

You can use any of the popular free LaTeX editors (e.g. Kile, TexMaker, etc).

\section{2Electronic submission of the full paper}
The submission process for ICPRS 2021 should be done on
line at http://www.icprs.org

A PDF version of your final paper is required. It should
be expected that after your submission, your paper is
published directly from the file you send without any further
proofreading. Therefore, it is advisable for the authors to
print a hard copy of their final version and read it carefully.

Note that the publisher reserves the right not to publish a paper that is deemed to be poorly formatted or with poor use of English.

\section{2Your References}
The list of references should be ordered in the same order as
first cited in the text. All references should be cited in the
text, and using square brackets such as \cite{ref01} and \cite{ref01,ref02}. We
recommend the use of IEEE Transactions style for references. \textit{Avoid any references
that could identify any of the authors, e.g. avoid "as we showed in ..."}

\section*{2Acknowledgements}
The acknowledgement for funding organisations etc. should
be placed in a separate section at the end of the text.



Thank you for your cooperation in complying with these
instructions.


\bibliography{IEEEabrv, icdp2009}

\end{document}
