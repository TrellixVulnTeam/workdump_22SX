%1\pr{23(45)678}9
\documentclass{book}
\usepackage[utf8]{inputenc}
\usepackage[english]{babel}
\usepackage[margin=0.3in]{geometry} % for margins

\usepackage{hyperref}
\hypersetup{
    colorlinks=true,
    linkcolor=blue,
    citecolor=green
    filecolor=cyan,
    urlcolor=magenta,
}

\urlstyle{same}

\usepackage{enumitem}
\setlist[enumerate]{align=left, %, label*=\arabic*.}
  leftmargin=3.0mm, %
  labelsep=3.0mm, %
  rightmargin=0.0mm, %
  listparindent=0.0mm, %
  labelwidth=0.0mm, %
  itemindent=0.0mm, %
}
\setlist[enumerate,1]{label=(\arabic*)}
\setlist[enumerate,2]{label=(\arabic{enumi}.\arabic*)}
\setlist[enumerate,3]{label=(\arabic{enumi}.\arabic{enumii}.\arabic*)}
\setlist[enumerate,4]{label=(\arabic{enumi}.\arabic{enumii}.\arabic{enumiii}.\arabic*)}
\setlist[enumerate,5]{label=(\arabic{enumi}.\arabic{enumii}.\arabic{enumiii}.\arabic{enumiv}.\arabic*)}
\setlist[enumerate,6]{label=(\arabic{enumi}.\arabic{enumii}.\arabic{enumiii}.\arabic{enumiv}.\arabic{enumv}.\arabic*)}
\setlist[enumerate,7]{label=(\arabic{enumi}.\arabic{enumii}.\arabic{enumiii}.\arabic{enumiv}.\arabic{enumv}.\arabic{enumvi}.\arabic*)}
\setlist[enumerate,8]{label=(\arabic{enumi}.\arabic{enumii}.\arabic{enumiii}.\arabic{enumiv}.\arabic{enumv}.\arabic{enumvi}.\arabic{enumvii}.\arabic*)}

\setlist{nolistsep,after=}
\setlistdepth{8}
\renewlist{itemize}{itemize}{8}
\renewlist{enumerate}{enumerate}{8}

\usepackage{amsmath}
\usepackage{bbold}
\usepackage{stix}
\usepackage{verbatim}

\usepackage{mleftright} % for left right
\mleftright
\delimitershortfall-1sp

\usepackage{xcolor}
\usepackage{framed}
\colorlet{shadecolor}{black!100} % \begin{shaded}\end{shaded}

\usepackage{silence}
\WarningFilter*{amsmath}{Foreign command}

%\usepackage{ocgx}

\newcommand{\df}[1]{\hypertarget{#1}{\textcolor{red}{#1}}}
\newcommand{\rf}[1]{\hyperlink{#1}{\textcolor{cyan}{#1}}}
%\newcommand{\df}[1]{#1}
%\newcommand{\rf}[1]{#1}
\newcommand{\abr}{:=}
\newcommand{\cont}{\phantom{.}. . .\phantom{.}}
\newcommand{\pipe}{$\phantom{(}\vrectangleblack\phantom{)}$}
\newcommand{\pr}[1]{\left(#1\right)}
\newcommand{\thmcontext}[1]{\centerline{\\************************ #1 ************************ \\}}


\usepackage{xifthen}% provides \isempty test
\newcommand{\ann}[2]{%
  \hfill %
  $\stackrel%
  {\hbox{\tiny{\ifthenelse{\isempty{#1}}%
    {\phantom{from}}%
    {from: $#1$}}}}%
  {\hbox{\tiny{\ifthenelse{\isempty{#2}}%
    {\phantom{wts}}%
    {wts: $#2$}}}}$%
\ }

\newcommand{\setbackgroundcolour}{\pagecolor[rgb]{0.2,0.2,0.2}}
\newcommand{\settextcolour}{\color[rgb]{0.8,0.8,0.8}}
\newcommand{\invertbackgroundtext}{\setbackgroundcolour\settextcolour}
\newcommand{\bbin}[1]{\mathbin{{\bar{#1}}}}
\newcommand{\st}{\mathbin{|}}
\newcommand{\utup}[1]{\{#1\}}
\newcommand{\notation}[2]{\fbox{{[Notation] \phantom{-} $#1 \abr #2$}}}
\newcommand{\floor}[1]{\left\lfloor #1 \right\rfloor}
\newcommand{\ceil}[1]{\left\lceil #1 \right\rceil}

%\usepackage[author={Max Schlepzig}]{pdfcomment}
%Here we\pdfcomment[icon=Insert]{insert: miss} a word!

\newenvironment{lit}{\item \vspace{.75mm} \hrule \vspace{.75mm}}

% "sumatra": "C:\\Users\\guzma\\Downloads\\PortableInstallations\\PortableGit\\usr\\bin\\echo.exe", "keep_focus_delay": 0
\begin{document}

\invertbackgroundtext
\setlength{\parindent}{0pt}

%\tableofcontents
%\chapter{Problem Set 1}
\section{Problem Set 1}
1.) Prove Proposition 6: For any $a, b \in \mathbb{R}$,
$\left(\begin{array}{ll}
    a \lor b = \frac{1}{2}(a + b + |a - b|)
\\  a \land b = \frac{1}{2}(a + b - |a - b|)
\end{array}\right)$.

Let $a, b \in \mathbb{R}$. Suppose $a \geq b$. Clearly, $a \lor b = a$ and $a \land b = b$. Furthermore, $|a - b| = a - b$ since $a - b \geq 0$. Thus, 
\begin{align}
    a \lor b = a = \frac{1}{2}\pr{a + b + (a - b)} = \frac{1}{2}(a + b + |a - b|)
\\  a \land b = b = \frac{1}{2}\pr{a + b - (a - b)} = \frac{1}{2}(a + b - |a - b|).
\end{align}

Suppose $a < b$. Clearly, $a \lor b = b$ and $a \land b = a$. Furthermore, $|a - b| = -(a - b)$ since $a - b < 0$. Thus, 
\begin{align}
    a \lor b = b = \frac{1}{2}\pr{a + b + \pr{-(a - b)}} = \frac{1}{2}(a + b + |a - b|)
\\  a \land b = a = \frac{1}{2}\pr{a + b - \pr{-(a - b)}} = \frac{1}{2}(a + b - |a - b|).
\end{align}

In either case, $a \lor b = \frac{1}{2}(a + b + |a - b|)$ and $a \land b = \frac{1}{2}(a + b - |a - b|)$ holds.
$\hfill \blacksquare$ \\ \\


2.) Prove Proposition 7: For any $a, b, r \in \mathbb{R}$,
$\left(\begin{array}{ll}
    a \lor b = b \lor a
\\  a \land b = b \land a
\\  (a \land b \leq r \leq a \lor b) \implies \pr{(|r - a| \leq |a - b|) \land (r - b \leq |a - b|)}
\end{array}\right)$. \\

Let $a, b, r \in \mathbb{R}$. The first two statements immediately follow by applying the commutativity of real numbers and $|a - b| = |-(a - b)| = |b - a|$ to Proposition 6. \\

Suppose $a \land b \leq r \leq a \lor b$. Without loss of generality, let $a \geq b$. Thus, 
\begin{align}
    b \leq r \leq a
\\  r - a \leq 0
\\  b - r \leq 0
\\  b - a \leq 0
\end{align}

From (5), $r - a \geq b - a$. This along with (6) and (8) implies $|r - a| = -(r - a) \leq -(b - a) = |b - a|$.

From (5), $r - b \leq a - b$. This along with (7) and (8) implies $|r - b| = r - b \leq a - b = |a - b|$.
$\hfill \blacksquare$ \\ \\

%%%%%%%%%%%%%%%%%%%%%%%%%%%%%%%%%%%%%%%%%%%%%%%%%%%%%%%%%%
%%%%%%%%%%%%%%%%%%%%%%%%%%%%%%%%%%%%%%%%%%%%%%%%%%%%%%%%%%
%%%%%%%%%%%%%%%%%%%%%%%%%%%%%%%%%%%%%%%%%%%%%%%%%%%%%%%%%%

% \newpage

%%%%%%%%%%%%%%%%%%%%%%%%%%%%%%%%%%%%%%%%%%%%%%%%%%%%%%%%%%
%%%%%%%%%%%%%%%%%%%%%%%%%%%%%%%%%%%%%%%%%%%%%%%%%%%%%%%%%%
%%%%%%%%%%%%%%%%%%%%%%%%%%%%%%%%%%%%%%%%%%%%%%%%%%%%%%%%%%

%%%%%%%%%%%%%%%%%%%%%%%%%%%%%%%%%%%%%%%%%%%%%%%%%%%%%%%%%%
%%%%%%%%%%%%%%%%%%%%%%%%%%%%%%%%%%%%%%%%%%%%%%%%%%%%%%%%%%
%%%%%%%%%%%%%%%%%%%%%%%%%%%%%%%%%%%%%%%%%%%%%%%%%%%%%%%%%%


%%%%%%%%%%%%%%%%%%%%%%%%%%%%%%%%%%%%%%%%%%%%%%%%%%%%%%%%%%%%%%%%%%%%%%%%%%%%%%%%%%%%%%%%%%%%%%%%%%%%%%%%%%%%%%%%%%%%%%%%%%%%%%%%%%%%%%%%%%%%%%%%%%%%%%%%%%%%%%%%%%%%%%%%%%%%%%%%%%%%%%%%%%%%%%%%%%%%%%%%%%%%%%%%%%%%%%%%%%%%%%%%%%%%%%%%%%%%%%%%%%%%%%%%%%%%%%%%%%%%%%%%%%%%%%%%%%%%%%%%%%%%%%%%%%%%%%%%%%%%%%%%%%%%%%%%%%%%%%%%%%%%%%%%%%%%%%%%%%%%%%%%%%%%
\end{document}

%############################################################################
% \left(\begin{array}{ll}
%   XXX \hfill \land \hfill \\
%   YYY \hfill \land \hfill \\
% \end{array}\right)
%############################################################################
